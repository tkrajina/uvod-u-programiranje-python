\chapter{Izrazi}

\section{Aritmetički izrazi}

Aritmetički izrazi su matematički izrazi s kakvima se računa u
osnovnoj školi. Najčešće se sastoje od brojeva ili varijabli
koje imaju brojčanu vrijednost i matematičkih operacija. Primjer
aritmetičkog izraza moze biti: 
$2+4-2$, 
$\displaystyle \frac{3+6\cdot 5}{7-4}$,
ili 
$\displaystyle \frac{s_2-s_1}{t_2-t_1}$. 

Kad nam negdje u programu zatreba
aritmetički izraz zapisujemo ga na sličan način kako bismo ga napisali u
biljeznici s nekoliko sitnih razlika;

\begin{itemize}
	\item Mnozenje zapisujemo pomoću znaka *, a ne $\cdot$.
	\item Dijeljenje zapisujemo pomoću znaka / umjesto :
	\item Razlomke zapisujemo pomoću operacije dijeljenja.
	\item Potencije zapisujemo pomoću "**". Dakle $3^2$ bi zapisalo kao
		3**2
	\item Ostatak\footnote{"Modulo"} pri dijeljenju dobijemo pomoću
		operacije \%.
\end{itemize}

Izraz 
$\displaystyle \frac{3+6\cdot 5}{7-4}$
bi napisali: \verb+(3+6*5)/(7-4)+. U ovom
slučaju brojnik i nazivnik treba staviti unutar zagrada jer bi u
slučaju da je izraz \verb+3+6*5/7-4+ kompjuter pokušao prvo
izračunati \verb+6*5/7+, python naime izraze računa pazeći na
prednost računskih operacija (npr. mnozenje i dijeljenje imaju
prednost pred zbrajanjem i oduzimanjem).

\textbf{Vazno:}
Ima još jedna stvar na koju treba pripaziti pri pisanju algebarskih
izraza; ukoliko su brojevi s kojima računamo cjelobrojni onda će (u
programskom jeziku python) i rezultat biti cjelobrojan. Dakle, ako probate
izračunati \verb+13/4+ dobiti ćete \verb+3+, a ne \verb+3.25+! To se
moze riješiti tako da barem jedan od brojeva definiramo kao realan, a za
to je dovoljno dodati mu decimalnu točku na kraju. Da bi dobili točan
rezultat dijeljenje 13 podijeljeno s 4 trebali bi dakle napisati
\verb+13/4+.

Zadaci:

\section{Logički izrazi}

Slično kao aritmetički izrazi logički izrazi se sastoje od
operacija i članova izraza nad kojima ze izvršavaju te operacije.
Kod aritmetičkih izraza članovi su brojevi ili varijable s
brojevnom vrijednošću, a članovi logičkih izraza mogu biti
\emph{sudovi} ili čak drugi aritmetički izrazi.

\emph{Sud} je tvrdnja koja moze biti istinita ili lazna. Primjer
suda je \emph{"Zemlja kruzi oko Mjeseca"} ili \emph{"Postoji
beskonačno mnogo prirodnih brojeva"}.  Svaki sud mora imati svoju
istinosnu vrijednost koja moze biti \emph{"istina"} ili \emph{"laz"}.
Ukoliko za neku tvrdnju ne mozemo sa sigurnošću kazati je li
istinita ili lazna tada to nije sud. Na primjer \emph{"Zemlja
kruzi oko Mjeseca"} jest sud zato što ima istinosnu vrijednost
\emph{"laz"}, kao i \emph{"Postoji beskonačno mnogo prirodnih
brojeva"} čija je istinita vrijednost \emph{"istina"}.  Tvrdnja
\emph{"Frank Sinatra pjeva bolje od Tine Turner"} nije sud jer je
nemoguće odrediti istinitost te tvrdnje budući da je ona čisto
subjektivne prirode (nekome se više svidja Sinatra, a nekome Tina
Turner). Isto tako nije sud "Učini to!" ili "Mozda ću jednog
dana naučiti programirati".

Umjesto "istina" ili "laz" se često koriste velika slova "T" (od
engleskog "true" = "istina") odnosno "F" (eng. "false" = "laz").

Sud mozemo zapisati i matematičkim simbolima: $1<2$ je sud u
kojemu se tvrti da je jedan manji od 2, a istinosna vrijednost tog
suda je T (= istina). Matematički sudovi kojima se koristimo u
programiranju najčešće se koriste za opisivanje odnosa izmedju
brojeva. Pri tome se koristimo sljedećim simbolima iz sljedeće tablice;
u prvom stupcu se nalazi simbol kako bismo za zapisali u biljeznicu
ili na ploču, u drugom stupcu zapisa tog istog simbola u programu,
a u trećem kako čitamo taj simbol:

\begin{tabular}{lll}
	$=$ & $==$ & je jednako \\
	$\neq$ & $!=$ & nije jednako, je različito \\
	$<$ & $<$ & je manje od \\
	$\leq$ & $<=$ & je manje ili jednako \\
	$>$ & $>$ & je veće od \\
	$\geq$ & $>=$ & je veće ili jednako
\end{tabular}

Probajmo sada utvrditi istinosnu vrijednost nekih matematičkih izraza:

\begin{tabular}{lll}
	$12<12.01$ & \verb+12<12.01+ & T\\
	$1+2\leq 5$ & \verb"1+2<=5" & T\\
	$10-3\geq 6+1$ & \verb"10-3>=6+1" & T \\
	$10-2\geq 6+1$ & \verb"10-2>=6+1" & F \\
	$5\neq 5$ & \verb"5!=5" & F
\end{tabular}

Gornji primjeri su primjeri \emph{jednostavnih sudova}. Slozeni
sudovi su sudovi koji se dobijaju od jednostavnih sudova i logičkih
operacija \emph{and}, \emph{or} i \emph{not}.

\subsection{Logička operacija \emph{and}}

Promotrimo rečenicu \emph{"Ako je lijepo vrijeme
idemo na izlet."}. O čemu ovisi oćete li
otići na izlet? Ovisi o tome je li lijepo vrijeme, dakle ovisi o
istinosnoj vrijednosti suda \emph{"Lijepo je vrijeme."}. Ako je
taj sud istinit (T) otići ćete na izlet, a ako nije (F) -- ništa
od izleta.

Malo ćemo stvar zakomplicirati ako osim lijepog vremena vaš izleta
ovisi o još nečemu; \emph{"Ako je lijepo vrijeme i nemam drugih obaveza
ići ću na izlet"}.
Sad vaš izlet ovisi o istinitosti suda \emph{"Lijepo je vrijeme i nemam obaveza"}, a on je istinit
kad su istovremeno istinita sljedeća dva suda:

\begin{itemize}
	\item[\emph{(a)}] \emph{"Lijepo je vrijeme."}
	\item[\emph{(b)}] \emph{"Nemam obaveza."}
\end{itemize}

Dakle, treba vrijediti da je istinito i \emph{(a)} i \emph{(b)},
jer ako je bilo koje od ta dva laz onda je i tvrdnja \emph{"Lijepo
je vrijeme i nemam obaveza"} lazna.

Ukoliko imamo dva suda koja ćemo ovdje označiti s $A$, odnosno
$B$ onda ćemo takvu kombinaciju zapisivati s $A and B$. Suda $A
and B$ je \emph{slozeni sud} koji se sastoji od jednostavnijih
sudova $A$ i $B$. Istinitost suda $A and B$ ovisi o istinitosti
sudova $A$ i $B$; tek ako su oba istinita onda je i $A and B$
istinit. To se moze prikazati pomoću sljedeće tablice:

\begin{tabular}{ll|l}
	$A$ & $B$ & $A and B$ \\
	\hline
	istina & istina & istina \\
	istina & laz & laz \\
	laz & istina & laz \\
	laz & laz & laz 
\end{tabular}

Garnju tablicu zovemo \emph{tablica istinitosti} logičke operacije $and$.

\subsection{Logička operacija \emph{or}}

Pretpostavimo da ćete otići na izlet ako vrijedi \emph{"Lijepo
je vrijeme ili imam dobro društvo"}. Dakle, ako je lijepo vrijeme
idete na izlet, ako nije lijepo vrijeme i imate dobro društvo ipak
idete na izlet, ako nemate dobro društvo i lijepo je vrijeme opet
idete na izlet, a jedini slučaj kad ne idete na izlet je kad, niti
je vrijeme lijepo niti imate dobro društvo.

Tablica istinitosti logičke operacije $or$ izgleda ovako:

\begin{tabular}{ll|l}
	$A$ & $B$ & $A or B$ \\
	\hline
	istina & istina & istina \\
	istina & laz & istina \\
	laz & istina & istina \\
	laz & laz & laz 
\end{tabular}

\subsection{Logička operacija \emph{not}}

U zadnjem slučaju otići ćete na izlet tek ako vrijedi \emph{"Nemam
drugih obaveza"}. Dakle, tek ako \emph{nije} istit sud \emph{"Imam
drugih obaveza"}. Logička operacija $not$ nekom sudu pridodaje
suprotanu istinosnu vrijednost.

\begin{tabular}{l|l}
	$A$ & $not A$ \\
	\hline
	istina & laz\\
	laz & istina
\end{tabular}

Zadaci: Probajte odrediti istinitost sljedećih sudova:
	- 1<2 and 13!=5
	- 2>5 or 1=2
	- 1<2 and 5!=5
	- ( 1<2 or 9=5 ) and 3==3
	- ( 2==2 and 3!=5 ) or ( 3==4 )

\subsection{Nekoliko dodatnih pravila}

U programskom jeziku python (slično kao u mnogim drugim) postoji
još nekoliko dodadnih pravila kod utvrdjivanjaistinitosti sudova:
- svaki broj/izraz je po definiciji sud:
	- ako njegova vrijednost 0 onda je njegova istinosna vrijednost "laz" (F)
	- ako je različit od 0 onda je njegova istinosna vrijednost "istina" (T)
- svaki string je po definiciji sud:
	- ako je string prazan onda je njegova istinosna vrijednost "laz" (F)
	- ako string nije prazan onda je njegova istinosna vrijednost "istina" (T)

dakl ima smisla sud \verb"2<3 or 0". Budući da je $2<3$ ondaje prvi dio suda istinit, a (broj) 0 je
po gornjim pravilima lazan, dakle imamo slučaj "istina" $or$ "laz", dakle rezultat je istina.

Zadaci: Probajte odrediti istinitost sljedećih sudova:
\begin{itemize}
\item \verb"1 or 3>3"
\item \verb"2!=3 and 1"
\item \verb+"" or "jkljkl"+
\item \verb+( 0 or "jkljkl" ) and 2<3+
\item \verb+( not "jkljkl" ) and ( not 12 )+
\end{itemize}

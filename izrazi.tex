\chapter{Izrazi}

\section{Aritmeti\v{c}ki izrazi}

Aritmeti\v{c}ki izrazi su matemati\v{c}ki izrazi s kakvima se ra\v{c}una u
osnovnoj \v{s}koli. Naj\v{c}e\v{s}\'{c}e se sastoje od brojeva ili varijabli
koje imaju broj\v{c}anu vrijednost i matemati\v{c}kih operacija. Primjer
aritmeti\v{c}kog izraza mo\v{z}e biti: 
$2+4-2$, 
$\displaystyle \frac{3+6\cdot 5}{7-4}$,
ili 
$\displaystyle \frac{s_2-s_1}{t_2-t_1}$. 

Kad nam negdje u programu zatreba
aritmeti\v{c}ki izraz zapisujemo ga na sli\v{c}an na\v{c}in kako bismo ga napisali u
bilje\v{z}nici s nekoliko sitnih razlika;

\begin{itemize}
	\item Mno\v{z}enje zapisujemo pomo\'{c}u znaka *, a ne $\cdot$.
	\item Dijeljenje zapisujemo pomo\'{c}u znaka / umjesto :
	\item Razlomke zapisujemo pomo\'{c}u operacije dijeljenja.
	\item Potencije zapisujemo pomo\'{c}u "**". Dakle $3^2$ bi zapisalo kao
		3**2
	\item Ostatak\footnote{"Modulo"} pri dijeljenju dobijemo pomo\'{c}u
		operacije \%.
\end{itemize}

Izraz 
$\displaystyle \frac{3+6\cdot 5}{7-4}$
bi napisali: \verb+(3+6*5)/(7-4)+. U ovom
slu\v{c}aju brojnik i nazivnik treba staviti unutar zagrada jer bi u
slu\v{c}aju da je izraz \verb+3+6*5/7-4+ kompjuter poku\v{s}ao prvo
izra\v{c}unati \verb+6*5/7+, python naime izraze ra\v{c}una paze\'{c}i na
prednost ra\v{c}unskih operacija (npr. mno\v{z}enje i dijeljenje imaju
prednost pred zbrajanjem i oduzimanjem).

\textbf{Va\v{z}no:}
Ima jo\v{s} jedna stvar na koju treba pripaziti pri pisanju algebarskih
izraza; ukoliko su brojevi s kojima ra\v{c}unamo cjelobrojni onda \'{c}e (u
programskom jeziku python) i rezultat biti cjelobrojan. Dakle, ako probate
izra\v{c}unati \verb+13/4+ dobiti \'{c}ete \verb+3+, a ne \verb+3.25+! To se
mo\v{z}e rije\v{s}iti tako da barem jedan od brojeva definiramo kao realan, a za
to je dovoljno dodati mu decimalnu to\v{c}ku na kraju. Da bi dobili to\v{c}an
rezultat dijeljenje 13 podijeljeno s 4 trebali bi dakle napisati
\verb+13/4+.

Zadaci:

\section{Logi\v{c}ki izrazi}

Sli\v{c}no kao aritmeti\v{c}ki izrazi logi\v{c}ki izrazi se sastoje od
operacija i \v{c}lanova izraza nad kojima ze izvr\v{s}avaju te operacije.
Kod aritmeti\v{c}kih izraza \v{c}lanovi su brojevi ili varijable s
brojevnom vrijedno\v{s}\'{c}u, a \v{c}lanovi logi\v{c}kih izraza mogu biti
\emph{sudovi} ili \v{c}ak drugi aritmeti\v{c}ki izrazi.

\emph{Sud} je tvrdnja koja mo\v{z}e biti istinita ili la\v{z}na. Primjer
suda je \emph{"Zemlja kru\v{z}i oko Mjeseca"} ili \emph{"Postoji
beskona\v{c}no mnogo prirodnih brojeva"}.  Svaki sud mora imati svoju
istinosnu vrijednost koja mo\v{z}e biti \emph{"istina"} ili \emph{"la\v{z}"}.
Ukoliko za neku tvrdnju ne mo\v{z}emo sa sigurno\v{s}\'{c}u kazati je li
istinita ili la\v{z}na tada to nije sud. Na primjer \emph{"Zemlja
kru\v{z}i oko Mjeseca"} jest sud zato \v{s}to ima istinosnu vrijednost
\emph{"la\v{z}"}, kao i \emph{"Postoji beskona\v{c}no mnogo prirodnih
brojeva"} \v{c}ija je istinita vrijednost \emph{"istina"}.  Tvrdnja
\emph{"Frank Sinatra pjeva bolje od Tine Turner"} nije sud jer je
nemogu\'{c}e odrediti istinitost te tvrdnje budu\'{c}i da je ona \v{c}isto
subjektivne prirode (nekome se vi\v{s}e svidja Sinatra, a nekome Tina
Turner). Isto tako nije sud "U\v{c}ini to!" ili "Mo\v{z}da \'{c}u jednog
dana nau\v{c}iti programirati".

Umjesto "istina" ili "la\v{z}" se \v{c}esto koriste velika slova "T" (od
engleskog "true" = "istina") odnosno "F" (eng. "false" = "la\v{z}").

Sud mo\v{z}emo zapisati i matemati\v{c}kim simbolima: $1<2$ je sud u
kojemu se tvrti da je jedan manji od 2, a istinosna vrijednost tog
suda je T (= istina). Matemati\v{c}ki sudovi kojima se koristimo u
programiranju naj\v{c}e\v{s}\'{c}e se koriste za opisivanje odnosa izmedju
brojeva. Pri tome se koristimo sljede\'{c}im simbolima iz sljede\'{c}e tablice;
u prvom stupcu se nalazi simbol kako bismo za zapisali u bilje\v{z}nicu
ili na plo\v{c}u, u drugom stupcu zapisa tog istog simbola u programu,
a u tre\'{c}em kako \v{c}itamo taj simbol:

\begin{tabular}{lll}
	$=$ & $==$ & je jednako \\
	$\neq$ & $!=$ & nije jednako, je razli\v{c}ito \\
	$<$ & $<$ & je manje od \\
	$\leq$ & $<=$ & je manje ili jednako \\
	$>$ & $>$ & je ve\'{c}e od \\
	$\geq$ & $>=$ & je ve\'{c}e ili jednako
\end{tabular}

Probajmo sada utvrditi istinosnu vrijednost nekih matemati\v{c}kih izraza:

\begin{tabular}{lll}
	$12<12.01$ & \verb+12<12.01+ & T\\
	$1+2\leq 5$ & \verb"1+2<=5" & T\\
	$10-3\geq 6+1$ & \verb"10-3>=6+1" & T \\
	$10-2\geq 6+1$ & \verb"10-2>=6+1" & F \\
	$5\neq 5$ & \verb"5!=5" & F
\end{tabular}

Gornji primjeri su primjeri \emph{jednostavnih sudova}. Slo\v{z}eni
sudovi su sudovi koji se dobijaju od jednostavnih sudova i logi\v{c}kih
operacija \emph{and}, \emph{or} i \emph{not}.

\subsection{Logi\v{c}ka operacija \emph{and}}

Promotrimo re\v{c}enicu \emph{"Ako je lijepo vrijeme
idemo na izlet."}. O \v{c}emu ovisi o\'{c}ete li
oti\'{c}i na izlet? Ovisi o tome je li lijepo vrijeme, dakle ovisi o
istinosnoj vrijednosti suda \emph{"Lijepo je vrijeme."}. Ako je
taj sud istinit (T) oti\'{c}i \'{c}ete na izlet, a ako nije (F) -- ni\v{s}ta
od izleta.

Malo \'{c}emo stvar zakomplicirati ako osim lijepog vremena va\v{s} izleta
ovisi o jo\v{s} ne\v{c}emu; \emph{"Ako je lijepo vrijeme i nemam drugih obaveza
i\'{c}i \'{c}u na izlet"}.
Sad va\v{s} izlet ovisi o istinitosti suda \emph{"Lijepo je vrijeme i nemam obaveza"}, a on je istinit
kad su istovremeno istinita sljede\'{c}a dva suda:

\begin{itemize}
	\item[\emph{(a)}] \emph{"Lijepo je vrijeme."}
	\item[\emph{(b)}] \emph{"Nemam obaveza."}
\end{itemize}

Dakle, treba vrijediti da je istinito i \emph{(a)} i \emph{(b)},
jer ako je bilo koje od ta dva la\v{z} onda je i tvrdnja \emph{"Lijepo
je vrijeme i nemam obaveza"} la\v{z}na.

Ukoliko imamo dva suda koja \'{c}emo ovdje ozna\v{c}iti s $A$, odnosno
$B$ onda \'{c}emo takvu kombinaciju zapisivati s $A and B$. Suda $A
and B$ je \emph{slo\v{z}eni sud} koji se sastoji od jednostavnijih
sudova $A$ i $B$. Istinitost suda $A and B$ ovisi o istinitosti
sudova $A$ i $B$; tek ako su oba istinita onda je i $A and B$
istinit. To se mo\v{z}e prikazati pomo\'{c}u sljede\'{c}e tablice:

\begin{tabular}{ll|l}
	$A$ & $B$ & $A and B$ \\
	\hline
	istina & istina & istina \\
	istina & la\v{z} & la\v{z} \\
	la\v{z} & istina & la\v{z} \\
	la\v{z} & la\v{z} & la\v{z} 
\end{tabular}

Garnju tablicu zovemo \emph{tablica istinitosti} logi\v{c}ke operacije $and$.

\subsection{Logi\v{c}ka operacija \emph{or}}

Pretpostavimo da \'{c}ete oti\'{c}i na izlet ako vrijedi \emph{"Lijepo
je vrijeme ili imam dobro dru\v{s}tvo"}. Dakle, ako je lijepo vrijeme
idete na izlet, ako nije lijepo vrijeme i imate dobro dru\v{s}tvo ipak
idete na izlet, ako nemate dobro dru\v{s}tvo i lijepo je vrijeme opet
idete na izlet, a jedini slu\v{c}aj kad ne idete na izlet je kad, niti
je vrijeme lijepo niti imate dobro dru\v{s}tvo.

Tablica istinitosti logi\v{c}ke operacije $or$ izgleda ovako:

\begin{tabular}{ll|l}
	$A$ & $B$ & $A or B$ \\
	\hline
	istina & istina & istina \\
	istina & la\v{z} & istina \\
	la\v{z} & istina & istina \\
	la\v{z} & la\v{z} & la\v{z} 
\end{tabular}

\subsection{Logi\v{c}ka operacija \emph{not}}

U zadnjem slu\v{c}aju oti\'{c}i \'{c}ete na izlet tek ako vrijedi \emph{"Nemam
drugih obaveza"}. Dakle, tek ako \emph{nije} istit sud \emph{"Imam
drugih obaveza"}. Logi\v{c}ka operacija $not$ nekom sudu pridodaje
suprotanu istinosnu vrijednost.

\begin{tabular}{l|l}
	$A$ & $not A$ \\
	\hline
	istina & la\v{z}\\
	la\v{z} & istina
\end{tabular}

Zadaci: Probajte odrediti istinitost sljede\'{c}ih sudova:
	- 1<2 and 13!=5
	- 2>5 or 1=2
	- 1<2 and 5!=5
	- ( 1<2 or 9=5 ) and 3==3
	- ( 2==2 and 3!=5 ) or ( 3==4 )

\subsection{Nekoliko dodatnih pravila}

U programskom jeziku python (sli\v{c}no kao u mnogim drugim) postoji
jo\v{s} nekoliko dodadnih pravila kod utvrdjivanjaistinitosti sudova:
- svaki broj/izraz je po definiciji sud:
	- ako njegova vrijednost 0 onda je njegova istinosna vrijednost "la\v{z}" (F)
	- ako je razli\v{c}it od 0 onda je njegova istinosna vrijednost "istina" (T)
- svaki string je po definiciji sud:
	- ako je string prazan onda je njegova istinosna vrijednost "la\v{z}" (F)
	- ako string nije prazan onda je njegova istinosna vrijednost "istina" (T)

dakl ima smisla sud \verb"2<3 or 0". Budu\'{c}i da je $2<3$ ondaje prvi dio suda istinit, a (broj) 0 je
po gornjim pravilima la\v{z}an, dakle imamo slu\v{c}aj "istina" $or$ "la\v{z}", dakle rezultat je istina.

Zadaci: Probajte odrediti istinitost sljede\'{c}ih sudova:
\begin{itemize}
\item \verb"1 or 3>3"
\item \verb"2!=3 and 1"
\item \verb+"" or "jkljkl"+
\item \verb+( 0 or "jkljkl" ) and 2<3+
\item \verb+( not "jkljkl" ) and ( not 12 )+
\end{itemize}

\chapter{Interaktivni python}

Python možete pokrenuti na dva načina:

\begin{itemize}
	\item \emph{Interaktivni način} -- u njemu se rezultat svake naredbe vidi
		točno u trenutku kad je utipkana.
	\item \emph{Programski način} -- u kojemu je potrebno unaprijed pripremiti
		niz naredbi koje će kasnije Python izvršiti "u komadu".
\end{itemize}

U ovom poglavlju ćemo na brzinu proći osnove Pythona uz interaktivni
\emph{shell}\footnote{\emph{Shell}, engleski "školjka" je općeniti naziv za pristup
nekom računalnom sustavu u kojem sustav komande izvršava u trenutku kad su upisane.
Primjeri drugih \emph{shell}ova su linux shell, MS-DOS, isl.}. Na neki način
interaktivni \emph{shell} je kao kalkulator s kojim možete komunicirati putem komandi
programskog jezika Python. Ovdje ćemo na brzinu
dotaknuti neke od pojmova i principa koji će u kasnijim poglavljima biti detaljnije
objašnjeni.

\section{Interaktivno sučelje Pythona}

Jednostavno u komandnoj liniji utpikajte \verb+python+ i dobiti ćete taj interaktivni
\verb+shell+:

\begin{verbatim}
puzz@puzz:~$ python
Python 2.6.6 (r266:84292, Sep 15 2010, 15:52:39) 
[GCC 4.4.5] on linux2
Type "help", "copyright", "credits" or "license" for more information.
>>> 
\end{verbatim}

Ovdje sad možete upisivati python naredbe i odmah dobiti povratnu informaciju od \verb+python+ovog
interpretera. Na primjer, idemo zatražiti da nam zbroji dva broja:

\begin{verbatim}
>>> zbroji mi dva i pet
  File "<stdin>", line 1
    zbroji mi dva i pet
            ^
SyntaxError: invalid syntax
\end{verbatim}

Rezultat je poruka o grešci, jer \verb+python+ ne razumije normalan ljudski jezik, a posebno ne
hrvatski. Ispravno bi bilo:

\begin{verbatim}
>>> 2 + 5
7
\end{verbatim}

\section{Varijable}

Ukoliko želimo da nam \verb+python+ privremeno zapamti neku vrijednost, možemo koristiti varijable:

\begin{verbatim}
>>> a = 5 + 7
>>> a
12
>>> b = a * 2
>>> b
24
>>> print 'a = ', a
a =  12
>>> print 'b = ', b
b =  24
>>> 
\end{verbatim}

\section{Brojevi i matematički izrazi}

\section{Stringovi}

\section{Klase i objekti}

\section{Pregled atributa}

\section{Dokumentacija}

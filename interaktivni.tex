\chapter{Interaktivni python}

Python možete pokrenuti na dva načina:

\begin{itemize}
	\item \emph{Interaktivni način} -- u njemu se rezultat svake naredbe vidi
		točno u trenutku kad je utipkana.
	\item \emph{Programski način} -- u kojemu je potrebno unaprijed pripremiti
		niz naredbi koje će kasnije Python izvršiti "u komadu".
\end{itemize}

U ovom poglavlju ćemo na brzinu proći osnove Pythona uz interaktivni
\emph{shell}\footnote{\emph{Shell}, engleski "školjka" je općeniti naziv za pristup
nekom računalnom sustavu u kojem sustav komande izvršava u trenutku kad su upisane.
Primjeri drugih \emph{shell}ova su linux shell, MS-DOS, isl.}. Na neki način
interaktivni \emph{shell} je kao kalkulator s kojim možete komunicirati putem komandi
programskog jezika Python. Ovdje ćemo na brzinu
dotaknuti neke od pojmova i principa koji će u kasnijim poglavljima biti detaljnije
objašnjeni.

\section{Interaktivno sučelje Pythona}

Jednostavno u komandnoj liniji 

\section{Varijable}

\section{Brojevi i matematički izrazi}

\section{Stringovi}

\section{Klase i objekti}

\section{Pregled atributa}

\section{Dokumentacija}

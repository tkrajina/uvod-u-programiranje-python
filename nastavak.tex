\chapter{Izrazi}

\section{Aritmeticcki izrazi}

Aritmeticcki izrazi su matematiccki izrazi s kakvima se raccuna u
osnovnoj sskoli. Najccessche se sastoje od brojeva ili varijabli
koje imaju brojccanu vrijednost i matematicckih operacija. Primjer
aritmeticckog izraza mozze biti: $2+4-2$, 
$\frac{3+6\cdot 5}{7-4}$,
ili $\frac{s_2-s_1}{t_2-t_1}$. 

Kad nam negdje u programu zatreba
aritmeticcki izraz zapisujemo ga na sliccan naccin kako bismo ga napisali u
biljezznici s nekoliko sitnih razlika;

\begin{itemize}
	\item Mnozzenje zapisujemo pomochu znaka *, a ne $\cdot$.
	\item Dijeljenje zapisujemo pomochu znaka / umjesto :
	\item Razlomke zapisujemo pomochu operacije dijeljenja.
	\item Potencije zapisujemo pomochu "**". Dakle $3^2$ bi zapisalo kao
		3**2
	\item Ostatak\footnote{"Modulo"} pri dijeljenju dobijemo pomochu
		operacije \%.
\end{itemize}

Izraz 
$\frac{3+6\cdot 5}{7-4}$, bi napisali: \verb+(3+6*5)/(7-4)+. U ovom
sluccaju brojnik i nazivnik treba staviti unutar zagrada jer bi u
sluccaju da je izraz \verb+3+6*5/7-4+ kompjuter pokussao prvo
izraccunati \verb+6*5/7+, python naime izraze raccuna pazechi na
prednost raccunskih operacija (npr. mnozzenje i dijeljenje imaju
prednost pred zbrajanjem i oduzimanjem).

\textbf{Vazzno:}
Ima joss jedna stvar na koju treba pripaziti pri pisanju algebarskih
izraza; ukoliko su brojevi s kojima raccunamo cjelobrojni onda che (u
programskom jeziku python) i rezultat biti cjelobrojan. Dakle, ako probate
izraccunati \verb+13/4+ dobiti chete \verb+3+, a ne \verb+3.25+! To se
mozze rijessiti tako da barem jedan od brojeva definiramo kao realan, a za
to je dovoljno dodati mu decimalnu toccku na kraju. Da bi dobili toccan
rezultat dijeljenje 13 podijeljeno s 4 trebali bi dakle napisati
\verb+13/4+.

Zadaci:

\section{Logiccki izrazi}

Sliccno kao aritmeticcki izrazi logiccki izrazi se sastoje od
operacija i cclanova izraza nad kojima ze izvrssavaju te operacije.
Kod aritmeticckih izraza cclanovi su brojevi ili varijable s
brojevnom vrijednosschu, a cclanovi logicckih izraza mogu biti
\emph{sudovi} ili ccak drugi aritmeticcki izrazi.

\emph{Sud} je tvrdnja koja mozze biti istinita ili lazzna. Primjer
suda je \emph{"Zemlja kruzzi oko Mjeseca"} ili \emph{"Postoji
beskonaccno mnogo prirodnih brojeva"}.  Svaki sud mora imati svoju
istinosnu vrijednost koja mozze biti \emph{"istina"} ili \emph{"lazz"}.
Ukoliko za neku tvrdnju ne mozzemo sa sigurnosschu kazati je li
istinita ili lazzna tada to nije sud. Na primjer \emph{"Zemlja
kruzzi oko Mjeseca"} jest sud zato ssto ima istinosnu vrijednost
\emph{"lazz"}, kao i \emph{"Postoji beskonaccno mnogo prirodnih
brojeva"} ccija je istinita vrijednost \emph{"istina"}.  Tvrdnja
\emph{"Frank Sinatra pjeva bolje od Tine Turner"} nije sud jer je
nemoguche odrediti istinitost te tvrdnje buduchi da je ona ccisto
subjektivne prirode (nekome se visse svidja Sinatra, a nekome Tina
Turner). Isto tako nije sud "Uccini to!" ili "Mozzda chu jednog
dana naucciti programirati".

Umjesto "istina" ili "lazz" se ccesto koriste velika slova "T" (od
engleskog "true" = "istina") odnosno "F" (eng. "false" = "lazz").

Sud mozzemo zapisati i matematicckim simbolima: $1<2$ je sud u
kojemu se tvrti da je jedan manji od 2, a istinosna vrijednost tog
suda je T (= istina). Matematiccki sudovi kojima se koristimo u
programiranju najccessche se koriste za opisivanje odnosa izmedju
brojeva. Pri tome se koristimo sljedechim simbolima iz sljedeche tablice;
u prvom stupcu se nalazi simbol kako bismo za zapisali u biljezznicu
ili na ploccu, u drugom stupcu zapisa tog istog simbola u programu,
a u trechem kako ccitamo taj simbol:

\begin{tabular}{lll}
	$=$ & $==$ & je jednako \\
	$\neq$ & $!=$ & nije jednako, je razliccito \\
	$<$ & $<$ & je manje od \\
	$\leq$ & $<=$ & je manje ili jednako \\
	$>$ & $>$ & je veche od \\
	$\geq$ & $>=$ & je veche ili jednako
\end{tabular}

Probajmo sada utvrditi istinosnu vrijednost nekih matematicckih izraza:

\begin{tabular}{lll}
	$12<12.01$ & \verb+12<12.01+ & T\\
	$1+2\leq 5$ & \verb*1+2<=5* & T\\
	$10-3\geq 6+1$ & \verb"10-3>=6+1" & T \\
	$10-2\geq 6+1$ & \verb"10-2>=6+1" & F \\
	$5\neq 5$ & \verb"5!=5" & F
\end{tabular}

Gornji primjeri su primjeri \emph{jednostavnih sudova}. Slozzeni
sudovi su sudovi koji se dobijaju od jednostavnih sudova i logicckih
operacija \emph{and}, \emph{or} i \emph{not}.

\subsection{Logiccka operacija \emph{and}}

Promotrimo reccenicu \emph{"Ako je lijepo vrijeme
idemo na izlet."}. O ccemu ovisi ochete li
otichi na izlet? Ovisi o tome je li lijepo vrijeme, dakle ovisi o
istinosnoj vrijednosti suda \emph{"Lijepo je vrijeme."}. Ako je
taj sud istinit (T) otichi chete na izlet, a ako nije (F) -- nissta
od izleta.

Malo chemo stvar zakomplicirati ako osim lijepog vremena vass izleta
ovisi o joss neccemu; \emph{"Ako je lijepo vrijeme i nemam drugih obaveza ichi chu na izlet"}.
Sad vass izlet ovisi o istinitosti suda \emph{"Lijepo je vrijeme i nemam obaveza"}, a on je istinit
kad su istovremeno istinita sljedecha dva suda:

\begin{itemize}
	\item[\emph{(a)}] \emph{"Lijepo je vrijeme."}
	\item[\emph{(b)}] \emph{"Nemam obaveza."}
\end{itemize}

Dakle, treba vrijediti da je istinito i \emph{(a)} i \emph{(b)},
jer ako je bilo koje od ta dva lazz onda je i tvrdnja \emph{"Lijepo
je vrijeme i nemam obaveza"} lazzna.

Ukoliko imamo dva suda koja chemo ovdje oznacciti s $A$, odnosno
$B$ onda chemo takvu kombinaciju zapisivati s $A and B$. Suda $A
and B$ je \emph{slozzeni sud} koji se sastoji od jednostavnijih
sudova $A$ i $B$. Istinitost suda $A and B$ ovisi o istinitosti
sudova $A$ i $B$; tek ako su oba istinita onda je i $A and B$
istinit. To se mozze prikazati pomochu sljedeche tablice:

\begin{tabular}{ll|l}
	$A$ & $B$ & $A and B$ \\
	\hline
	istina & istina & istina \\
	istina & lazz & lazz \\
	lazz & istina & lazz \\
	lazz & lazz & lazz 
\end{tabular}

Garnju tablicu zovemo \emph{tablica istinitosti} logiccke operacije $and$.

\subsection{Logiccka operacija \emph{or}}

Pretpostavimo da chete oticci na izlet ako vrijedi \emph{"Lijepo
je vrijeme ili imam dobro drusstvo"}. Dakle, ako je lijepo vrijeme
idete na izlet, ako nije lijepo vrijeme i imate dobro drusstvo ipak
idete na izlet, ako nemate dobro drusstvo i lijepo je vrijeme opet
idete na izlet, a jedini sluccaj kad ne idete na izlet je kad, niti
je vrijeme lijepo niti imate dobro drusstvo.

Tablica istinitosti logiccke operacije $or$ izgleda ovako:

\begin{tabular}{ll|l}
	$A$ & $B$ & $A or B$ \\
	\hline
	istina & istina & istina \\
	istina & lazz & istina \\
	lazz & istina & istina \\
	lazz & lazz & lazz 
\end{tabular}

\subsection{Logiccka operacija \emph{not}}

U zadnjem sluccaju otichi chete na izlet tek ako vrijedi \emph{"Nemam
drugih obaveza"}. Dakle, tek ako \emph{nije} istit sud \emph{"Imam
drugih obaveza"}. Logiccka operacija $not$ nekom sudu pridodaje
suprotanu istinosnu vrijednost.

\begin{tabular}{l|l}
	$A$ & $not A$ \\
	\hline
	istina & lazz\\
	lazz & istina
\end{tabular}

Zadaci: Probajte odrediti istinitost sljedechih sudova:
	- 1<2 and 13!=5
	- 2>5 or 1=2
	- 1<2 and 5!=5
	- ( 1<2 or 9=5 ) and 3==3
	- ( 2==2 and 3!=5 ) or ( 3==4 )

\subsection{Nekoliko dodatnih pravila}

U programskom jeziku python (sliccno kao u mnogim drugim) postoji
joss nekoliko dodadnih pravila kod utvrdjivanjaistinitosti sudova:
- svaki broj/izraz je po definiciji sud:
	- ako njegova vrijednost 0 onda je njegova istinosna vrijednost "lazz" (F)
	- ako je razliccit od 0 onda je njegova istinosna vrijednost "istina" (T)
- svaki string je po definiciji sud:
	- ako je string prazan onda je njegova istinosna vrijednost "lazz" (F)
	- ako string nije prazan onda je njegova istinosna vrijednost "istina" (T)

dakl ima smisla sud \verb"2<3 or 0". Buduchi da je $2<3$ ondaje prvi dio suda istinit, a (broj) 0 je
po gornjim pravilima lazzan, dakle imamo sluccaj $"istina" or "lazz"$, dakle rezultat je istina.

Zadaci: Probajte odrediti istinitost sljedechih sudova:
	- 1 or 3>3
	- 2!=3 and 1
	- "" or "jkljkl"
	- ( 0 or "jkljkl" ) and 2<3
	- ( not "jkljkl" ) and ( not 12 )

\chapter{Grananje programa}

\section{\verb+if+ \dots \ver+then+ \dots \verb+else+ \dots}

Pretpostavimo da imamo program koji od korisnika trazzi da upisse
koliko je bodova zaradio na odredjenom testu i zatim ovisno o broju
bodova ispisuje koju je ocjenu dobio. Ako je dobio manje ili jednako
39 bodova ocjena je 1, ako ima 40-54 ocjena je 2, za 55-69 ocjena
je 3, za 70-85 ocjena je 4, a za visse od 85 ocjenjen je s 5.

Probati chemo za poccetak napisati program koji samo ispisuje je
li ocjena 1 ili vecha. Za to program treba nekako imati naccin kako
che provjeriti je li broj bodova vechi, manji ili jednak 40.

Program izgleda ovako:

---------------------
bodovi = input( "Upissi broj bodova:" )
print "Imate", bodovi, "bodova..."
if bodovi < 40:
	print "Nazzalost dobili ste negativnu ocjenu :("
print "Kraj programa"
---------------------

Korisstena je naredba if, ona se koristi na sljedechi naccin:

------
if \emph{logiccki izraz}:
	\emph{komande programa u sluccaju da je logiccki izraz istinit}
------

Kao prvo uoccite da je dio programa koji seizvrssava u sluccaju da
je logiccki izraz istinit \emph{uvuccen} u odnosu na ostatak
programa. To uvlaccenje je toccno definirano i mora biti jedan
<tab> (ili 8 razmaknica) raccunajuchi od lijevog ruba polja u kojem se
editira program!

U poccetnom programu se komanda nakon \verb"if" naredbe izvrssava
jedino u sluccaju ako je broj bodova manji od 40, ako je broj bodova
vechi ili jednak 40 program tu liniju jednostavno preskacce. Mozzemo
srediti i da program preskacce vechi broj linija:

---------------------
bodovi = input( "Upissi broj bodova:" )
print "Imate", bodovi, "bodova..."
if bodovi < 40:
	print "Nazzalost dobili ste negativnu ocjenu :("
	print "Molimo vas lijepo da za sljedechi put malo bolje nauccite"
print "Kraj programa"
---------------------

Poruka koja se sad ispisuje ukoliko nemate dovoljno bodova je 

--------------
Nazzalost dobili ste negativnu ocjenu :(
Molimo vas lijepo da za sljedechi put malo bolje nauccite
--------------

Opet, \verb"Kraj programa" se ispisuje bez obzira na broj bodova.

Zzelimo li da nass program ispisuje i poruku ukoliko imate visse ili jednako od 40 bodova to se mozze
tako da koristite dva puta naredbu \verb"if":

---------------------
bodovi = input( "Upissi broj bodova:" )
print "Imate", bodovi, "bodova..."
if bodovi < 40:
	print "Nazzalost dobili ste negativnu ocjenu :("
if bodovi >= 40:
	print "Imate visse od 40 bodova"
print "Kraj programa"
---------------------

Ili pomochu jednog dodatka naredbi \verb"if":

---------------------
bodovi = input( "Upissi broj bodova:" )
print "Imate", bodovi, "bodova..."
if bodovi < 40:
	print "Nazzalost dobili ste negativnu ocjenu :("
else:
	print "Imate visse od 40 bodova"
print "Kraj programa"
---------------------

\emph{Ako je broj bodova manji od 40 tada ste dobili negativnu
ocjenu, inacce imate visse od 40 bodova}

Sve ono ssto se nalazi nakon \verb"else", naravno uvucceno za jedan
<tab> che biti ispisano u sluccaju da uvjet \verb"bodovi<40" nije
istinit!

Ssta ako ne zzelimo samo podatak o tome jesmo li dobili visse ili
manje od 40 bodova nego i koju smo ocjenu dobili:

Prvi naccin koristechi niz \verb"if"-ova:

------------------
bodovi = input( "Upissi broj bodova:" )
print "Imate", bodovi, "bodova..."
if bodovi < 40:
	print "Nazzalost dobili ste negativnu ocjenu :("
if 40<=bodovi and bodovi<=54:
	print "Dovoljan (2)"
if 55<=bodovi and bodovi<=69:
	print "Dobar (3)"
if 70<=bodovi and bodovi<=84:
	print "Vrlo dobar (4)"
if 85<=bodovi:
	print "Odliccan (5)"
------------------

Drugi naccin koristechi \verb"elif" (od eng. "else if"):

------------------
bodovi = input( "Upissi broj bodova:" )
print "Imate", bodovi, "bodova..."
if bodovi < 40:
	print "Nazzalost dobili ste negativnu ocjenu :("
elif 40<=bodovi and bodovi<=54:
	print "Dovoljan (2)"
elif 55<=bodovi and bodovi<=69:
	print "Dobar (3)"
elif 70<=bodovi and bodovi<=84:
	print "Vrlo dobar (4)"
elif 85<=bodovi:
	print "Odliccan (5)"
------------------

Na hrvatskom bi ovo napisali: \emph{Ako imate manje od 40 bodova dobili ste 2, inacce ukoliko imate
izmedju 40 i 54 dobili ste 2, inacce ukoliko imate izmedju 55 i 59 dobili ste 3, inacce ukoliko imate
izmedju 70 i 84 dobili ste 4, inacce ukoliko imate visse od 84 dobili ste 5}.

Mali problem mozze nastati u tome ssto mozzete napisati i 120 ili -340 za broj bodova. Probajmo
napisati program koji ispravlja tu gressku:

------------------
bodovi = input( "Upissi broj bodova:" )
print "Imate", bodovi, "bodova..."
if 0<= bodovi and bodovi < 40:
	print "Nazzalost dobili ste negativnu ocjenu :("
elif 40<=bodovi and bodovi<=54:
	print "Dovoljan (2)"
elif 55<=bodovi and bodovi<=69:
	print "Dobar (3)"
elif 70<=bodovi and bodovi<=84:
	print "Vrlo dobar (4)"
elif 85<=bodovi and bodovi <=100:
	print "Odliccan (5)"
else:
	print "Niste upisali broj izmedju 0 i 100 za broj bodova"
------------------

Sad bi "prijevod" ovog programa na hrvatski glasio: \emph{Ako imate
izmedju 0 i 40 bodova dobili ste 2, inacce ukoliko imate izmedju
40 i 54 dobili ste 2, inacce ukoliko imate izmedju 55 i 59 dobili
ste 3, inacce ukoliko imate izmedju 70 i 84 dobili ste 4, inacce
ukoliko imate izmedju 84 i 100 dobili ste 5, a ako nije niti jedan
od ovih sluccajeva onda ste pogressno upsali broj bodova}.

Zadaci: \textbf{Za napraviti}

\section{\verb"for" \dots \verb"in range(" \dots \verb")"}

Problem je sljedechi; treba ispisati tablicu kvadrata brojeva od do 10.
Tablica kvadrata je tablica koja ima dva stupca, u prvom se nalazeprirodni
brojevi, a u drugom njihovi kvadrati (da podsjetimo, kvadrat prirodnog
broja dobijemo tako da taj broj pomnozzimo sa samim sobom).

To se mozze postichi na sljedechi naccin:

-------------------
print "n -> n*n"
print 1, " -> ", 1**2
print 2, " -> ", 2**2
print 3, " -> ", 3**2
print 4, " -> ", 4**2
print 5, " -> ", 5**2
print 6, " -> ", 6**2
print 7, " -> ", 7**2
print 8, " -> ", 8**2
print 9, " -> ", 9**2
print 10, " -> ", 10**2
-------------------

\verb"5**2" znacci $5^2$. U svakom retku se naredbom print ispisuje broj i
njegov kvadrat. 

Medjutim, ovo je jedan priliccno \emph{neelegantan} naccin rjessavanja
problema, a to zato ssto je ovakav program tessko generalizirati,
odnosno tessko ga je upotrijebiti u sluccaju da njime moramo
rijessiti neki analogni ili opcehnitiji problem. 
\footnote{
	Ako je nass trenutni
	problem \emph{"Napissi program koji ispisuje tablicku kvadrata
	prvih 10 brojeva"} analogni (ali slozzeniji) problem mozze glasiti
	\emph{"Napissi program koji ispisuje tablicu kvadrata za brojeve
	od 0 do 100"}. Opchenitiji problem bi mogao biti \emph{"Napissi
	program koji ispisuje tablicu kvadrata za brojeve od 0 do n (gdje
	je n proizvoljan prirodan broj"}. 
}
Taj program se mozze napisati
na naccin sluccan nassem naccinu, ali priznati chete da pisanje
sto i jedne linije tipa s naredbom \verb"print" koja ne radi nissta
drugo negoli ispisuje broj i njegov kvadrat i nije neki pretjerano
kreativan posao.

Zato postoji nareda \verb"for". Ta naredba od raccunala trazzi da
odredjeni broj puta ponovi neki postupak uz odredjene uvjete. Uz
naredbu \verb"for" se nalazi ime varijable i nekakav \emph{skupa}
ili \emph{lista} prema kojima se ta varijabla "kreche". Buduchi da
che se ova skripta baviti skupovima, listama i ostalim slozzenim
tipovima podataka baviti tek kasnije ovdje chu objasniti samo jedan
od naccina koji se ccesto koriste s naredbom \verb"for".

--------------------
print "n -> n*n"
for x in range(11):
	print x, " -> ", x**2
--------------------

Rezultat programa che biti potpuno isti kao i kod prosslog programa.

Druga i trecha linija ovog programa kazze otprilike: \emph{"Neka
varijabla x uzima redom vrijednost 0, 1, 2\dots sve dok je manje
od 11, i za svaku od tih vrijednosti ispissi vrijednost od x i
kvadrat od x"}. Nakon ssto python interpreter dodje do druge linije
programa raccunalo samo varijabli x pridodaje vrijednost 0, i
izvrssava trechu liniju programa (x je tamo 0). Nakon ssto to
izvrssi varijabli x se pridruzzuje sljedecha vrijednost; 1. Sad se
opet ispisuje trecha linija programa, ali s novom vrijednoschu
varijeble x. Nakon toga x poprima vrijednost 2, izvrssava se trecha
linija, i tako dalje\dots

\textbf{Vazzno:}
Nikad ne zaboravite dvotoccku iza "for" linije -- to je naime vrlo
ccesta poccetniccka gresska gresska

\textbf{Definicija:} Za liniju s \verb"for" naredbom i niz naredbi koje
se izvrssavaju pri svakoj promjeni varijable nakon \verb"for"
kazzemo da se zovu \textbf{for-petlja}. Ukoliko je \verb"x" varijabla
nakon \verb"for" kazzemo da smo izvrssili \textbf{for-petlju po varijabli
x}. Svaki put kad varijabla \verb"x" promijeni vrijednost zbog \verb"for"
petlje kazzemo da je izvrsseno \textbf{iteracija}. Niz naredbi koje se
izvrssavaju u svakoj iteraciji zovemi tijelo for-petlje.

Isto tako, mozzete srediti da se izvrssava i visse linija svaki
put kad varijabla x u zbog \verb"for" mijenja vrijednost. Jednostavno
nakon trecheg reda programa napisali biste joss jedan, ali i taj
obavezno mora biti pomaknuti za jedan <tab> udesno.

Pretpostavimo sad da treba napisati prgram koji od koristnika trazzi
da mu upisse jedan broj, a zatim ispisuje prvo kvadrat tog broja,
a onda rezultat pri cjelobrojnom dijeljenju 
\footnote{
	Cjelobrojno dijeljenje znacci da se radi o dijeljenju u
	kojemu se ignorira dio iza decimalne toccke u rezultatu
	nekog dijeljenja. Npr ukoliko podijelimo 7 sa 4 dobiti
	chemo 1.75, ali rezultat cjelobrojnog dijeljenja 7 sa 4 je
	1. Podsjetimo da python vrssi cjelobrojno dijeljenje ukoliko
	su djeljenik i djelitelj cijeli brojevi (nemaju decimalnog
	dijela).
}
tog broja s 2.

------------
n = input( "Upissi broj:" )
for i in range( n+1 ):
	print "kvadrat od", i, "je", i**2
	print "rezultat cjelobrojnog dijeljenja od", i, "s 2 je", i/2
print "Kraj programa"
------------

Ovdje se tijelo for-petlje sastoji od dvije neradbe (trechi i
ccetvrti redprograma. Zadnja linija ne spada u tijelo petlje jer
se ne izvrssava kod svake iteracije nego samo jednom nakon ssto se
cijela petlja "izvrssi".

Prvo se varijabli \verb"n" pridjeljuje vrijednost koju upisuje
osoba koja pokreche program. Zatim se vrssi petlja po varijabli
\verb"i". Obratite pazznju da se petlja sad greche u granicama od
0, pa po svim cijelim brojevima manjim od \verb"n+1", a najvechi
cijeli broj manji od \verb"n+1" je \verb"n". Dakle petlja se ne
izvrassava za \verb"n+1" nego samo do \verb"n", a to je upravo ono
ssto mi zzelimo.

Zzelim sad program koji che za svaki broj od 0 do 100 ispisati
njegov kvadrat i, ukoliko je taj broj paran ispisati i poruku o
tome. Broj je paran u koliko je rezultati pri dijeljenju tog broja
s 2 jednak s 0. Rezultat pri dijeljenju dobijemo pomochu operacije
modulo (\verb"%" u python programu)

-------------------
1: for x in range(101):
2: 	print x, "na kvadrat iznosi", x**2
3: 	if x % 2 == 0:
4: 		print x, "je paran broj"
5: print "Kraj programa"
-------------------

\textbf{Komentar:} Tijelo for-petlje koja poccinje s prvim redom
programa je niz naredbi koji se sastoji od 2-4 linije programa.
Obratite pazznju da je ovdje 4. linija, koja se izvrssava jedino
kad je istinit sud \verb"x%2==0" pomaknuta za \emph{dva} <tab>-a!
Za svaki \verb"x" if-uvjet isprobava je li istina da je rezultat
pri djeljenju x s 2 jednak 0 (\verb"x%2==0"), a ako je to istina
izvrssava se 4. linija.

Evo joss jedna varijanta programa:

-------------------
1: for x in range(101):
2: 	print x, "na kvadrat iznosi", x**2
3: 	if x % 2 == 0:
4: 		print x, "je paran broj"
5:		else:
6: 		print x, "nije paran broj"
7: 	print
8: print "Kraj programa"
-------------------

U ccemu se ona razlikuje od gornjeg programa? Probajte sami otkriti ssta
radi osma linija programa? Ovisi li njeno izvrssavanje o tome je li
izvrssen uvjet u 3. redu? Probajte promijeniti program tako da se petlja ne
izvrssava za brojeve do 100 nego da korisnik mozze sam odrediti do kojeg do
kojeg broja se petlja izvrssava.

Probajte objasniti zassto donji program radi potpuno istu stvar
kao i nass zadnji program:\footnote{Pomoch: svaki broj je po definiciji sud
ccija istinitost ovisi o\dots}

-------------------
1: for x in range(101):
2: 	print x, "na kvadrat iznosi", x**2
3: 	if not x % 2:
4: 		print x, "je paran broj"
5:		else:
6: 		print x, "nije paran broj"
7: 	print
8: print "Kraj programa"
-------------------

Joss jedna korisna varijanta varijanta for-petlja s \verb"range(...)" je
sluccaj u kojem ne zzelim da se petlja izvrssava za brojeve od 0 do nekog
broja nego od nekog broja razliccitog od 0 do nekog drugog broja:

-----------------
for i in range( 15, 70 ):
	print i
-----------------

Ispisati che brojeve od 15 do 70.

Probajte objasniti ssta radi sljedechi program:

----------
a = input( "Upissi prvi broj:" )
b = input( "Upissi prvi broj:" )
if a < b:
	for x in range( a, b + 1 ):
		print x, "na trechu iznosi", x**3
else:
	print "hmmm..."
----------

Napissite sada program koji od korisnika trazzi da upisse dva broja, a
zatim ispisuje tablicu kvadrata svh brojeva od prvog do drugog.

Evo joss jedan zanimljiv program koji ispisuje tablicu mnozzenja brojeva od
1 do 10. Treba dakle napisati sve izraze oblika $a\cdot b$ gdje $a$ i $b$
mogu biti brojevi od 1 do 10. Ali (!) za svaki $a$ od 1 do 10 (for-petlja)
i $b$ mora mochi poprimiti vrijednosti od 1 do 10.

----------
for a in range(1, 11):
	for b in range(1, 11):
		print a, "puta", b, "je jednako", a*b
----------

Tijelo prve petlje je drugi i trechi red, a tijelo druge petlje je samo
trechi red. Dakle, za svaki a od 1 do 10 izvrssiti che se druga petlja u
kojoj se sad b mijenja od 1 do 10 i kod svake promjene ispisuje poruka.

Ukoliko mislite da ste shvatili rad ovih programa probajte sami sebi
postavljati probleme i nachi naccina kako te probleme rijessavati
programima u pythonu. Isto tako, korisno mozze biti da probati modificirati
vech napisane programe i onda jednostavno pogledate ssta se dessava kad ih
pokussate pokrenuti.

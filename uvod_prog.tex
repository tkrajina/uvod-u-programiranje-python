\chapter{Uvod u programiranje}

\section{\v{S}to je danas lijep i sun\v{c}an dan}

Kad savladate nekoliko programskih jezika uo\v{c}iti \'{c}ete
nekoliko pravilnosti koje se pojavljuju kod svih njih. Svi oni
koriste varijable, potprograme, dobar dio njih koristi objekte. Ne
samo to, nego u svim priru\v{c}nicima za programiranje prvi program
ima naziv "Hello world"\footnote{eng.  Zdravo svijete}.

Budu\'{c}i da autor ovog spisa nije nimalo revolucionaran tip, i
mi \'{c}emo zapo\v{c}eti Zdravo-svijetom\footnote{\hspace*{2mm}"Hello
world"}.  Daklem. "Hello world" u Pythonu izgleda ovako:

	\sourcee{
		\wrd{print} "Hello world"
	}

\textbf{Komentar:} Druga linija je ono \v{s}to je bitno u ovom
programu. Ona ka\v{z}e prevoditelju-interpreteru da ispi\v{s}e na
monitor niz znakova "Hello world" (bez navodnika, navodnici samo
slu\v{z}e da se zna gdje po\v{c}inje i gdje zavr\v{s}ava ono \v{s}to
\v{z}elimo ispisati).

\subsection{Snimanje programa}

Jedan od gornjih "zdravo-svijete" programa napi\v{s}ite u nekom
tekst editoru (ako radie na Windowsima onda to mo\v{z}e biti NotePad,
ako ste na Linuxu vi, joe, emacs, ako na Macintoshu\dots)

\dots

\subsection{Pokretanje programa}

Na windowsima; nakon \v{s}to ste program snimili kao datoteku s ekstenzijom
.py, na\dj{}ite gdje je ikona s tom datotekom i dvaput kliknite na nju.

Na linuxu; u komandnoj liniji treba napisati: "python $<$ime\_datoteke$>$".
Naravno, $<$ime\_datoteke$>$ zamijenite s imenom kojeg ste upisali pri
snimanju.

\dots

\subsection{Python kao kalkulator}

Pokrenete li\footnote{Na windowsima tako da kliknete na ikonu, a
na linuxu tako da u prompt-u upi\v{s}ete "python"} program kojeg
dobijete kad instalirate python dobiti cete nesto kao:

\sourcee{
Python 2.2 (\#1, Jan  9 2002, 02:47:26) \\
$[$GCC 2.95.2 19991024 (release)$]$ on darwin \\
Type "help", "copyright", "credits" or "license" for more information.  \\
$>>>$ 
}

Ovdje sada  mo\v{z}ete upisivati naredbe programskog jezika python i
vidjeti rezultat:

\sourcee{
Python 2.2 (\#1, Jan  9 2002, 02:47:26) \\
$[$GCC 2.95.2 19991024 (release)$]$ on darwin \\
Type "help", "copyright", "credits" or "license" for more information.  \\
$>>>$ \var{print 2+3} \\
5 \\
$>>>$ \var{d=2} \\
$>>>$ \var{print d+4} \\
6 \\
$>>>$ 
}

Crvenom bojom su ozna\v{c}ene naredbe koje sam u gornjem slu\v{c}aju
pisao ja, a ostatak je generiralo ra\v{c}unalo (odnosno python
interpreter).

Ovaj na\v{c}in pokretanja python interpretera se zove interaktivni. U
principu sve programe u ovom poglavlju mo\v{z}ete pokretati u interaktivnom
na\v{c}inu jednostavno tako da upisujete liniju po liniju programa.

\section{Komentari u programu}

	Gornji program je sasvim mogao izgledati i ovako:

	\sourcee{
		\com{\# "Hello world" (c) 2001 by "Trlababalan" d.d.}\\
		\wrd{print} "Hello world"
	}

	\textbf{Komentar:}
	Prva linija, ona koja zapo\v{c}inje sa znakom \# je \emph{komentar}.
	Kad interpreter do\dj{}e do ovog znaka on jednostavno presko\v{c}i sve
	\v{s}to na\dj{}e nakon tog znaka pa sve do kraja reda.

	Nije lo\v{s}e stavljati komentare u va\v{s}e programe. Oni koji \'{c}e slijediti
	\'{c}e u sebi imati puno komentara. 

	Mogli se napisati i ovako:

	\sourcee{
		\com{\# Do\v{s}ao \v{s}krtac u restoran pojesti juhu, a u jednom trenutku je\\
		\# morao oti\'{c}i na WC. Naravno, uhvatila ga panika da mu netko\\
		\# ne bi pojeo juhu za to vrijeme, pa je ispod juhe stavio papiri\'{c}\\
		\# na kojeg je napisao "Pljunuo sam u juhu" -- nadaju\'{c}i se da \\
		\# tako nikome ne\'{c}e pasti na pamet niti primirisati njegovoj tanjuru\\
		\# \\
		\# Kad se vratio, otkrio je da je na dnu njegovog papiri\'{c}a netko\\
		\# nadopisao "\dots i ja"}\\
		\\
		\wrd{print} "Hello world"
	}

	Mo\v{z}e i ovo:

	\sourcee{
		\wrd{print} "Hello world" 
		\com{\# ovaj program ispisuje na monitor: "Hello world"}
	}

	U stvari gornja \v{c}etiri slu\v{c}aja predstavljaju jedan te isti program.

	Postoji jo\v{s} jedan na\v{c}in kako se mogu pisati komentari u programu:

	\sourcee{
		\wrd{print} "Hello world"\\
		\com{""" Ovo je samo \\
		komentar """}
	}

	\vspace{3mm}
	Komentar po\v{c}inje s nizom od tri navodnika i zavr\v{s}ava s tri navodnika, a
	mo\v{z}e se protezati preko vi\v{s}e linija.

\section{Varijable}	

	Promotrite sljede\'{c}i primjer:



\sourcee{
\var{x} = 5 \\
\textcolor{green}{\# od sada (do daljnjega) varijabla x ima vrijednost 5}\\
\textcolor{blue}{print} "x je", \var{x} \\
\textcolor{green}{\# ispisuje prvo poruku "x je", a nakon nje vrijednost varijable x}\\
\var{y} = 10\\
\textcolor{blue}{print} "...a y je", \var{y}
}

	Nakon \v{s}to ga snimite i pokrenete ispis \'{c}e biti:

\sourcee{
x je sada 5\\
y ima vrijednost 10
}

	\textbf{Obja\v{s}njenje:} u prevom redu programa, gdje pi\v{s}e \verb|x = 5| smo
	definirali varijablu \verb|x|. \verb|x| od sada ima vrijednost 5. To ne zna\v{c}i
	da se ta vrijednost ne\'{c}e promijeniti u budu�nosti. 
	Od sad, svaki put kad poku\v samo ispisati vrijednost x-a dobiti \' cemo broj 5.

	\textbf{Napomena:} znak "=" u prvom redu programa ne treba shvatiti kao
	matemati\v{c}ku tvrdnju da je \verb+x+ isto \v{s}to i 5, odnosno da \verb+x+ ima
	vrijednost 5. Redak "\verb+x = 5+" treba shvatiti kao "Od sada \'{c}e \verb+x+
	imati rijednost 5" ili "Neka \verb+x+ ima vrijednost 5". Dakle varijabli
	\verb+x+ \underline{pridru\v{z}ujemo vrijednost} 5.

	U trenutku kad smo u programu napisali \verb|x = 5| ka\v{z}emo da smo varijablu
	\emph{inicijalizirali}, dakle odredili smo joj neku po\v cetnu vrijednost. Ako
	poku\v samo ispisati neku neinicijaliziranu varijablu interpreter \' ce nam ispisati
	poruku o gre\v sci:

	\sourcee{
	\textcolor{blue}{print} \var{x} \textcolor{green}{\# ...ali varijabla x nije prethodno inicijalizirana!}
	}

	Rezultat je:

	\sourcee{
Traceback (most recent call last):
\\
File "primjer.py", line 1, in ?
\\
print x
\\
NameError: name 'x' is not defined
	}

	Obratite pa\v znju na to da vam se to\v cno ka\v ze gdje je gre\v ska u programu
	("line 1"). To nam sad i nije od neke koristi jer na\v s program ima samo jednu
	liniju, ali kad se stvari zakompliciraju onda \' cemo takve informacije jako puno
	koristiti!
	
	Pogledajmo sljede\'{c}i primjer:

	\sourcee{
	\var{x} = 5 \textcolor{green}{\# x ima sada vrijednost 5}\\
	\textcolor{blue}{print} "x je", \var{x} \textcolor{green}{\# ispisujemo vrijednost od x}\\
	\var{y} = \var{x} \textcolor{green}{\# sada y poprima vrijednost varijable x}\\
	\textcolor{blue}{print} "y je", \var{y} \textcolor{green}{\# ispisujemo vrijednost od y}\\
	\var{x} = 10 \textcolor{green}{\# x je sad 10}\\
	\textcolor{blue}{print} "x je sad ", \var{x} \textcolor{green}{\# ispisujemo x}\\
	\textcolor{blue}{print} "y je sad", \var{y} \textcolor{green}{\# ispisujemo y}
	}

	Kad pokrenemo program ispis je:

	\sourcee{
	x je 5\\
	y je 5\\
	x je sad 10\\
	y je sad 5
	}

	Prvo je \verb|x| bio 5, tada smo varijabli \verb|y| dodijelili vrijednost varijable
	\verb|x|, dakle 5. Tada smo \verb|x| promijenili da sadr\v zi 10, ali \verb|y| je
	ostao 5 otprije.

\section{Naredbe}

	Program se sastoji od niza naredbi. Svaka naredba ra\v{c}unalu govori \v{s}ta
	treba \v{c}initi. Razmotriti \'{c}emo jo\v{s} jednom jedan primjer programa
	pisanog u Pythonu:
	
	\sourcee{
		\var{x} = 5;\\
		\wrd{if} \var{x} $<$ 10:\\
		\hspace*{10mm}\textcolor{blue}{print} "x je manje od 10"\\
		\wrd{else}:\\
		\hspace*{10mm}\textcolor{blue}{print} "x je ve\'{c}e od 10"
	}

	Python je programski jezik koji zahtijeva da se svaka naredba pi\v{s}e u odvojenom
	retku. Tako, svaki red predstavlja jednu naredbu, u prvom redu naredba "=" ka\v{z}e
	ra\v{c}unalu da varijabli \verb|x| pridru\v{z}i vrijednost 5. Nakon toga jedan
	\verb+if+--\emph{uvijet}\footnote{Vi\v{s}e rije\v{c}i o uvjetima kasnije\dots} 
	(naredba \verb+if+)
	provjerava je
	li x manji od 10, ako jest onda ispisuje s \verb+print+ poruku "x je manji od 10", a ako nije onda
	poruku "x je ve\'{c}e od 10". Program \'{c}e, naravno, ispisati prvu poruku

	Naredbu \verb+print+ smo upoznali kod pisanja "Hello world" programa -- ona ispisuje na monitor ono \v{s}to se nalazi iza nje.
	O toj naredbi \'{c}emo jo\v{s} puno pri\v{c}ati kasnije.

	U nekim programskim jezicima mogu se pisati
	programi koji u jednoj liniji imaju vi\v{s}e naredbi. Pogledajte na primjer kra\'{c}i
	od dva programa pisana u programskom jeziku Perl u odjeljku 1.3.1. -- ono za
	\v{s}to nam u Pythonu treba 5 redova mo\v{z}emo u Perlu napisati dva reda (mogo bi
	se i u jednom).
	Mo\v{z}da vam se to mo\v{z}e \v{c}initi prakti\v{c}nim, ali takvi programski jezici
	rezultiraju s programima koji su vrlo te\v{s}ko shvatljivi. U takvom programu je
	vrlo te\v{s}ko na\'{c}i gre\v{s}ku.

	Pogledajmo jo\v{s} jednom taj program u programskom jeziku Perl:

	\source{Perl:}{
		x = 5;\\
		print x $==$ 10 ? "x je manji od 10$\setminus$n" : "x je ve\'{c}i od 10$\setminus$n";
		}

	Ako znate imalo engleskog onda bi vam prvi primjer pisan u Pythonu trebao biti
	dosta jasan, \v{c}ak i ako ne znate programirati. Pogledajte sad ovaj drugi
	primjer, on je toliko zgusnut i kripti\v{c}an da je programeru-po\v{c}etniku vrlo
	te\v{s}ko shvatiti \v{s}ta je ovdje \v{s}ta. Zamislite da sad imate program koji
	ima 1000 ovakvih linija, a znate da se negdje me\dj{}u njima nalazi jedna
	gre\v{s}ka?!

	\textbf{Zapamtite:} Uvijek se potrudite pisati pregledne programe. Svaku komandu
	pi\v{s}ite u novu liniju. Svaki slo\v{z}eniji dio programa popratite s komentarima!
	Desiti \'{c}e vam se da morate prepraviti program koji ste nekad davno pisali. To
	\'{c}e biti tim lak\v{s}e \v{s}to ste tada taj program preglednije pisali!
	
\section{Imena}

	U zadnja dva primjera u na\v{s}im programima koristili smo se varijablama. Te
	varijable smo nazvali \verb+x+ i \verb+y+. Naravno, mogli smo ih nazvati i
	druk\v{c}ije. Probati \'{c}emo ne\v{s}to nau\v{c}iti iz sljede\'{c}eg primjera:

	\sourcee{
\var{pi} = 3.1415926 \\
\var{v} = 5 \\
\var{V} = 10 \\
\var{nova}\_varijabla = 12345 \\
\var{var1} = 12 \\
\var{var2} = 13 \\
\textcolor{blue}{print} "pi je",\var{pi} \\
\textcolor{blue}{print} "varijabla v ima vrijesnost",\var{v} \\
\textcolor{blue}{print} "varijabla V (veliko slovo!) ima vrijesnost",\var{V} \\
\textcolor{blue}{print} "jos jedna varijabla:", \var{nova\_varijabla} \\
\textcolor{blue}{print} "var1 je", \var{var1}, ", a var2 je", \var{var2} 
	}

	Prvih 6 linija programa varijablama pridru\v{z}uje neke vrijednosti. Imena tih
	varijabli su 
	\verb+pi+,
	\verb+v+,
	\verb+V+,
	\verb+nova_varijabla+,
	\verb+var1+ i 
	\verb+var2+.

	\textbf{Va\v{z}no:} Uo\v{c}ite da Python razlikuje velika i mala
	slova: varijabli \verb+v+ smo
	pridru\v{z}ili vrijednost 5, a varijabli \verb+V+ (veliko slovo V) vrijednost 10.
	Kasnije u programu se njih tretira kao dvije razli\v{c}ite varijable!

	Kad daj program upi\v{s}ete i pokrenete dobiti \'{c}ete:

	\sourcee{
	pi je 3.1415926\\
	varijabla v ima vrijesnost 5\\
	varijabla V (veliko slovo!) ima vrijesnost 5\\
	jos jedna varijabla: 12345\\
	var1 je 12 , a var2 je 13
	}

	\textbf{Zapamtite:} Imena varijabli se mogu sastojati kombinacije velikih i malih
	slova, decimalnih znamenaka
	i znaka "\_"\footnote{Engleski: underscore}. Decimalna znamenka se ne smije
	nalaziti na prvom mjestu imena varijable. Razlikuju se velika i mala slova; dakle
	\verb+kikiriki = 5+ i \verb+Kikiriki = 10+ su dvije razli\v{c}ite varijable s
	razli\v{c}itim vrijednostima! 

	Ima jo\v{s} jedno pravilo kojeg se morate dr\v{z}ati kod odre\dj{}ivanja imena
	varijable; ime varijable \emph{ne smije biti klju\v{c}na rije\v{c}!}

\section{Klju\v{c}ne rije\v{c}i}

	Klju\v{c}ne rije\v{c}i su "rije\v{c}i" ili nizovi znakova koje su rezervirane za
	kori\v{s}tenje u programskom jeziku i ne smiju se koristiti za imenovanje varijabli
	ili nekih drugih dijelova programa koje odre\dj{}uje programer. Ve\'{c} smo se
	sreli s  naredbom \verb+print+. \verb+print+ je klju\v{c}na rije\v{c} i ne smije
	biti kori\v{s}tena kao ime varijable. Mo\v{z}emo na\v{s}u varijablu imenovati npr
	\verb+var_print+ ili \verb+_print+ ili \verb+print1+, ali nikako \verb+print+!

	Svaki programski jezik ima odre\dj{}eni broj klju\v{c}nih rije\v{c}i. Klju\v{c}ne
	rije\v{c}i u programskom jeziku Python su:

\begin{center}
\begin{tabular}{llll}
and & del & for & is \\
raise & assert & elif & from\\
lambda & return & break & else\\
global & not & try & class\\
except & if & or & while\\
continue & exec & import & pass\\
def & finally & in & print
\end{tabular}
\end{center}

	Radi prakti\v{c}nosti, u programima koji se nalaze u ovoj knji\v{z}ici klju\v{c}ne
	rije\v{c}i su prikazane u plavoj boji.

\section{Komuniciranje s okolinom}

	Ra\v cunalo nema previ\v se smisla ako ne komunicira s okolinom. Pod
	\emph{komunikacija s okolinom} podradzumijeva se komunikacija s korisnikom ili
	komunikacija s nekim drugim ure\dj{}ajem vezanim uz ra\v cunalo. Ra\v cunalo
	komunicira s korisnikom na razli\v cite na\v cine; npr. klikanjem po ikonama,
	upisivanjem teksta u nekakav formular, diktiranjem u mikrofon spojen 
	na ra\v cunalo, i tako dalje i tako bli\v ze.

	Nau\v citi \' cemo sada jedan jednostavan na\v cin kako mo\v zemo na\v s program u
	pythonu natjerati da i komunicira s nama.

	\sourcee{
		\var{name} = \textcolor{blue}{raw\_input}( "Upi\v site svoje ime:" )\\
		\textcolor{blue}{print} "Dobar dan", \var{name}
	}

	Kad pokrenete program ra\v cunalo \' ce ispisati poruku "Upi\v site svoje ime:" i
	tra\v ziti od vas da upisujete ime. Rezultat mo\v ze biti:

	\sourcee{
		Upi\v site svoje ime:\textcolor{blue}{Aleksandar Makedonski}\\
		Dobar dan  Aleksandar Makedonski
	}

	Plavom bojom je ozna\v cen tekst kojeg upisuje sam korisnik.

	Pomo\' cu naredbe \verb+raw_input+ sad mo\v zemo bilo kojoj varijabli 
	pridru\v ziti vrijednost broja kojeg \'{c}e korisnik utipkati tek u trenutku kad 
	se program pokrene. \verb"raw_input" je sli\v{c}an, ali s njime mo\v{z}emo toj
	varijabli pridru\v{z}iti i vrijednost stringa (a ne isklju\v{c}ivo brojevnu
	vrijednost).

	Porobati \' cemo sada napisati program koji tra\v zi od korisnika da upi\v se 
	tri broja, a nakon toga ispisuje njihovu aritmeti\v cku sredinu (prosijek).

	\sourcee{
		\var{n1} = \textcolor{blue}{input}( "Upi\v si prvi broj:" )\\
		\var{n2} = \textcolor{blue}{input}( "Upi\v si drugi broj:" )\\
		\var{n3} = \textcolor{blue}{input}( "Upi\v si tre\' ci broj:" )\\
		\var{avg} = ( \var{n1} + \var{n2} + \var{n3} ) / 3\\
		\textcolor{blue}{print} "Aritmeti\v cka sredina je ", \var{avg}
	}

	\textbf{Komentar:} Program prvo u varijable \verb+n1+, \verb+n2+ i \verb+n2+
	smje\v sta ono \v sto \' ce korisnik sam upisati kad ga se upita da upi\v se broj.
	Varijabla \verb"avg" (eng. "average" = "prosjek") zatim prime vrijednost
	aritmeti\v cke sredine brojeva 
	\verb"n1",
	\verb"n2" i 
	\verb"n3".
	Na kraju se samo ispisuje vrijednost od \verb"avg".

	Evo jo\v s dvije varijante istog programa:

	\sourcee{
		\var{n1} = \textcolor{blue}{input}( "Upi\v si prvi broj:" )\\
		\var{n2} = \textcolor{blue}{input}( "Upi\v si drugi broj:" )\\
		\var{n3} = \textcolor{blue}{input}( "Upi\v si tre\' ci broj:" )\\
		\var{avg} = ( float( \var{n1} ) + float( \var{n2} ) + float( \var{n3} ) ) / 3\\
		\textcolor{blue}{print} "Aritmeti\v cka sredina je ", \var{avg}
	}

	Ili jo\v s kra\' ce:

	\sourcee{
		\var{n1} = float( \textcolor{blue}{input}( "Upi\v si prvi broj:" ) )\\
		\var{n2} = float( \textcolor{blue}{input}( "Upi\v si drugi broj:" ) )\\
		\var{n3} = float( \textcolor{blue}{input}( "Upi\v si tre\' ci broj:" ) )\\
		\textcolor{blue}{print} "Aritmeti\v cka sredina je ", ( \var{n1} + \var{n2} + \var{n3} ) / 3
	}

	\textbf{Napomena:} \verb+float()+ nije jedini na\v cin kako se string mo\v ze
	pretvoriti u broj. U sljede\' cem poglavlju \' ce biti obja\v snjeno za\v sto se
	ovdje koristi ba\v s taj.


\chapter{Uvod u programiranje}

\section{Sto je danas lijep i sunčan dan}

Kad savladate nekoliko programskih jezika uočiti ćete
nekoliko pravilnosti koje se pojavljuju kod svih njih. Svi oni
koriste varijable, potprograme, dobar dio njih koristi objekte. Ne
samo to, nego u svim priručnicima za programiranje prvi program
ima naziv "Hello world"\footnote{eng.  Zdravo svijete}.

Budući da autor ovog spisa nije nimalo revolucionaran tip, i
mi ćemo započeti Zdravo-svijetom\footnote{\hspace*{2mm}"Hello
world"}.  Daklem. "Hello world" u Pythonu izgleda ovako:

	\sourcee{
		\wrd{print} "Hello world"
	}

\textbf{Komentar:} Druga linija je ono što je bitno u ovom
programu. Ona kaze prevoditelju-interpreteru da ispiše na
monitor niz znakova "Hello world" (bez navodnika, navodnici samo
sluze da se zna gdje počinje i gdje završava ono što
zelimo ispisati).

\subsection{Snimanje programa}

Jedan od gornjih "zdravo-svijete" programa napišite u nekom
tekst editoru (ako radie na Windowsima onda to moze biti NotePad,
ako ste na Linuxu vi, joe, emacs, ako na Macintoshu\dots)

\dots

\subsection{Pokretanje programa}

Na windowsima; nakon što ste program snimili kao datoteku s ekstenzijom
.py, nađite gdje je ikona s tom datotekom i dvaput kliknite na nju.

Na linuxu; u komandnoj liniji treba napisati: "python $<$ime\_datoteke$>$".
Naravno, $<$ime\_datoteke$>$ zamijenite s imenom kojeg ste upisali pri
snimanju.

\dots

\subsection{Python kao kalkulator}

Pokrenete li\footnote{Na windowsima tako da kliknete na ikonu, a
na linuxu tako da u prompt-u upišete "python"} program kojeg
dobijete kad instalirate python dobiti cete nesto kao:

\sourcee{
Python 2.2 (\#1, Jan  9 2002, 02:47:26) \\
$[$GCC 2.95.2 19991024 (release)$]$ on darwin \\
Type "help", "copyright", "credits" or "license" for more information.  \\
$>>>$ 
}

Ovdje sada  mozete upisivati naredbe programskog jezika python i
vidjeti rezultat:

\sourcee{
Python 2.2 (\#1, Jan  9 2002, 02:47:26) \\
$[$GCC 2.95.2 19991024 (release)$]$ on darwin \\
Type "help", "copyright", "credits" or "license" for more information.  \\
$>>>$ \var{print 2+3} \\
5 \\
$>>>$ \var{d=2} \\
$>>>$ \var{print d+4} \\
6 \\
$>>>$ 
}

Crvenom bojom su označene naredbe koje sam u gornjem slučaju
pisao ja, a ostatak je generiralo računalo (odnosno python
interpreter).

Ovaj način pokretanja python interpretera se zove interaktivni. U
principu sve programe u ovom poglavlju mozete pokretati u interaktivnom
načinu jednostavno tako da upisujete liniju po liniju programa.

\section{Komentari u programu}

	Gornji program je sasvim mogao izgledati i ovako:

	\sourcee{
		\com{\# "Hello world" (c) 2001 by "Trlababalan" d.d.}\\
		\wrd{print} "Hello world"
	}

	\textbf{Komentar:}
	Prva linija, ona koja započinje sa znakom \# je \emph{komentar}.
	Kad interpreter dođe do ovog znaka on jednostavno preskoči sve
	što nađe nakon tog znaka pa sve do kraja reda.

	Nije loše stavljati komentare u vaše programe. Oni koji će slijediti
	će u sebi imati puno komentara. 

	Mogli se napisati i ovako:

	\sourcee{
		\com{\# Došao škrtac u restoran pojesti juhu, a u jednom trenutku je\\
		\# morao otići na WC. Naravno, uhvatila ga panika da mu netko\\
		\# ne bi pojeo juhu za to vrijeme, pa je ispod juhe stavio papirić\\
		\# na kojeg je napisao "Pljunuo sam u juhu" -- nadajući se da \\
		\# tako nikome neće pasti na pamet niti primirisati njegovoj tanjuru\\
		\# \\
		\# Kad se vratio, otkrio je da je na dnu njegovog papirića netko\\
		\# nadopisao "\dots i ja"}\\
		\\
		\wrd{print} "Hello world"
	}

	Moze i ovo:

	\sourcee{
		\wrd{print} "Hello world" 
		\com{\# ovaj program ispisuje na monitor: "Hello world"}
	}

	U stvari gornja četiri slučaja predstavljaju jedan te isti program.

	Postoji još jedan način kako se mogu pisati komentari u programu:

	\sourcee{
		\wrd{print} "Hello world"\\
		\com{""" Ovo je samo \\
		komentar """}
	}

	\vspace{3mm}
	Komentar počinje s nizom od tri navodnika i završava s tri navodnika, a
	moze se protezati preko više linija.

\section{Varijable}	

	Promotrite sljedeći primjer:



\sourcee{
\var{x} = 5 \\
\textcolor{green}{\# od sada (do daljnjega) varijabla x ima vrijednost 5}\\
\textcolor{blue}{print} "x je", \var{x} \\
\textcolor{green}{\# ispisuje prvo poruku "x je", a nakon nje vrijednost varijable x}\\
\var{y} = 10\\
\textcolor{blue}{print} "...a y je", \var{y}
}

	Nakon što ga snimite i pokrenete ispis će biti:

\sourcee{
x je sada 5\\
y ima vrijednost 10
}

	\textbf{Objašnjenje:} u prevom redu programa, gdje piše \verb|x = 5| smo
	definirali varijablu \verb|x|. \verb|x| od sada ima vrijednost 5. To ne znači
	da se ta vrijednost neće promijeniti u budućnosti. 
	Od sad, svaki put kad pokušamo ispisati vrijednost x-a dobiti ćemo broj 5.

	\textbf{Napomena:} znak "=" u prvom redu programa ne treba shvatiti kao
	matematičku tvrdnju da je \verb+x+ isto što i 5, odnosno da \verb+x+ ima
	vrijednost 5. Redak "\verb+x = 5+" treba shvatiti kao "Od sada će \verb+x+
	imati rijednost 5" ili "Neka \verb+x+ ima vrijednost 5". Dakle varijabli
	\verb+x+ \underline{pridruzujemo vrijednost} 5.

	U trenutku kad smo u programu napisali \verb|x = 5| kazemo da smo varijablu
	\emph{inicijalizirali}, dakle odredili smo joj neku početnu vrijednost. Ako
	pokušamo ispisati neku neinicijaliziranu varijablu interpreter će nam ispisati
	poruku o grešci:

	\sourcee{
	\textcolor{blue}{print} \var{x} \textcolor{green}{\# ...ali varijabla x nije prethodno inicijalizirana!}
	}

	Rezultat je:

	\sourcee{
Traceback (most recent call last):
\\
File "primjer.py", line 1, in ?
\\
print x
\\
NameError: name 'x' is not defined
	}

	Obratite paznju na to da vam se točno kaze gdje je greška u programu
	("line 1"). To nam sad i nije od neke koristi jer naš program ima samo jednu
	liniju, ali kad se stvari zakompliciraju onda ćemo takve informacije jako puno
	koristiti!
	
	Pogledajmo sljedeći primjer:

	\sourcee{
	\var{x} = 5 \textcolor{green}{\# x ima sada vrijednost 5}\\
	\textcolor{blue}{print} "x je", \var{x} \textcolor{green}{\# ispisujemo vrijednost od x}\\
	\var{y} = \var{x} \textcolor{green}{\# sada y poprima vrijednost varijable x}\\
	\textcolor{blue}{print} "y je", \var{y} \textcolor{green}{\# ispisujemo vrijednost od y}\\
	\var{x} = 10 \textcolor{green}{\# x je sad 10}\\
	\textcolor{blue}{print} "x je sad ", \var{x} \textcolor{green}{\# ispisujemo x}\\
	\textcolor{blue}{print} "y je sad", \var{y} \textcolor{green}{\# ispisujemo y}
	}

	Kad pokrenemo program ispis je:

	\sourcee{
	x je 5\\
	y je 5\\
	x je sad 10\\
	y je sad 5
	}

	Prvo je \verb|x| bio 5, tada smo varijabli \verb|y| dodijelili vrijednost varijable
	\verb|x|, dakle 5. Tada smo \verb|x| promijenili da sadrzi 10, ali \verb|y| je
	ostao 5 otprije.

\section{Naredbe}

	Program se sastoji od niza naredbi. Svaka naredba računalu govori šta
	treba činiti. Razmotriti ćemo još jednom jedan primjer programa
	pisanog u Pythonu:
	
	\sourcee{
		\var{x} = 5;\\
		\wrd{if} \var{x} $<$ 10:\\
		\hspace*{10mm}\textcolor{blue}{print} "x je manje od 10"\\
		\wrd{else}:\\
		\hspace*{10mm}\textcolor{blue}{print} "x je veće od 10"
	}

	Python je programski jezik koji zahtijeva da se svaka naredba piše u odvojenom
	retku. Tako, svaki red predstavlja jednu naredbu, u prvom redu naredba "=" kaze
	računalu da varijabli \verb|x| pridruzi vrijednost 5. Nakon toga jedan
	\verb+if+--\emph{uvijet}\footnote{Više riječi o uvjetima kasnije\dots} 
	(naredba \verb+if+)
	provjerava je
	li x manji od 10, ako jest onda ispisuje s \verb+print+ poruku "x je manji od 10", a ako nije onda
	poruku "x je veće od 10". Program će, naravno, ispisati prvu poruku

	Naredbu \verb+print+ smo upoznali kod pisanja "Hello world" programa -- ona ispisuje na monitor ono što se nalazi iza nje.
	O toj naredbi ćemo još puno pričati kasnije.

	U nekim programskim jezicima mogu se pisati
	programi koji u jednoj liniji imaju više naredbi. Pogledajte na primjer kraći
	od dva programa pisana u programskom jeziku Perl u odjeljku 1.3.1. -- ono za
	što nam u Pythonu treba 5 redova mozemo u Perlu napisati dva reda (mogo bi
	se i u jednom).
	Mozda vam se to moze činiti praktičnim, ali takvi programski jezici
	rezultiraju s programima koji su vrlo teško shvatljivi. U takvom programu je
	vrlo teško naći grešku.

	Pogledajmo još jednom taj program u programskom jeziku Perl:

	\source{Perl:}{
		x = 5;\\
		print x $==$ 10 ? "x je manji od 10$\setminus$n" : "x je veći od 10$\setminus$n";
		}

	Ako znate imalo engleskog onda bi vam prvi primjer pisan u Pythonu trebao biti
	dosta jasan, čak i ako ne znate programirati. Pogledajte sad ovaj drugi
	primjer, on je toliko zgusnut i kriptičan da je programeru-početniku vrlo
	teško shvatiti šta je ovdje šta. Zamislite da sad imate program koji
	ima 1000 ovakvih linija, a znate da se negdje među njima nalazi jedna
	greška?!

	\textbf{Zapamtite:} Uvijek se potrudite pisati pregledne programe. Svaku komandu
	pišite u novu liniju. Svaki slozeniji dio programa popratite s komentarima!
	Desiti će vam se da morate prepraviti program koji ste nekad davno pisali. To
	će biti tim lakše što ste tada taj program preglednije pisali!
	
\section{Imena}

	U zadnja dva primjera u našim programima koristili smo se varijablama. Te
	varijable smo nazvali \verb+x+ i \verb+y+. Naravno, mogli smo ih nazvati i
	drukčije. Probati ćemo nešto naučiti iz sljedećeg primjera:

	\sourcee{
\var{pi} = 3.1415926 \\
\var{v} = 5 \\
\var{V} = 10 \\
\var{nova}\_varijabla = 12345 \\
\var{var1} = 12 \\
\var{var2} = 13 \\
\textcolor{blue}{print} "pi je",\var{pi} \\
\textcolor{blue}{print} "varijabla v ima vrijesnost",\var{v} \\
\textcolor{blue}{print} "varijabla V (veliko slovo!) ima vrijesnost",\var{V} \\
\textcolor{blue}{print} "jos jedna varijabla:", \var{nova\_varijabla} \\
\textcolor{blue}{print} "var1 je", \var{var1}, ", a var2 je", \var{var2} 
	}

	Prvih 6 linija programa varijablama pridruzuje neke vrijednosti. Imena tih
	varijabli su 
	\verb+pi+,
	\verb+v+,
	\verb+V+,
	\verb+nova_varijabla+,
	\verb+var1+ i 
	\verb+var2+.

	\textbf{Vazno:} Uočite da Python razlikuje velika i mala
	slova: varijabli \verb+v+ smo
	pridruzili vrijednost 5, a varijabli \verb+V+ (veliko slovo V) vrijednost 10.
	Kasnije u programu se njih tretira kao dvije različite varijable!

	Kad daj program upišete i pokrenete dobiti ćete:

	\sourcee{
	pi je 3.1415926\\
	varijabla v ima vrijesnost 5\\
	varijabla V (veliko slovo!) ima vrijesnost 5\\
	jos jedna varijabla: 12345\\
	var1 je 12 , a var2 je 13
	}

	\textbf{Zapamtite:} Imena varijabli se mogu sastojati kombinacije velikih i malih
	slova, decimalnih znamenaka
	i znaka "\_"\footnote{Engleski: underscore}. Decimalna znamenka se ne smije
	nalaziti na prvom mjestu imena varijable. Razlikuju se velika i mala slova; dakle
	\verb+kikiriki = 5+ i \verb+Kikiriki = 10+ su dvije različite varijable s
	različitim vrijednostima! 

	Ima još jedno pravilo kojeg se morate drzati kod određivanja imena
	varijable; ime varijable \emph{ne smije biti ključna riječ!}

\section{Ključne riječi}

	Ključne riječi su "riječi" ili nizovi znakova koje su rezervirane za
	korištenje u programskom jeziku i ne smiju se koristiti za imenovanje varijabli
	ili nekih drugih dijelova programa koje određuje programer. Već smo se
	sreli s  naredbom \verb+print+. \verb+print+ je ključna riječ i ne smije
	biti korištena kao ime varijable. Mozemo našu varijablu imenovati npr
	\verb+var_print+ ili \verb+_print+ ili \verb+print1+, ali nikako \verb+print+!

	Svaki programski jezik ima određeni broj ključnih riječi. Ključne
	riječi u programskom jeziku Python su:

\begin{center}
\begin{tabular}{llll}
and & del & for & is \\
raise & assert & elif & from\\
lambda & return & break & else\\
global & not & try & class\\
except & if & or & while\\
continue & exec & import & pass\\
def & finally & in & print
\end{tabular}
\end{center}

	Radi praktičnosti, u programima koji se nalaze u ovoj knjizici ključne
	riječi su prikazane u plavoj boji.


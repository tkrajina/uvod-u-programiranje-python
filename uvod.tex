\chapter*{Uvod}
\addcontentsline{toc}{chapter}{Uvod}

Kad su se prije dvadesetak godina po\v cela pojavljivati prva osobna ra\v cunala, ona
su bila bitno druk\v cija od dana\v snjih -- rad na njima nije bio toliko intuitivan i
jednostavan kao \v sto je rad s dana\v snjim ra\v cunalima. Ipak ljudima koji su tada
bili djeca doti\v cna ra\v cunala predstavljala izazov, a mnogi dana\v snji
in\v zinjeri i znanstvenici su u to vrijeme po\v celi eksperimentiranje s naredbama
programskih jezika koji su im bili na raspolaganju. Veliki dio tih malih "hakera" je s
vremenom postalo informati\v carima, ali i oni koji to nisu sigurno svoj uspjeh u
znanosti duguju iskustvima s programiranjem, jer programiranje razvija nekoliko
izuzetno va\v znih sposobnosti:

\begin{itemize}
   \item sposobnost egzaktnog i logi\v ckog razmi\v sljanja,
   \item matemati\v cku intuiciju,
   \item vizualnu predod\v zbu problema i
   \item sposobnost kreiranja matemati\v ckog modela za problem pred koji smo
   posta\-vlje\-ni.
\end{itemize}

Danas ra\v{c}unala nisu egzoti\v{c}ne sprave koje si mogu dopustiti samo oni
najimu\'{c}niji, ali postoji jedan veliki problem.
Ra\v cunala su danas dovoljno intuitivna i jednostavna za kori\v stenje da u stvari ne
nude neki veliki izazov nadarenima. To ne zna\v ci da ih oni ne koriste, ali
pitanje je koliko ih kreativno koriste? Internet je danas \emph{in}, ali ako ga ljudi
koriste za \emph{chat}-anje ili tra\v zenje informacija o omiljenoj grupi mo\v zemo se
pitati iskori\v stavaju li oni tu mogu\'cnost koliko mogu? Programiranje koje je nekad
bilo izazov danas je to prestalo biti jer programski jezici s kojima se oni susre\' cu
u nisu toliko jaki da bi oni mogli raditi programe koji bi barem izgledom sli\v{c}ili
onima koji se besplatno dobiju na CD-ovima raznih informati\v ckih \v casopisa.

Ovom knji\v{z}icom (skriptom, knji\v{z}uljkom) namjera mi je one mla\dj{}e (i koji se
tako osje\'{c}aju)
uvesti u svijet programiranja. 

Prije svega napomenuo bih da "nau\v{c}iti programirati" uop\'{c}e nije jednostavno.
Moje je iskustvo da dobar programer mo\v{z}e biti samo osoba koja je dovoljno
inteligentna (u matemati\v{c}ko-logi\v{c}kom smislu). Proces u\v{c}enja programiranja
je dug i naporan. Biti \'{c}ete bombardirani hrpom \v{c}injenica od kojih ne\'{c}e sve
biti toliko va\v{z}ne da ih morate pamtiti napamet (ali mogu\'{c}e je da \'{c}e vam
prije ili kasnije trebati). Jedan od te\v{z}ih zadataka u cijelom tom procesu je i
sposobnost da znate prepoznate ono \v{s}to je bitno od onog \v{s}to nije.
Kad i ako jednog dana uspijete savladati neki programski jezik to \'{c}e vam
sigurno predstavljati veliko zadovoljstvo.

\section*{Kako se u\v{c}i programirati?}
\addcontentsline{toc}{section}{Kako se u\v{c}i programirati?}

	"Nau\v{c}iti programirati" je dug i naporan proces koji traje godinama i nikad ne
	prestaje. Ne\v{s}to jednostavnije je "nau\v{c}iti odre\dj{}eni programski jezik",
	ali nije dovoljno. Kad i ako nau\v{c}ite programirati uprogramskom jeziku Python,
	jo\v{s} niste niti blizu tome da se mo\v{z}ete smatrati dobrim programerom.

	Da bi postali dobar programer morate:
	\begin{itemize}
		\item puno vje\v{z}bati; rje\v{s}avati razne programerske probleme, 
			poku\v{s}avati na\'{c}i uvijek bolja rje\v{s}enja,
		\item nau\v{c}iti nekoliko programskih jezika (\v{s}to vi\v{s}e to bolje). 
			Kad savladate 3-4 programska jezika otkriti \'{c}ete da novi programski
			jezici nisu niti otprilike onoliko te\v{s}ki koliko je to bilo na
			po\v{c}etku.
		\item \v{c}itati knjige i prou\v{c}avati programe iskusnih programera
		\item stvoriti na\v{c}in matemati\v{c}ko-logi\v{c}ki na\v{c}in razmi\v{s}ljanja
			razmi\v{s}ljanja za rje\v{s}avanje raznih problema
		\item \dots i za kraj; morate imati \emph{strast} za programiranjem. Ako vam se
			ikad desi da pi\v{s}ete neki program i nai\dj{}ete naproblem kojeg ne znate
			rije\v{s}iti. Promatrajte sebe; Kako reagirate u tom trenutku? Ako odmah
			odustajete, onda ovo nije za vas. Ako ste uporni i provodite sate, dane 
			poku\v{s}avaju\'{c}i na\'{c}i izlaz ili gre\v{s}ku u nekom va\v{s}em programu
			onda je ovo pravi izazov za vas!
	\end{itemize}

\section*{Za\v{s}to Python?}
\addcontentsline{toc}{section}{Za\v{s}to Python}

	Ovih nekoliko redova je namijenjeno ljudima koji smatraju da je najbolji programski
	jezik za po\v{c}eti u\v{c}iti programiranje Pascal, BASIC, Logo ili C.

	Dakle, Python zato jer\dots

	\begin{itemize}
		\item Python ima \v{c}istu sintaksu koja programera prisiljava da pi\v{s}e
			pregledne programe (za razliku od kripti\v{c}ne sintakse programskog jezika
			C)
		\item je Python \v{z}ivi programski jezik kojeg programeri koriste i koji \'{c}e
			se sve vi\v{s}e koristiti (za razliku od BASIC-a i Logo-a)
		\item Pythonova stroga sintakti\v{c}ka pravila programera \emph{prisiljavaju} na
			neke vrlo dobre programerske navike (za razliku od BASIC-a)
		\item je Python objektno objektno orijentiran, pa se za u\v{c}enje OO programiranja
			mo\v{z}e koristiti isti jezik, a ne u\v{c}iti jedan za strukturalno, a jedan
			za OO programiranje (za razliku od BASIC-a i C-a)
		\item se u Pythonu, iako je zami\v{s}ljen kao OO jezik bez problema mogu pisati 
			programi koji nemaju niti traga objektnom programiranju (za razliku od Jave)
		\item je Python besplatan i dostupan na skoro svim platformama koje vam padaju
			na pamet
	\end{itemize}

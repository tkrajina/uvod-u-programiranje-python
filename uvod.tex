\chapter*{Uvod}
\addcontentsline{toc}{chapter}{Uvod}

Kad su se prije dvadesetak godina počela pojavljivati prva osobna računala, ona
su bila bitno drukčija od današnjih -- rad na njima nije bio toliko intuitivan i
jednostavan kao što je rad s današnjim računalima. Ipak ljudima koji su tada
bili djeca dotična računala predstavljala izazov, a mnogi današnji
inzinjeri i znanstvenici su u to vrijeme počeli eksperimentiranje s naredbama
programskih jezika koji su im bili na raspolaganju. Veliki dio tih malih "hakera" je s
vremenom postalo informatičarima, ali i oni koji to nisu sigurno svoj uspjeh u
znanosti duguju iskustvima s programiranjem, jer programiranje razvija nekoliko
izuzetno vaznih sposobnosti:

\begin{itemize}
   \item sposobnost egzaktnog i logičkog razmišljanja,
   \item matematičku intuiciju,
   \item vizualnu predodzbu problema i
   \item sposobnost kreiranja matematičkog modela za problem pred koji smo
   posta\-vlje\-ni.
\end{itemize}

Danas računala nisu egzotične sprave koje si mogu dopustiti samo oni
najimućniji, ali postoji jedan veliki problem.
Računala su danas dovoljno intuitivna i jednostavna za korištenje da u stvari ne
nude neki veliki izazov nadarenima. To ne znači da ih oni ne koriste, ali
pitanje je koliko ih kreativno koriste? Internet je danas \emph{in}, ali ako ga ljudi
koriste za \emph{chat}-anje ili trazenje informacija o omiljenoj grupi mozemo se
pitati iskorištavaju li oni tu mogu\'cnost koliko mogu? Programiranje koje je nekad
bilo izazov danas je to prestalo biti jer programski jezici s kojima se oni susreću
u nisu toliko jaki da bi oni mogli raditi programe koji bi barem izgledom sličili
onima koji se besplatno dobiju na CD-ovima raznih informatičkih časopisa.

Ovom knjizicom (skriptom, knjizuljkom) namjera mi je one mlađe (i koji se
tako osjećaju)
uvesti u svijet programiranja. 

Prije svega napomenuo bih da "naučiti programirati" uopće nije jednostavno.
Moje je iskustvo da dobar programer moze biti samo osoba koja je dovoljno
inteligentna (u matematičko-logičkom smislu). Proces učenja programiranja
je dug i naporan. Biti ćete bombardirani hrpom činjenica od kojih neće sve
biti toliko vazne da ih morate pamtiti napamet (ali moguće je da će vam
prije ili kasnije trebati). Jedan od tezih zadataka u cijelom tom procesu je i
sposobnost da znate prepoznate ono što je bitno od onog što nije.
Kad i ako jednog dana uspijete savladati neki programski jezik to će vam
sigurno predstavljati veliko zadovoljstvo.

\section*{Kako se uči programirati?}
\addcontentsline{toc}{section}{Kako se uči programirati?}

	"Naučiti programirati" je dug i naporan proces koji traje godinama i nikad ne
	prestaje. Nešto jednostavnije je "naučiti određeni programski jezik",
	ali nije dovoljno. Kad i ako naučite programirati uprogramskom jeziku Python,
	još niste niti blizu tome da se mozete smatrati dobrim programerom.

	Da bi postali dobar programer morate:
	\begin{itemize}
		\item puno vjezbati; rješavati razne programerske probleme, 
			pokušavati naći uvijek bolja rješenja,
		\item naučiti nekoliko programskih jezika (što više to bolje). 
			Kad savladate 3-4 programska jezika otkriti ćete da novi programski
			jezici nisu niti otprilike onoliko teški koliko je to bilo na
			početku.
		\item čitati knjige i proučavati programe iskusnih programera
		\item stvoriti način matematičko-logički način razmišljanja
			razmišljanja za rješavanje raznih problema
		\item \dots i za kraj; morate imati \emph{strast} za programiranjem. Ako vam se
			ikad desi da pišete neki program i naiđete naproblem kojeg ne znate
			riješiti. Promatrajte sebe; Kako reagirate u tom trenutku? Ako odmah
			odustajete, onda ovo nije za vas. Ako ste uporni i provodite sate, dane 
			pokušavajući naći izlaz ili grešku u nekom vašem programu
			onda je ovo pravi izazov za vas!
	\end{itemize}

\section*{Zašto Python?}
\addcontentsline{toc}{section}{Zašto Python}

	Ovih nekoliko redova je namijenjeno ljudima koji smatraju da je najbolji programski
	jezik za početi učiti programiranje Pascal, BASIC, Logo ili C.

	Dakle, Python zato jer\dots

	\begin{itemize}
		\item Python ima čistu sintaksu koja programera prisiljava da piše
			pregledne programe (za razliku od kriptične sintakse programskog jezika
			C)
		\item je Python zivi programski jezik kojeg programeri koriste i koji će
			se sve više koristiti (za razliku od BASIC-a i Logo-a)
		\item Pythonova stroga sintaktička pravila programera \emph{prisiljavaju} na
			neke vrlo dobre programerske navike (za razliku od BASIC-a)
		\item je Python objektno objektno orijentiran, pa se za učenje OO programiranja
			moze koristiti isti jezik, a ne učiti jedan za strukturalno, a jedan
			za OO programiranje (za razliku od BASIC-a i C-a)
		\item se u Pythonu, iako je zamišljen kao OO jezik bez problema mogu pisati 
			programi koji nemaju niti traga objektnom programiranju (za razliku od Jave)
		\item je Python besplatan i dostupan na skoro svim platformama koje vam padaju
			na pamet
	\end{itemize}

\chapter{Tipovi podataka}

\section{Jednostavni tipovi podataka}

	Varijable u dosadašnjim programima se kao vrijednosti imale brojeve, ali
	mogle su sadrzavati i razne druge vrste podataka.

	Ovisno o tome koliko je slozen problem kojeg zelimo riješiti s
	programom podaci s kojima trebamo raditi mogu biti i vrlo slozeni. Okvirno
	podatke mozemo podijeliti na dva jednostavne i slozene. Slozeni tipovi
	podataka su oni koji u sebi mogu sadrćavati nekoliko drugih podataka.

	Jednostavni su cjelobrojni tip, realni brojevi i stringovi.

\subsection{Cijeli i realni brojevi}

	Cijeli brojevi su svi prirodni brojevi, nula i brojevi suprotni prirodnim
	brojevima.

	Kad kazemo realni brojevi u programiranju obično mislimo samo na relalne
	brojeve koje mozemo zapisati u s točno određenim brojem decimalnih
	mjesta. Za svako računalo i programski jezik postoje točne granice koliko
	najvieš decimalnih mjesta mogu sadrzćavati, koja je najmanja, a koja
	najveća moguća brojevna vrijednost koju mozemo koristiti itd.

	\sourcee{
	\var{broj} = 12.13 \textcolor{green}{\# realni broj}\\
	\var{c} = 12 \textcolor{green}{\# cijeli broj}\\
	\var{r} = 12.0 \textcolor{green}{\# realni broj (jer ima decimalnu tocku!)}
	}

	U zadnjem slučaju varijabla "r" sadrzava realan broj jer je 12.0
	opisan kao decimalan broj. 

	Realne brojeve mozemo napisati i u "znanstvenom obliku", dakle u obliku $a\cdot
	10^{b}$.

	\vspace{3mm}
	\begin{tabular}{l|l}
		Broj & u Python programu\\
		\hline
		$5\cdot 10^{13}$ & \verb+5e13+\\
		$3.56\cdot 10^{-17}$ & \verb+3.56e-17+\\
	\end{tabular}
	\vspace{3mm}

	Brojevi u heksadecimalnom ili oktalno zapisu i imaginarni brojevi:

	\vspace{3mm}
	\begin{tabular}{l|l|l}
		& Zapis & U programu \\	
		\hline
		Heksadecimalni & 177 & 0177 \\
		\hline
		Oktalni & BAB7& 0xBAB7 \\
		\hline
		Imaginarni & $i$ & 1j \\
		\hline 
		Imaginarni & $0.5i$ & 0.5j \\
		\hline 
		Imaginarni & $2.5+3i$ & 2.5+3j \\
	\end{tabular}
	\vspace{3mm}

	Ukoliko radimo s cjelobrojnim varijablama i veličinama, računalo nam
	postavlja određene granice. Ne mozemo računati s \emph{običnim}
	cijeli brojem koji ima 20 znamenaka. Kad nam je zbog nekog slozenog računa
	potrebno raditi s tako veliki brojevima moramo iza samog broja dodati "L"\footnote{Ovo
	"L" moze slobodno biti napisano i malim slovom "l", ali ovdje je napisano "L"
	da se ne bi pomiješalo "l" (malo "l") s "I" (veliko "i")}. Dakle ne
	ne bi napisali 
	
	"a = 128904389523789123789"

	nego\dots

	"a = 128904389523789123789L"

	Probajte oba slučaja napisati kao dio jednog vašeg programa i nakon toga
	probati ispisati tu varijablu s "print a" i pogledajte što se desilo!

\subsection{Stringovi}

	String je niz znakova proizvoljne duzine. Član stringa moze biti svaki
	simbol kojeg mozete dobiti pritiskom na neku tipku tastature, a i omnogi drugi.
	Stringovi se zapisuju u navodnicima. Dakle primjeri stringova su 
	>>Ovo je string"123"<<, >>'sdjkl'<< ili >>"""String"""<<

\textbf{Zapamtite:} Stringove (nizove znakova) mozemo zapisati
na tri načina: 

	\begin{itemize}
		\item Unutar dvostrukih navodnika -- " 
		\item Unutar jednostrukih navodnika -- '
		\item Ograničenih (na početku i na kraju) s nizom od tri dvostruka
			navodnika
	\end{itemize}

	Ponekad će nam trebati da unutar stringa moramo imati neki drugi navodnik. Da
	bi to postigli promotrimo sljedeći primjer.
	
	\sourcee{
		\var{str} = "Ovo je jedan 'string'"\\
		\textcolor{blue}{print} \var{str}
	}

	Ispisati će

	\sourcee{
		Ovo je jedan 'string'
	}

	Postoji još nekoliko načina:

	\sourcee{
		\var{str} = "Ovo je jedan 'string'" \\
		\textcolor{blue}{print} str \\
		\var{str2} = 'Ovo je jos jedan "string"' \\
		\textcolor{blue}{print} \var{str2} \\
		\var{str3} = \textcolor{green}{"""Ovo je jedan "string" """} \\
		\textcolor{blue}{print} \var{str3} \\
		\var{str4} = "Ovo je jedan $\setminus$"string$\setminus$"" \\
		\textcolor{blue}{print} \var{str4}
	}

	Rezultat nakon pokretanja programa je:

	\sourcee{
		Ovo je jedan 'string' \\
		Ovo je jos jedan "string" \\
		Ovo je jedan "string" \\
		Ovo je jedan "string"
	}

	Dakle, dvostruki navodnik u stringu mozete dobiti tako da string
	ograničite s jednostrukim navodnicima ili s nizom trostrukih navodnika. I,
	postoji još jedan izuzetno vazan način, a to je ono 
	>>str4 = "Ovo je jedan \"string\""<<
	Tamo gdje zelimo da nam se u varijabli nalazi jednostruki ili dvostruki
	navodnik jednostavimo \verb+\"+. Isto tako bi mogli staviti i \verb+\'+. 
	
	Ako recimo zelimo da se naš string sastoji samo od jednog dvostrukog
	navodnika mozemo napisati \verb+a = "\""+, a ako zelimo da se sastoji od
	znaka \verb+\\+ i " napisali bi \verb+a = "\\\""+ -- prvi i zadnji navodnik su oznake gdje
	počinje, a gdje završava string. Nakon prvog navodnika niz \verb+\\+ znači
	da se tu nalazi simbol \verb+\+, a \verb+\+" je simbol dvostrukog navodnika.
	
	Postoji
	određeni broj znakova i i simbola koje mozemo staviti u string samo
	kombinacijom \verb+\+ i taj znak ili neko slovo:

	{\normalsize
	\begin{tabular}{ll}
		\verb+\+$<$newline$>$ & Ignored \\
		\verb+\\+ & Backslash (\verb+\+) \\
		\verb+\'+ & Jednostruki navodnik (') \\
		\verb+\"+ & Dvostruki navodnik (") \\
		\verb+\a+ & ASCII Bell (BEL)\\
		\verb+\b+ & ASCII Backspace (BS)\\
		\verb+\f+ & ASCII Formfeed (FF)\\
		\verb+\n+ & ASCII Linefeed (LF)\\
		\verb+\N{name}+ & Character named name in the Unicode database (Unicode only)\\
		\verb+\r+ & ASCII Carriage Return (CR)\\
		\verb+\t+ & ASCII Horizontal Tab (TAB)\\
		\verb+\uxxxx+ & Character with 16-bit hex value xxxx (Unicode only)\\
		\verb+\Uxxxxxxxx+ & Character with 32-bit hex value xxxxxxxx (Unicode only)\\
		\verb+\v+ & ASCII Vertical Tab (VT)\\
		\verb+\ooo+ & ASCII character with octal value ooo\\
		\verb+\xhh+ & ASCII character with hex value hh\\
	\end{tabular}
	}

	\vspace{2mm}
Veliku većinu njih mozete slobodno zaboraviti, ali ima
nekoliko njih koje ćete često koristiti: \verb+\<newline>+
(ovdje \verb+<newline>+ predstavlja tipku "enter", "newline" ili
"return"). Interpreter to jednostavno ignorira, što je jako
korisno ako imamo string koji je prevelik za jedan red pa ga u
programu zelimo imati napisanog u više redova.

Izuzetno vazna je i kombinacija \verb+\n+ -- kada u string
stavimo tu kombinaciju, pri ispisu stringa će na tom mjestu
računalo preći u novi red.

	\sourcee{
		\var{a} = "Ovo je jedan $\setminus$"string$\setminus$" koji je $\setminus$ \\
		toliko dug da mi ga je malo nezgodno $\setminus$ \\
		imati u jednom redu, pa sam ga napisao $\setminus$ \\
		u vise redova" \\
		\var{b} = "A, ovo je jedan$\setminus$n$\setminus$n$\setminus$n$\setminus$nhmmm..." \\
		\textcolor{blue}{print} \var{a} \\
		\textcolor{blue}{print} \var{b}
	}

	Sadrzaj stringa a je "Ovo je jedan $\setminus$"string$\setminus$" koji je toliko dug da mi ga je malo nezgodno imati u jednom redu, pa sam ga napisao u vise redova",
	a kad budemo ispisivali varijablu b vidjeti ćemo šta se zbiva s onim
	\verb+\n+ -- svaki put kad ga računalo "sretne" otići će u novi red.
	Dakle ispisati će "A ovo je jedan" zatim tri puta novi red (dakle tri razmaka
	od jedan red) i onda "hmmmm...".

	Rezultat je dakle:

	\sourcee{
	Ovo je jedan "string" koji je toliko dug da mi ga je malo nezgodno imati u jednom
	\\
	redu, pa sam ga napisao u vise redova
	\\
	A, ovo je jedan
	\\
\ 
	\\
\ 
	\\
\ 
	\\
	hmmm...
	}

Uočite da je računalo, ipak, string \verb+a+ napisalo u
dva reda i to jednostavno zato što mu nije stalo u jedan red.
Da smo imali dovoljno velik monitor bilo bi napisano sve u jednom
redu za razliku od string b u kojemu će uvijek ispisati ona
tri prazna reda upravo zato što smo mi eksplicitno traćili
da oni tu budu.

Još samo jedna zadnja napomena. Trebate znati razlikovati
između \verb+a = 123+ i \verb+a = "123"+. U prvom slučaju
varijabla \verb+a+ će sadrzavati \underline{broj} 123 i s
njim mozemo raditi sve ono što mozemo raditi s brojevima,
a u drugom slučaju varijabla a sadrzava \underline{string}
"123". Sa brojem 123 ćemo moći normalno računati kao
što općenito mozemo s brojevima, a sa stringom "123"
to ne mozemo.

\subsection{Konverzija tipova}

\dots

\subsection{Varijable i vrste podataka}

Neki programski jezici od programera zahtijevaju da točno odredi
kakve će podatke (odnosno vrste podataka) neka varijabla
sadrzavati. Na početku programa se odredi da će npr.
varijabla \verb+x+ sadrzavati samo cijele brojeve, a ako onda
negdje u programu toj varijabli pokušamo pridruziti neki
string ili realan broj, javiti će nam se greška. Za primjer
takvog programskog jezika pogledajte Pascal-program u 1.3.1.  U
Pythonu svaka varijabla moze sadrzavati podatak bilo kojeg
tipa. Ipak, postoje neke dobre programerske navike, a jedna od njih
je da se trudite varijablama . Dakle, ako na početku programa
imate varijablu \verb+n+ koja ima cjelobrojnu vrijednost --
pozeljno je da ta varijabla i dalje u programu sadrzi cijele
brojeve.

\section{Liste}

Jednostavni tipovi podataka kao što su brojevi i stringovi
često nisu dovoljni (ili pogodni) za rješavanje mnogih
problema. Ako imamo neki jako dug program, često će nam se
desiti da broj varijabli postane prevelik. Pretpostavimo npr. da
imamo program u kojemu se mora raditi s jako velikom količinom
podataka; npr. broj učenika neke škole.

\dots

\section{Riječnici}

\section{Datoteka}

\section{Ostali slozeni tipovi podataka}

\chapter{O programiranju, programskim jezicima i programima}

\section{O programiranju}

	Program je skup naredbi pomoću kojih računalu pokušavamo objasniti kako
	da riješava neki problem.

	Dakle, program pišemo kad treba riješiti neki
	problem. Moze se raditi o nekom
	svakodnevnom problemu (napisati program koji će nam pamtiti brojeve telefona),
	nekom matematičkom ili logičkom zadatku (napisati program
	koji zbraja kvadrate brojeva od 1 do 100000)\dots Računalo samo, ne zna
	rješavati probleme, netko treba prije svega naš problem prevesti na 
	računalu razumljiv jezik.

	Program počinjemo pisati kad \emph{znamo} kako ćemo neki problem
	riješiti, ali nemam dovoljno vremena da to riješavamo na neki klasičan
	način. Na primjer, zamislite da vam u jednom trenutku treba podatak je li
	71283789238881999511\footnote{Broj je prost ako je djeljiv samo s 1 i sa samim sobom} prost broj ili nije. Prije nego li se uhvatite za glavu moramo
	se sloziti da postoji nekoliko vrsta problema:

	\begin{itemize}
		\item[\emph{(a)}] Problem kod kojeg znamo postupak kako doći do rješenja i
			mozemo jednostavno doći do tog riješenja.
		\item[\emph{(b)}] Problemi kod kojih znamo postupak kako doći do 
			riješenja, ali zbog
			nekog razloga nije tako jednostavno doći do riješenja.
		\item[\emph{(c)}] Problemi kod kojih (još) ne znamo postupak do riješenja.
	\end{itemize}

	Tako bi recimo sljedeći zadatak "Nađi peto slovo abecede" sigurno spadao u 
	probleme pod \emph{(a)}, i bilo bi besmisleno pisati program i tjerati računalo
	da rješava jedan tako jednostavan problem. Pitanje prostosti broja 
	71283789238881999511 spada u \emph{(b)}, jer \emph{znamo kako} ispitati je li neki
	broj prost ili nije. Prvo treba provjeriti djeljivost s $2$, pa s $3$, pa s
	$5$\dots Ako ispadne da naš broj nije djeljiv s ničim drugim osim s $1$ i
	sa samim sobom onda jest prost. 

	Postupak je smiješno jednostavan,\footnote{\dots u stvari i nije. Ima tu puno sitnica kako se taj postupak moze
	još više ubrzati.}
	ali pitanje je koliko dugo vremena nam treba
	da završimo. 

	Evo i primjer problema koji moze spadati u skupinu \emph{(c)}: "Pomoću
	brojeva 3, 3, 7, 7 i koristeći elementarne računske operacije trebate
	dobiti broj 24". Sad bi trebalo napisati program s kojim će računalo samo
	riješiti taj problem.

\section{O programskim jezicima}

	Znati postupak za riješavanje nekog problema nije garancija da ćemo taj
	problem i riješiti. 
	Čak i ako budem ikada imao dovoljno vremena da ispitam je li broj
	71283789238881999511 
	prost nitko mi ne moze garantirati da neću negdje
	pogriješiti u računu ili, ako radim s kalkulatorom -- da neću negdje
	pogrešno utipkati broj.
	Htio bih, zato, da taj posao računalo obavi umjesto mene.
	
	Problem je u tome što računalo ne govori moj jezik. 
	Mogao bih se do besvjesti truditi mom PC-ju objašnjavati kako sam ja 
	zamišljao da se riješi neki problem. 

	Računala razumiju samo jedan jezik -- \emph{mašinski jezik}. Naredbe tog
	jezika se sastoje od nula i jedinica (poznati binarni brojevni sustav), a pisanje
	programa u mašinskom jeziku je sve samo ne jednostavno.

	Zato koristimo neke druge programske jezike koji se onda prevode u 
	računalu razumljiv mašinski jezik. 
	Postoje stotine programskih jezika. Nabrojati
	ću samo neke od popularnijih: Python, C, C++, Java, Pascal, Lisp, Fortran, 
	Perl, Forth, PHP, JavaScript, BASIC, Smalltalk, Ada\dots Neki od tih jezika su
	\emph{interpreteri}, neki \emph{kompajleri}. Neki su \emph{objektno orijentirani},
	neki \emph{strukturalni}. Neki su \emph{komercijalni}, neki su \emph{open source},
	ali svi oni imaju neke slične osobine: svi oni koriste \emph{varijable},
	\emph{potprograme}, \emph{naredbe grananje}, razne \emph{kontrolne strukture} i
	tako dalje i tako blize\dots Kada dobro savladate jedan programski
	jezik neće vam predstavljati velik problem savladati mnoge druge.

	O svim ovim terminima će se naći ponešto u ovoj knjizici, ali dva
	su izuzetno bitna pa ću ih objasniti odmah:

\subsection{Interpreteri i kompajleri}

	Već spomenuh da računalo razumije samo mašinski jezik. Kako onda
	računalo moze shvatiti naredbe bilo kojeg od gore nabrojanih jezika?
	Odgovor je jednostavan, niz naredbi u nekom programskom jeziku se \emph{prevode} na
	mašinski jezik. 

	Zamislite da vam netko tko govori \emph{Swahili}\footnote{\emph{Swahili}
	je jedan od jezika koji se govore u Zanzibaru. Pretpostavka je da prosječni
	čitatelj ove knjizice ne razumije Swahili} Zeli reći sljedeće:
	\emph{"Otvori knjigu X na stranici 137. pročitaj prvu riječ u petom retku.
	Ako ta riječ počinje suglasnikom odi u kuhinju i skuhaj ručak, a ako
	počinje samoglasnikom -- idi se prošeći i pusti me na miru"}.
	Govornik Swahilija se nalazi u situaciji u kojoj se nalazi programer, a vi glumite
	računalo koje pokušava shvatiti šta vam ovaj ima za kazati. 
	
	Postoji, naravno i prevoditelj. Prevoditelj moze prevoditi simultano ili
	prevesti cijelu poruku odjednom, a vi ćete onda to pročitati i slijediti
	upute. 

	Ako se prevodi simultano cijela stvar izgleda otprilike ovako ("Sw" je govornik
	Swahilija, "V" ste vi, "V" je prevoditelj):

	\begin{itemize}
		\item[\textbf{Sw:}] ".... .... .... ..... ..... ... .... ..... .... .... ....."\footnote{Zamislite da je ovdje neka rečenica na jeziku Swahili}
		\item[\textbf{P:}] "Sw je rekao da trebaš otvoriti knjigu X"
		\item[\textbf{V:}] (otvarate knjigu X)
		\item[\textbf{Sw:}] "..... .... .... .... 137. ....."
		\item[\textbf{P:}] "Sw kaze da otvoriš stranicu 137."
		\item[\textbf{V:}] (otvarate knjigu 137.)
		\item[\textbf{Sw:}] "..... . .... ....  ....."
		\item[\textbf{P:}] "Sw kaze da nađeš prvu riječ u petom retku."
		\item[\textbf{V:}] (trazite prvu riječ u petom retku)
		\item[\textbf{Sw:}] "... ....  ....."
		\item[\textbf{P:}] "Sw kaze da pogledaš prvo slovo te riječi"
		\item[\textbf{V:}] (OK, prvo slovo te riječi je "r")
		\item[\textbf{Sw:}] "...  ... ... ....  ....."
		\item[\textbf{P:}] "Sw kaze; odi skuhati ručak, ako je to slovo samoglasnik"
		\item[\textbf{V:}] (nije samoglasnik)
		\item[\textbf{Sw:}] "...  ... ... ......  .. ...!"
		\item[\textbf{P:}] "Sw kaze da je najbolje da odeš u šetnju"
		\item[\textbf{V:}] (odlazite u šetnju\dots Bilo je i vrijeme da vas puste na miru!)
	\end{itemize}

	Ovako funkcioniraju programski jezici \emph{interpreteri} -- čitaju naredbu
	programa kojeg ste vi napisali, prevedu ju na mašinski jezik, a računalo to
	onda izvršava, nakon toga pročitaju sljedeću naredbu, prevedu,
	računalo izvršava, pročitaju, prevedu, računalo izvršava\dots

	Drugi način je da prevodioc cijelu poruku posluša do kraja, prevede ju
	na vama razumljiv jezik, a vi to onda idete izvršavati. Programski jezici
	kompajleri rade upravo to -- pročitaju cijeli vaš program, prevedu na
	mašinski jezik, a računalo onda izvrši program kojeg sada
	ima u obliku kojeg razumije.

	Da bi izvršili program u programskom jeziku koji se interpretira morate uvijek
	imati interpreter (prevodilac) već instaliran na svom računalu. Kod
	izvršavanja kompajliranog programa dovoljno je da program jednom kompajlirate,
	odmah dobijete program u obliku kojeg računalo razumije, a moći ćete ga
	pokrenuti na nekom drugom računalu (na kojem taj programski jezik "nije
	instaliran"). 

	Ako na računalu imate program koji je prethodno kompajliran nećete moći
	vidjeti kako taj program izgleda, jer on je u memoriji sačuvan u mašinskom
	jeziku (a kojeg vjerojatno ne razumijete, kao Swahili uostalom, jelte 
	:-)\footnote{Ako lutate bespućima interneta onda znate šta ova
	dvotočka-crtica-zagrada znače, ako vam je surfanje nepoznanica, onda
	pogledajte taj niz znakova tako da ukosite glavu ulijevo. Uz malo truda trebali
	biste vidjeti nasmijano lice}. 
	S druge
	strane, ako u memoriji imate program u nekom od interpretiranih programskih jezika,
	moći ćete slobodno pogledati kako taj program izgleda (editirati kao tekst
	datoteku). Moći ćete ga pokrenuti ako i samo ako imate instaliran i
	interpreter-prevodioc za upravo taj programski jezik.

	I, još samo ovo. Razlika između programskih jezika koji se interpretiraju
	i onih koji se kompajliraju nije točno određena. Postoje, naime programski
	jezici koji se mogu pokretati i interpreterom, a mogu se i kompajlirati. Još
	jedna komplikacija su programski jezici koji se kompajliraju, ali ne u mašinski
	jezik nego u nešto što razumije samo poseban program koji to onda
	interpretira\dots Zaboravite, prezivjeti ćete čak i ako ne shvatite sve
	ove tehnikalije.

\section{O programima}
	
	Program je niz naredbi koji opisuje kako se moze riješiti neki problem. Taj
	niz naredbi mora \emph{točno} i \underline{vrlo detaljno} opisati taj
	postupak. Zamislite da nekom stroju morate opisati kako se jede za ručkom.
	Izgleda vrlo jednostavno, ipak -- razmislite malo koliko tu ima detalja. Kao prvo
	morate sjesti za stol, ali i to ne mozete bez da prethodno ne odmaknete stolicu
	od stola. Kad sjednete, stolicu treba opet pribliziti stolu. OK, idemo na juhu:
	imate zlicu, grabite u tanjur i stavljate u usta. Opet! Jeste li sigurno da je
	juha uopće na tanjuru, ako nije trebate opisati kako staviti juhu u tanjur.
	Zlicom, jel? Hm, jeste li sigurni da ste točno opisali kako se zlicom
	stavlja juha, jer ako zlicu ne drzite pod pravim kutom juha će
	se proliti iz zlice. A još nismo niti počeli s jelom. Zamislite
	koliko komplicirano moze biti ako morate objašnjavati neki specijalitet od
	ribe kao drugo jelo.
	Tko bi imao volje stroju objašnjavati kako se
	čiste riblje kosti?

	\subsection{Primjeri u raznim programskim jezicima}

	OK, uspio sam vas obeshrabriti\dots Molim? Nisam? Hhhmmm\dots Ajmo onda probati
	ovako: slijedi nekoliko programa u različitim programskim jezicima koji rade
	jednu te istu stvar:

	\source{Pascal:}{
		program trlababalan;\\
		integer x;\\
		begin\\
		\hspace*{10mm}x := 5;\\
		\hspace*{10mm}if x $=$ 10 then\\
		\hspace*{20mm}writeln( "x je jednako 10" );\\
		\hspace*{10mm}else\\
		\hspace*{20mm}writeln( "x nije jednako 10" );\\
		end.
	}

	Ili, recimo ovako:

	\source{Perl:}{
		x = 5;\\
		if( x == 10 ) \{\\
		\hspace*{10mm}print "x je jednako 10$\setminus$n";\\
		\}\\
		else \{\\
		\hspace*{10mm}print "x nije jednako 10$\setminus$n";\\
		\}
	}

	Često je u istom programskom jeziku moguće isti problem riješiti na
	različite načine, a programski jezik Perl je posebno poznat po tome. 
	Ovaj gornji programčić iskusan programer bi napisao ovako:

	\source{Perl:}{
		x = 5;\\
		print x $==$ 10 ? "x je manji od 10$\setminus$n" : "x je veći od 10$\setminus$n";
		}

	U Pythonu bi to izgledalo:

	\source{Python:}{
		x = 5;\\
		if x $<$ 10:\\
		\hspace*{10mm}print "x je manje od 10"\\
		else:\\
		\hspace*{10mm}print "x je veće od 10"
	}
	
	U programskom jeziku Java:

	\source{Java:}{
		public class trlababalan \{\\
		\hspace*{10mm}public static void main( String[] args ) \{\\
		\hspace*{20mm}int x = 5;\\
		\hspace*{20mm}if( x == 10 ) \{\\
		\hspace*{30mm}System.out.println( "x je jednako 10" );\\
		\hspace*{20mm}\}\\
		\hspace*{20mm}else \{\\
		\hspace*{30mm}System.out.println( "x nije jednako 10" );\\
		\hspace*{20mm}\}\\
		\hspace*{10mm}\}\\
		\}
	}
	
	U programskomjeziku HP48 kalkulatora:

	\source{hp48:}{
		$<<$ 5 'x' STO x 10 IF == THEN "x je jednako 10" ELSE "x nije jednako 10" END
		MSGBOX $>>$
	}
	

	Vjerujem da ste uočili neke sličnosti u ovim kratkim programčićima.
	Nemojte dopustiti vas ono što ne razumijete obeshrabri. Toga će uvijek
	biti -- na veliku zalost onih koji brzo odustaju, a na zadovoljstvo onima koji
	je "nepoznato" samo još jedan izazov.

\section{O programskom jeziku \emph{Python}}

	Python je programski jezik kojeg je stvorio/kreirao/dizajnirao/smislio Guido Van
	Rossum. Za razliku od komercijalnih programskih jezika, Guido je odlučio da
	njegov programski jezik mora biti svima dostupan. I, ne samo da će biti svima
	dostupan nego nego programeri sami mogu mijenjati dotični jezik prema svojim
	potrebama. S vremenom se stvorila grupa ljudi koji su počeli pisati programe u
	tom jeziku, a ako im se sviđala neka sitnica iz nekog drugog jezika jednostavno
	bi i nju dodali Pythonu.

	Python je interpretirani (ali ne baš u smislu "simultanog prevođenja") i
	objektno orijentirani (ali bez problema mogu se pisati strukturalni programi)
	programski jezik. Zbog svoje vrlo čiste i stroge sintakse je vrlo pogodan da
	bude "prvi programski jezik" ne-programerima. Često je ulogu programskog jezika
	za učenje najčešće imao BASIC, Pascal, Logo i razni drugi jezici. 
	Well\dots Neću u detalje, ali Python je bolji od njih :-)

	Python je i besplatan, a moze se pokrenuti na skoro svakom računalu koje
	vam padne na pamet (osim onih \emph{stvarno} prastarih). Ukoliko zelite
	nastaviti čitanje ove knjige pozeljno bi bilo da s internet adrese
	http://www.python.org/ skinete Python interpreter i instalirate ga na svom
	računalu.

\subsection{"Python kao Monty Python?"}

	\emph{"Python kao Monty Python?"} -- Da, \emph{Python} kao \emph{Monty Python}. Da
	citiram "Python tutorial" (za one koji razumiju engleski):

	\begin{quote}
	By the way, the language is named after the BBC show ``Monty Python's
	Flying Circus'' and has nothing to do with nasty reptiles.  Making
	references to Monty Python skits in documentation is not only allowed,
	it is encouraged!
	\end{quote}


\chapter{O programiranju, programskim jezicima i programima}

\section{O programiranju}

	Program je skup naredbi pomo\'{c}u kojih ra\v{c}unalu poku\v{s}avamo objasniti kako
	da rije\v{s}ava neki problem.

	Dakle, program pi\v{s}emo kad treba rije\v{s}iti neki
	problem. Mo\v{z}e se raditi o nekom
	svakodnevnom problemu (napisati program koji \'{c}e nam pamtiti brojeve telefona),
	nekom matemati\v{c}kom ili logi\v{c}kom zadatku (napisati program
	koji zbraja kvadrate brojeva od 1 do 100000)\dots Ra\v{c}unalo samo, ne zna
	rje\v{s}avati probleme, netko treba prije svega na\v{s} problem prevesti na 
	ra\v{c}unalu razumljiv jezik.

	Program po\v{c}injemo pisati kad \emph{znamo} kako \'{c}emo neki problem
	rije\v{s}iti, ali nemam dovoljno vremena da to rije\v{s}avamo na neki klasi\v{c}an
	na\v{c}in. Na primjer, zamislite da vam u jednom trenutku treba podatak je li
	71283789238881999511\footnote{Broj je prost ako je djeljiv samo s 1 i sa samim sobom} prost broj ili nije. Prije nego li se uhvatite za glavu moramo
	se slo\v{z}iti da postoji nekoliko vrsta problema:

	\begin{itemize}
		\item[\emph{(a)}] Problem kod kojeg znamo postupak kako do\'{c}i do rje\v{s}enja i
			mo\v{z}emo jednostavno do\'{c}i do tog rije\v{s}enja.
		\item[\emph{(b)}] Problemi kod kojih znamo postupak kako do\'{c}i do 
			rije\v{s}enja, ali zbog
			nekog razloga nije tako jednostavno do\'{c}i do rije\v{s}enja.
		\item[\emph{(c)}] Problemi kod kojih (jo\v{s}) ne znamo postupak do rije\v{s}enja.
	\end{itemize}

	Tako bi recimo sljede\'{c}i zadatak "Na\dj{}i peto slovo abecede" sigurno spadao u 
	probleme pod \emph{(a)}, i bilo bi besmisleno pisati program i tjerati ra\v{c}unalo
	da rje\v{s}ava jedan tako jednostavan problem. Pitanje prostosti broja 
	71283789238881999511 spada u \emph{(b)}, jer \emph{znamo kako} ispitati je li neki
	broj prost ili nije. Prvo treba provjeriti djeljivost s $2$, pa s $3$, pa s
	$5$\dots Ako ispadne da na\v{s} broj nije djeljiv s ni\v{c}im drugim osim s $1$ i
	sa samim sobom onda jest prost. 

	Postupak je smije\v{s}no jednostavan,\footnote{\dots u stvari i nije. Ima tu puno sitnica kako se taj postupak mo\v{z}e
	jo\v{s} vi\v{s}e ubrzati.}
	ali pitanje je koliko dugo vremena nam treba
	da zavr\v{s}imo. 

	Evo i primjer problema koji mo\v{z}e spadati u skupinu \emph{(c)}: "Pomo\'{c}u
	brojeva 3, 3, 7, 7 i koriste\'{c}i elementarne ra\v{c}unske operacije trebate
	dobiti broj 24". Sad bi trebalo napisati program s kojim \'{c}e ra\v{c}unalo samo
	rije\v{s}iti taj problem.

\section{O programskim jezicima}

	Znati postupak za rije\v{s}avanje nekog problema nije garancija da \'{c}emo taj
	problem i rije\v{s}iti. 
	\v{C}ak i ako budem ikada imao dovoljno vremena da ispitam je li broj
	71283789238881999511 
	prost nitko mi ne mo\v{z}e garantirati da ne\'{c}u negdje
	pogrije\v{s}iti u ra\v{c}unu ili, ako radim s kalkulatorom -- da ne\'{c}u negdje
	pogre\v{s}no utipkati broj.
	Htio bih, zato, da taj posao ra\v{c}unalo obavi umjesto mene.
	
	Problem je u tome \v{s}to ra\v{c}unalo ne govori moj jezik. 
	Mogao bih se do besvjesti truditi mom PC-ju obja\v{s}njavati kako sam ja 
	zami\v{s}ljao da se rije\v{s}i neki problem. 

	Ra\v{c}unala razumiju samo jedan jezik -- \emph{ma\v{s}inski jezik}. Naredbe tog
	jezika se sastoje od nula i jedinica (poznati binarni brojevni sustav), a pisanje
	programa u ma\v{s}inskom jeziku je sve samo ne jednostavno.

	Zato koristimo neke druge programske jezike koji se onda prevode u 
	ra\v{c}unalu razumljiv ma\v{s}inski jezik. 
	Postoje stotine programskih jezika. Nabrojati
	\'{c}u samo neke od popularnijih: Python, C, C++, Java, Pascal, Lisp, Fortran, 
	Perl, Forth, PHP, JavaScript, BASIC, Smalltalk, Ada\dots Neki od tih jezika su
	\emph{interpreteri}, neki \emph{kompajleri}. Neki su \emph{objektno orijentirani},
	neki \emph{strukturalni}. Neki su \emph{komercijalni}, neki su \emph{open source},
	ali svi oni imaju neke sli\v{c}ne osobine: svi oni koriste \emph{varijable},
	\emph{potprograme}, \emph{naredbe grananje}, razne \emph{kontrolne strukture} i
	tako dalje i tako bli\v{z}e\dots Kada dobro savladate jedan programski
	jezik ne\'{c}e vam predstavljati velik problem savladati mnoge druge.

	O svim ovim terminima \'{c}e se na\'{c}i pone\v{s}to u ovoj knji\v{z}ici, ali dva
	su izuzetno bitna pa \'{c}u ih objasniti odmah:

\subsection{Interpreteri i kompajleri}

	Ve\'{c} spomenuh da ra\v{c}unalo razumije samo ma\v{s}inski jezik. Kako onda
	ra\v{c}unalo mo\v{z}e shvatiti naredbe bilo kojeg od gore nabrojanih jezika?
	Odgovor je jednostavan, niz naredbi u nekom programskom jeziku se \emph{prevode} na
	ma\v{s}inski jezik. 

	Zamislite da vam netko tko govori \emph{Swahili}\footnote{\emph{Swahili}
	je jedan od jezika koji se govore u Zanzibaru. Pretpostavka je da prosje\v{c}ni
	\v{c}itatelj ove knji\v{z}ice ne razumije Swahili} \v{Z}eli re\'{c}i sljede\'{c}e:
	\emph{"Otvori knjigu X na stranici 137. pro\v{c}itaj prvu rije\v{c} u petom retku.
	Ako ta rije\v{c} po\v{c}inje suglasnikom odi u kuhinju i skuhaj ru\v{c}ak, a ako
	po\v{c}inje samoglasnikom -- idi se pro\v{s}e\'{c}i i pusti me na miru"}.
	Govornik Swahilija se nalazi u situaciji u kojoj se nalazi programer, a vi glumite
	ra\v{c}unalo koje poku\v{s}ava shvatiti \v{s}ta vam ovaj ima za kazati. 
	
	Postoji, naravno i prevoditelj. Prevoditelj mo\v{z}e prevoditi simultano ili
	prevesti cijelu poruku odjednom, a vi \'{c}ete onda to pro\v{c}itati i slijediti
	upute. 

	Ako se prevodi simultano cijela stvar izgleda otprilike ovako ("Sw" je govornik
	Swahilija, "V" ste vi, "V" je prevoditelj):

	\begin{itemize}
		\item[\textbf{Sw:}] ".... .... .... ..... ..... ... .... ..... .... .... ....."\footnote{Zamislite da je ovdje neka re\v{c}enica na jeziku Swahili}
		\item[\textbf{P:}] "Sw je rekao da treba\v{s} otvoriti knjigu X"
		\item[\textbf{V:}] (otvarate knjigu X)
		\item[\textbf{Sw:}] "..... .... .... .... 137. ....."
		\item[\textbf{P:}] "Sw ka\v{z}e da otvori\v{s} stranicu 137."
		\item[\textbf{V:}] (otvarate knjigu 137.)
		\item[\textbf{Sw:}] "..... . .... ....  ....."
		\item[\textbf{P:}] "Sw ka\v{z}e da na\dj{}e\v{s} prvu rije\v{c} u petom retku."
		\item[\textbf{V:}] (tra\v{z}ite prvu rije\v{c} u petom retku)
		\item[\textbf{Sw:}] "... ....  ....."
		\item[\textbf{P:}] "Sw ka\v{z}e da pogleda\v{s} prvo slovo te rije\v{c}i"
		\item[\textbf{V:}] (OK, prvo slovo te rije\v{c}i je "r")
		\item[\textbf{Sw:}] "...  ... ... ....  ....."
		\item[\textbf{P:}] "Sw ka\v{z}e; odi skuhati ru\v{c}ak, ako je to slovo samoglasnik"
		\item[\textbf{V:}] (nije samoglasnik)
		\item[\textbf{Sw:}] "...  ... ... ......  .. ...!"
		\item[\textbf{P:}] "Sw ka\v{z}e da je najbolje da ode\v{s} u \v setnju"
		\item[\textbf{V:}] (odlazite u \v{s}etnju\dots Bilo je i vrijeme da vas puste na miru!)
	\end{itemize}

	Ovako funkcioniraju programski jezici \emph{interpreteri} -- \v{c}itaju naredbu
	programa kojeg ste vi napisali, prevedu ju na ma\v{s}inski jezik, a ra\v{c}unalo to
	onda izvr\v{s}ava, nakon toga pro\v{c}itaju sljede\'{c}u naredbu, prevedu,
	ra\v{c}unalo izvr\v{s}ava, pro\v{c}itaju, prevedu, ra\v{c}unalo izvr\v{s}ava\dots

	Drugi na\v{c}in je da prevodioc cijelu poruku poslu\v{s}a do kraja, prevede ju
	na vama razumljiv jezik, a vi to onda idete izvr\v{s}avati. Programski jezici
	kompajleri rade upravo to -- pro\v{c}itaju cijeli va\v{s} program, prevedu na
	ma\v{s}inski jezik, a ra\v{c}unalo onda izvr\v{s}i program kojeg sada
	ima u obliku kojeg razumije.

	Da bi izvr\v{s}ili program u programskom jeziku koji se interpretira morate uvijek
	imati interpreter (prevodilac) ve\'{c} instaliran na svom ra\v{c}unalu. Kod
	izvr\v{s}avanja kompajliranog programa dovoljno je da program jednom kompajlirate,
	odmah dobijete program u obliku kojeg ra\v{c}unalo razumije, a mo\'{c}i \'{c}ete ga
	pokrenuti na nekom drugom ra\v{c}unalu (na kojem taj programski jezik "nije
	instaliran"). 

	Ako na ra\v{c}unalu imate program koji je prethodno kompajliran ne\'{c}ete mo\'{c}i
	vidjeti kako taj program izgleda, jer on je u memoriji sa\v{c}uvan u ma\v{s}inskom
	jeziku (a kojeg vjerojatno ne razumijete, kao Swahili uostalom, jelte 
	:-)\footnote{Ako lutate bespu\'{c}ima interneta onda znate \v{s}ta ova
	dvoto\v{c}ka-crtica-zagrada zna\v{c}e, ako vam je surfanje nepoznanica, onda
	pogledajte taj niz znakova tako da ukosite glavu ulijevo. Uz malo truda trebali
	biste vidjeti nasmijano lice}. 
	S druge
	strane, ako u memoriji imate program u nekom od interpretiranih programskih jezika,
	mo\'{c}i \'{c}ete slobodno pogledati kako taj program izgleda (editirati kao tekst
	datoteku). Mo\'{c}i \'{c}ete ga pokrenuti ako i samo ako imate instaliran i
	interpreter-prevodioc za upravo taj programski jezik.

	I, jo\v{s} samo ovo. Razlika izme\dj{}u programskih jezika koji se interpretiraju
	i onih koji se kompajliraju nije to\v{c}no odre\dj{}ena. Postoje, naime programski
	jezici koji se mogu pokretati i interpreterom, a mogu se i kompajlirati. Jo\v{s}
	jedna komplikacija su programski jezici koji se kompajliraju, ali ne u ma\v{s}inski
	jezik nego u ne\v{s}to \v{s}to razumije samo poseban program koji to onda
	interpretira\dots Zaboravite, pre\v{z}ivjeti \'{c}ete \v{c}ak i ako ne shvatite sve
	ove tehnikalije.

\section{O programima}
	
	Program je niz naredbi koji opisuje kako se mo\v{z}e rije\v{s}iti neki problem. Taj
	niz naredbi mora \emph{to\v{c}no} i \underline{vrlo detaljno} opisati taj
	postupak. Zamislite da nekom stroju morate opisati kako se jede za ru\v{c}kom.
	Izgleda vrlo jednostavno, ipak -- razmislite malo koliko tu ima detalja. Kao prvo
	morate sjesti za stol, ali i to ne mo\v{z}ete bez da prethodno ne odmaknete stolicu
	od stola. Kad sjednete, stolicu treba opet pribli\v{z}iti stolu. OK, idemo na juhu:
	imate \v{z}licu, grabite u tanjur i stavljate u usta. Opet! Jeste li sigurno da je
	juha uop\'{c}e na tanjuru, ako nije trebate opisati kako staviti juhu u tanjur.
	\v{Z}licom, jel? Hm, jeste li sigurni da ste to\v{c}no opisali kako se \v{z}licom
	stavlja juha, jer ako \v{z}licu ne dr\v{z}ite pod pravim kutom juha \'{c}e
	se proliti iz \v{z}lice. A jo\v{s} nismo niti po\v{c}eli s jelom. Zamislite
	koliko komplicirano mo\v{z}e biti ako morate obja\v{s}njavati neki specijalitet od
	ribe kao drugo jelo.
	Tko bi imao volje stroju obja\v{s}njavati kako se
	\v{c}iste riblje kosti?

	\subsection{Primjeri u raznim programskim jezicima}

	OK, uspio sam vas obeshrabriti\dots Molim? Nisam? Hhhmmm\dots Ajmo onda probati
	ovako: slijedi nekoliko programa u razli\v{c}itim programskim jezicima koji rade
	jednu te istu stvar:

	\source{Pascal:}{
		program trlababalan;\\
		integer x;\\
		begin\\
		\hspace*{10mm}x := 5;\\
		\hspace*{10mm}if x $=$ 10 then\\
		\hspace*{20mm}writeln( "x je jednako 10" );\\
		\hspace*{10mm}else\\
		\hspace*{20mm}writeln( "x nije jednako 10" );\\
		end.
	}

	Ili, recimo ovako:

	\source{Perl:}{
		x = 5;\\
		if( x == 10 ) \{\\
		\hspace*{10mm}print "x je jednako 10$\setminus$n";\\
		\}\\
		else \{\\
		\hspace*{10mm}print "x nije jednako 10$\setminus$n";\\
		\}
	}

	\v{C}esto je u istom programskom jeziku mogu\'{c}e isti problem rije\v{s}iti na
	razli\v{c}ite na\v{c}ine, a programski jezik Perl je posebno poznat po tome. 
	Ovaj gornji program\v{c}i\'{c} iskusan programer bi napisao ovako:

	\source{Perl:}{
		x = 5;\\
		print x $==$ 10 ? "x je manji od 10$\setminus$n" : "x je ve\'{c}i od 10$\setminus$n";
		}

	U Pythonu bi to izgledalo:

	\source{Python:}{
		x = 5;\\
		if x $<$ 10:\\
		\hspace*{10mm}print "x je manje od 10"\\
		else:\\
		\hspace*{10mm}print "x je ve\'{c}e od 10"
	}
	
	U programskom jeziku Java:

	\source{Java:}{
		public class trlababalan \{\\
		\hspace*{10mm}public static void main( String[] args ) \{\\
		\hspace*{20mm}int x = 5;\\
		\hspace*{20mm}if( x == 10 ) \{\\
		\hspace*{30mm}System.out.println( "x je jednako 10" );\\
		\hspace*{20mm}\}\\
		\hspace*{20mm}else \{\\
		\hspace*{30mm}System.out.println( "x nije jednako 10" );\\
		\hspace*{20mm}\}\\
		\hspace*{10mm}\}\\
		\}
	}
	
	U programskomjeziku HP48 kalkulatora:

	\source{hp48:}{
		$<<$ 5 'x' STO x 10 IF == THEN "x je jednako 10" ELSE "x nije jednako 10" END
		MSGBOX $>>$
	}
	

	Vjerujem da ste uo\v{c}ili neke sli\v{c}nosti u ovim kratkim program\v{c}i\'{c}ima.
	Nemojte dopustiti vas ono \v{s}to ne razumijete obeshrabri. Toga \'{c}e uvijek
	biti -- na veliku \v zalost onih koji brzo odustaju, a na zadovoljstvo onima koji
	je "nepoznato" samo jo\v s jedan izazov.

\section{O programskom jeziku \emph{Python}}

	Python je programski jezik kojeg je stvorio/kreirao/dizajnirao/smislio Guido Van
	Rossum. Za razliku od komercijalnih programskih jezika, Guido je odlu\v{c}io da
	njegov programski jezik mora biti svima dostupan. I, ne samo da \'{c}e biti svima
	dostupan nego nego programeri sami mogu mijenjati doti\v{c}ni jezik prema svojim
	potrebama. S vremenom se stvorila grupa ljudi koji su po\v{c}eli pisati programe u
	tom jeziku, a ako im se svi\dj{}ala neka sitnica iz nekog drugog jezika jednostavno
	bi i nju dodali Pythonu.

	Python je interpretirani (ali ne ba\v{s} u smislu "simultanog prevo\dj{}enja") i
	objektno orijentirani (ali bez problema mogu se pisati strukturalni programi)
	programski jezik. Zbog svoje vrlo \v{c}iste i stroge sintakse je vrlo pogodan da
	bude "prvi programski jezik" ne-programerima. \v{C}esto je ulogu programskog jezika
	za u\v{c}enje naj\v{c}e\v{s}\'{c}e imao BASIC, Pascal, Logo i razni drugi jezici. 
	Well\dots Ne\'{c}u u detalje, ali Python je bolji od njih :-)

	Python je i besplatan, a mo\v{z}e se pokrenuti na skoro svakom ra\v{c}unalu koje
	vam padne na pamet (osim onih \emph{stvarno} prastarih). Ukoliko \v{z}elite
	nastaviti \v{c}itanje ove knjige po\v{z}eljno bi bilo da s internet adrese
	http://www.python.org/ skinete Python interpreter i instalirate ga na svom
	ra\v{c}unalu.

\subsection{"Python kao Monty Python?"}

	\emph{"Python kao Monty Python?"} -- Da, \emph{Python} kao \emph{Monty Python}. Da
	citiram "Python tutorial" (za one koji razumiju engleski):

	\begin{quote}
	By the way, the language is named after the BBC show ``Monty Python's
	Flying Circus'' and has nothing to do with nasty reptiles.  Making
	references to Monty Python skits in documentation is not only allowed,
	it is encouraged!
	\end{quote}


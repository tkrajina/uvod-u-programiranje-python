\chapter*{Uvod}
\addcontentsline{toc}{chapter}{Uvod}

Kad su se prije dvadesetak godina po\v cela pojavljivati prva osobna ra\v cunala, ona
su bila bitno druk\v cija od dana\v snjih -- rad na njima nije bio toliko intuitivan i
jednostavan kao \v sto je rad s dana\v snjim ra\v cunalima. Ipak ljudima koji su tada
bili djeca doti\v cna ra\v cunala predstavljala izazov, a mnogi dana\v snji
in\v zinjeri i znanstvenici su u to vrijeme po\v celi eksperimentiranje s naredbama
programskih jezika koji su im bili na raspolaganju. Veliki dio tih malih "hakera" je s
vremenom postalo informati\v carima, ali i oni koji to nisu sigurno svoj uspjeh u
znanosti duguju iskustvima s programiranjem, jer programiranje razvija nekoliko
izuzetno va\v znih sposobnosti:

\begin{itemize}
   \item sposobnost egzaktnog i logi\v ckog razmi\v sljanja,
   \item matemati\v cku intuiciju,
   \item vizualnu predod\v zbu problema i
   \item sposobnost kreiranja matemati\v ckog modela za problem pred koji smo
   posta\-vlje\-ni.
\end{itemize}

Danas ra\v{c}unala nisu egzoti\v{c}ne sprave koje si mogu dopustiti samo oni
najimu\'{c}niji, ali postoji jedan veliki problem.
Ra\v cunala su danas dovoljno intuitivna i jednostavna za kori\v stenje da u stvari ne
nude neki veliki izazov nadarenima. To ne zna\v ci da ih oni ne koriste, ali
pitanje je koliko ih kreativno koriste? Internet je danas \emph{in}, ali ako ga ljudi
koriste za \emph{chat}-anje ili tra\v zenje informacija o omiljenoj grupi mo\v zemo se
pitati iskori\v stavaju li oni tu mogu\'cnost koliko mogu? Programiranje koje je nekad
bilo izazov danas je to prestalo biti jer programski jezici s kojima se oni susre\' cu
u nisu toliko jaki da bi oni mogli raditi programe koji bi barem izgledom sli\v{c}ili
onima koji se besplatno dobiju na CD-ovima raznih informati\v ckih \v casopisa.

Ovom knji\v{z}icom (skriptom, knji\v{z}uljkom) namjera mi je one mla\dj{}e (i koji se
tako osje\'{c}aju)
uvesti u svijet programiranja. 

Prije svega napomenuo bih da "nau\v{c}iti programirati" uop\'{c}e nije jednostavno.
Moje je iskustvo da dobar programer mo\v{z}e biti samo osoba koja je dovoljno
inteligentna (u matemati\v{c}ko-logi\v{c}kom smislu). Proces u\v{c}enja programiranja
je dug i naporan. Biti \'{c}ete bombardirani hrpom \v{c}injenica od kojih ne\'{c}e sve
biti toliko va\v{z}ne da ih morate pamtiti napamet (ali mogu\'{c}e je da \'{c}e vam
prije ili kasnije trebati). Jedan od te\v{z}ih zadataka u cijelom tom procesu je i
sposobnost da znate prepoznate ono \v{s}to je bitno od onog \v{s}to nije.
Kad i ako jednog dana uspijete savladati neki programski jezik to \'{c}e vam
sigurno predstavljati veliko zadovoljstvo.

\section*{Kako se u\v{c}i programirati?}
\addcontentsline{toc}{section}{Kako se u\v{c}i programirati?}

	"Nau\v{c}iti programirati" je dug i naporan proces koji traje godinama i nikad ne
	prestaje. Ne\v{s}to jednostavnije je "nau\v{c}iti odre\dj{}eni programski jezik",
	ali nije dovoljno. Kad i ako nau\v{c}ite programirati uprogramskom jeziku Python,
	jo\v{s} niste niti blizu tome da se mo\v{z}ete smatrati dobrim programerom.

	Da bi postali dobar programer morate:
	\begin{itemize}
		\item puno vje\v{z}bati; rje\v{s}avati razne programerske probleme, 
			poku\v{s}avati na\'{c}i uvijek bolja rje\v{s}enja,
		\item nau\v{c}iti nekoliko programskih jezika (\v{s}to vi\v{s}e to bolje). 
			Kad savladate 3-4 programska jezika otkriti \'{c}ete da novi programski
			jezici nisu niti otprilike onoliko te\v{s}ki koliko je to bilo na
			po\v{c}etku.
		\item \v{c}itati knjige i prou\v{c}avati programe iskusnih programera
		\item stvoriti na\v{c}in matemati\v{c}ko-logi\v{c}ki na\v{c}in razmi\v{s}ljanja
			razmi\v{s}ljanja za rje\v{s}avanje raznih problema
		\item \dots i za kraj; morate imati \emph{strast} za programiranjem. Ako vam se
			ikad desi da pi\v{s}ete neki program i nai\dj{}ete naproblem kojeg ne znate
			rije\v{s}iti. Promatrajte sebe; Kako reagirate u tom trenutku? Ako odmah
			odustajete, onda ovo nije za vas. Ako ste uporni i provodite sate, dane 
			poku\v{s}avaju\'{c}i na\'{c}i izlaz ili gre\v{s}ku u nekom va\v{s}em programu
			onda je ovo pravi izazov za vas!
	\end{itemize}

\section*{Za\v{s}to Python?}
\addcontentsline{toc}{section}{Za\v{s}to Python}

	Ovih nekoliko redova je namijenjeno ljudima koji smatraju da je najbolji programski
	jezik za po\v{c}eti u\v{c}iti programiranje Pascal, BASIC, Logo ili C.

	Dakle, Python zato jer\dots

	\begin{itemize}
		\item Python ima \v{c}istu sintaksu koja programera prisiljava da pi\v{s}e
			pregledne programe (za razliku od kripti\v{c}ne sintakse programskog jezika
			C)
		\item je Python \v{z}ivi programski jezik kojeg programeri koriste i koji \'{c}e
			se sve vi\v{s}e koristiti (za razliku od BASIC-a i Logo-a)
		\item Pythonova stroga sintakti\v{c}ka pravila programera \emph{prisiljavaju} na
			neke vrlo dobre programerske navike (za razliku od BASIC-a)
		\item je Python objektno objektno orijentiran, pa se za u\v{c}enje OO programiranja
			mo\v{z}e koristiti isti jezik, a ne u\v{c}iti jedan za strukturalno, a jedan
			za OO programiranje (za razliku od BASIC-a i C-a)
		\item se u Pythonu, iako je zami\v{s}ljen kao OO jezik bez problema mogu pisati 
			programi koji nemaju niti traga objektnom programiranju (za razliku od Jave)
		\item je Python besplatan i dostupan na skoro svim platformama koje vam padaju
			na pamet
	\end{itemize}
\chapter{O programiranju, programskim jezicima i programima}

\section{O programiranju}

	Program je skup naredbi pomo\'{c}u kojih ra\v{c}unalu poku\v{s}avamo objasniti kako
	da rije\v{s}ava neki problem.

	Dakle, program pi\v{s}emo kad treba rije\v{s}iti neki
	problem. Mo\v{z}e se raditi o nekom
	svakodnevnom problemu (napisati program koji \'{c}e nam pamtiti brojeve telefona),
	nekom matemati\v{c}kom ili logi\v{c}kom zadatku (napisati program
	koji zbraja kvadrate brojeva od 1 do 100000)\dots Ra\v{c}unalo samo, ne zna
	rje\v{s}avati probleme, netko treba prije svega na\v{s} problem prevesti na 
	ra\v{c}unalu razumljiv jezik.

	Program po\v{c}injemo pisati kad \emph{znamo} kako \'{c}emo neki problem
	rije\v{s}iti, ali nemam dovoljno vremena da to rije\v{s}avamo na neki klasi\v{c}an
	na\v{c}in. Na primjer, zamislite da vam u jednom trenutku treba podatak je li
	71283789238881999511\footnote{Broj je prost ako je djeljiv samo s 1 i sa samim sobom} prost broj ili nije. Prije nego li se uhvatite za glavu moramo
	se slo\v{z}iti da postoji nekoliko vrsta problema:

	\begin{itemize}
		\item[\emph{(a)}] Problem kod kojeg znamo postupak kako do\'{c}i do rje\v{s}enja i
			mo\v{z}emo jednostavno do\'{c}i do tog rije\v{s}enja.
		\item[\emph{(b)}] Problemi kod kojih znamo postupak kako do\'{c}i do 
			rije\v{s}enja, ali zbog
			nekog razloga nije tako jednostavno do\'{c}i do rije\v{s}enja.
		\item[\emph{(c)}] Problemi kod kojih (jo\v{s}) ne znamo postupak do rije\v{s}enja.
	\end{itemize}

	Tako bi recimo sljede\'{c}i zadatak "Na\dj{}i peto slovo abecede" sigurno spadao u 
	probleme pod \emph{(a)}, i bilo bi besmisleno pisati program i tjerati ra\v{c}unalo
	da rje\v{s}ava jedan tako jednostavan problem. Pitanje prostosti broja 
	71283789238881999511 spada u \emph{(b)}, jer \emph{znamo kako} ispitati je li neki
	broj prost ili nije. Prvo treba provjeriti djeljivost s $2$, pa s $3$, pa s
	$5$\dots Ako ispadne da na\v{s} broj nije djeljiv s ni\v{c}im drugim osim s $1$ i
	sa samim sobom onda jest prost. 

	Postupak je smije\v{s}no jednostavan,\footnote{\dots u stvari i nije. Ima tu puno sitnica kako se taj postupak mo\v{z}e
	jo\v{s} vi\v{s}e ubrzati.}
	ali pitanje je koliko dugo vremena nam treba
	da zavr\v{s}imo. 

	Evo i primjer problema koji mo\v{z}e spadati u skupinu \emph{(c)}: "Pomo\'{c}u
	brojeva 3, 3, 7, 7 i koriste\'{c}i elementarne ra\v{c}unske operacije trebate
	dobiti broj 24". Sad bi trebalo napisati program s kojim \'{c}e ra\v{c}unalo samo
	rije\v{s}iti taj problem.

\section{O programskim jezicima}

	Znati postupak za rije\v{s}avanje nekog problema nije garancija da \'{c}emo taj
	problem i rije\v{s}iti. 
	\v{C}ak i ako budem ikada imao dovoljno vremena da ispitam je li broj
	71283789238881999511 
	prost nitko mi ne mo\v{z}e garantirati da ne\'{c}u negdje
	pogrije\v{s}iti u ra\v{c}unu ili, ako radim s kalkulatorom -- da ne\'{c}u negdje
	pogre\v{s}no utipkati broj.
	Htio bih, zato, da taj posao ra\v{c}unalo obavi umjesto mene.
	
	Problem je u tome \v{s}to ra\v{c}unalo ne govori moj jezik. 
	Mogao bih se do besvjesti truditi mom PC-ju obja\v{s}njavati kako sam ja 
	zami\v{s}ljao da se rije\v{s}i neki problem. 

	Ra\v{c}unala razumiju samo jedan jezik -- \emph{ma\v{s}inski jezik}. Naredbe tog
	jezika se sastoje od nula i jedinica (poznati binarni brojevni sustav), a pisanje
	programa u ma\v{s}inskom jeziku je sve samo ne jednostavno.

	Zato koristimo neke druge programske jezike koji se onda prevode u 
	ra\v{c}unalu razumljiv ma\v{s}inski jezik. 
	Postoje stotine programskih jezika. Nabrojati
	\'{c}u samo neke od popularnijih: Python, C, C++, Java, Pascal, Lisp, Fortran, 
	Perl, Forth, PHP, JavaScript, BASIC, Smalltalk, Ada\dots Neki od tih jezika su
	\emph{interpreteri}, neki \emph{kompajleri}. Neki su \emph{objektno orijentirani},
	neki \emph{strukturalni}. Neki su \emph{komercijalni}, neki su \emph{open source},
	ali svi oni imaju neke sli\v{c}ne osobine: svi oni koriste \emph{varijable},
	\emph{potprograme}, \emph{naredbe grananje}, razne \emph{kontrolne strukture} i
	tako dalje i tako bli\v{z}e\dots Kada dobro savladate jedan programski
	jezik ne\'{c}e vam predstavljati velik problem savladati mnoge druge.

	O svim ovim terminima \'{c}e se na\'{c}i pone\v{s}to u ovoj knji\v{z}ici, ali dva
	su izuzetno bitna pa \'{c}u ih objasniti odmah:

\subsection{Interpreteri i kompajleri}

	Ve\'{c} spomenuh da ra\v{c}unalo razumije samo ma\v{s}inski jezik. Kako onda
	ra\v{c}unalo mo\v{z}e shvatiti naredbe bilo kojeg od gore nabrojanih jezika?
	Odgovor je jednostavan, niz naredbi u nekom programskom jeziku se \emph{prevode} na
	ma\v{s}inski jezik. 

	Zamislite da vam netko tko govori \emph{Swahili}\footnote{\emph{Swahili}
	je jedan od jezika koji se govore u Zanzibaru. Pretpostavka je da prosje\v{c}ni
	\v{c}itatelj ove knji\v{z}ice ne razumije Swahili} \v{Z}eli re\'{c}i sljede\'{c}e:
	\emph{"Otvori knjigu X na stranici 137. pro\v{c}itaj prvu rije\v{c} u petom retku.
	Ako ta rije\v{c} po\v{c}inje suglasnikom odi u kuhinju i skuhaj ru\v{c}ak, a ako
	po\v{c}inje samoglasnikom -- idi se pro\v{s}e\'{c}i i pusti me na miru"}.
	Govornik Swahilija se nalazi u situaciji u kojoj se nalazi programer, a vi glumite
	ra\v{c}unalo koje poku\v{s}ava shvatiti \v{s}ta vam ovaj ima za kazati. 
	
	Postoji, naravno i prevoditelj. Prevoditelj mo\v{z}e prevoditi simultano ili
	prevesti cijelu poruku odjednom, a vi \'{c}ete onda to pro\v{c}itati i slijediti
	upute. 

	Ako se prevodi simultano cijela stvar izgleda otprilike ovako ("Sw" je govornik
	Swahilija, "V" ste vi, "V" je prevoditelj):

	\begin{itemize}
		\item[\textbf{Sw:}] ".... .... .... ..... ..... ... .... ..... .... .... ....."\footnote{Zamislite da je ovdje neka re\v{c}enica na jeziku Swahili}
		\item[\textbf{P:}] "Sw je rekao da treba\v{s} otvoriti knjigu X"
		\item[\textbf{V:}] (otvarate knjigu X)
		\item[\textbf{Sw:}] "..... .... .... .... 137. ....."
		\item[\textbf{P:}] "Sw ka\v{z}e da otvori\v{s} stranicu 137."
		\item[\textbf{V:}] (otvarate knjigu 137.)
		\item[\textbf{Sw:}] "..... . .... ....  ....."
		\item[\textbf{P:}] "Sw ka\v{z}e da na\dj{}e\v{s} prvu rije\v{c} u petom retku."
		\item[\textbf{V:}] (tra\v{z}ite prvu rije\v{c} u petom retku)
		\item[\textbf{Sw:}] "... ....  ....."
		\item[\textbf{P:}] "Sw ka\v{z}e da pogleda\v{s} prvo slovo te rije\v{c}i"
		\item[\textbf{V:}] (OK, prvo slovo te rije\v{c}i je "r")
		\item[\textbf{Sw:}] "...  ... ... ....  ....."
		\item[\textbf{P:}] "Sw ka\v{z}e; odi skuhati ru\v{c}ak, ako je to slovo samoglasnik"
		\item[\textbf{V:}] (nije samoglasnik)
		\item[\textbf{Sw:}] "...  ... ... ......  .. ...!"
		\item[\textbf{P:}] "Sw ka\v{z}e da je najbolje da ode\v{s} u \v setnju"
		\item[\textbf{V:}] (odlazite u \v{s}etnju\dots Bilo je i vrijeme da vas puste na miru!)
	\end{itemize}

	Ovako funkcioniraju programski jezici \emph{interpreteri} -- \v{c}itaju naredbu
	programa kojeg ste vi napisali, prevedu ju na ma\v{s}inski jezik, a ra\v{c}unalo to
	onda izvr\v{s}ava, nakon toga pro\v{c}itaju sljede\'{c}u naredbu, prevedu,
	ra\v{c}unalo izvr\v{s}ava, pro\v{c}itaju, prevedu, ra\v{c}unalo izvr\v{s}ava\dots

	Drugi na\v{c}in je da prevodioc cijelu poruku poslu\v{s}a do kraja, prevede ju
	na vama razumljiv jezik, a vi to onda idete izvr\v{s}avati. Programski jezici
	kompajleri rade upravo to -- pro\v{c}itaju cijeli va\v{s} program, prevedu na
	ma\v{s}inski jezik, a ra\v{c}unalo onda izvr\v{s}i program kojeg sada
	ima u obliku kojeg razumije.

	Da bi izvr\v{s}ili program u programskom jeziku koji se interpretira morate uvijek
	imati interpreter (prevodilac) ve\'{c} instaliran na svom ra\v{c}unalu. Kod
	izvr\v{s}avanja kompajliranog programa dovoljno je da program jednom kompajlirate,
	odmah dobijete program u obliku kojeg ra\v{c}unalo razumije, a mo\'{c}i \'{c}ete ga
	pokrenuti na nekom drugom ra\v{c}unalu (na kojem taj programski jezik "nije
	instaliran"). 

	Ako na ra\v{c}unalu imate program koji je prethodno kompajliran ne\'{c}ete mo\'{c}i
	vidjeti kako taj program izgleda, jer on je u memoriji sa\v{c}uvan u ma\v{s}inskom
	jeziku (a kojeg vjerojatno ne razumijete, kao Swahili uostalom, jelte 
	:-)\footnote{Ako lutate bespu\'{c}ima interneta onda znate \v{s}ta ova
	dvoto\v{c}ka-crtica-zagrada zna\v{c}e, ako vam je surfanje nepoznanica, onda
	pogledajte taj niz znakova tako da ukosite glavu ulijevo. Uz malo truda trebali
	biste vidjeti nasmijano lice}. 
	S druge
	strane, ako u memoriji imate program u nekom od interpretiranih programskih jezika,
	mo\'{c}i \'{c}ete slobodno pogledati kako taj program izgleda (editirati kao tekst
	datoteku). Mo\'{c}i \'{c}ete ga pokrenuti ako i samo ako imate instaliran i
	interpreter-prevodioc za upravo taj programski jezik.

	I, jo\v{s} samo ovo. Razlika izme\dj{}u programskih jezika koji se interpretiraju
	i onih koji se kompajliraju nije to\v{c}no odre\dj{}ena. Postoje, naime programski
	jezici koji se mogu pokretati i interpreterom, a mogu se i kompajlirati. Jo\v{s}
	jedna komplikacija su programski jezici koji se kompajliraju, ali ne u ma\v{s}inski
	jezik nego u ne\v{s}to \v{s}to razumije samo poseban program koji to onda
	interpretira\dots Zaboravite, pre\v{z}ivjeti \'{c}ete \v{c}ak i ako ne shvatite sve
	ove tehnikalije.

\section{O programima}
	
	Program je niz naredbi koji opisuje kako se mo\v{z}e rije\v{s}iti neki problem. Taj
	niz naredbi mora \emph{to\v{c}no} i \underline{vrlo detaljno} opisati taj
	postupak. Zamislite da nekom stroju morate opisati kako se jede za ru\v{c}kom.
	Izgleda vrlo jednostavno, ipak -- razmislite malo koliko tu ima detalja. Kao prvo
	morate sjesti za stol, ali i to ne mo\v{z}ete bez da prethodno ne odmaknete stolicu
	od stola. Kad sjednete, stolicu treba opet pribli\v{z}iti stolu. OK, idemo na juhu:
	imate \v{z}licu, grabite u tanjur i stavljate u usta. Opet! Jeste li sigurno da je
	juha uop\'{c}e na tanjuru, ako nije trebate opisati kako staviti juhu u tanjur.
	\v{Z}licom, jel? Hm, jeste li sigurni da ste to\v{c}no opisali kako se \v{z}licom
	stavlja juha, jer ako \v{z}licu ne dr\v{z}ite pod pravim kutom juha \'{c}e
	se proliti iz \v{z}lice. A jo\v{s} nismo niti po\v{c}eli s jelom. Zamislite
	koliko komplicirano mo\v{z}e biti ako morate obja\v{s}njavati neki specijalitet od
	ribe kao drugo jelo.
	Tko bi imao volje stroju obja\v{s}njavati kako se
	\v{c}iste riblje kosti?

	\subsection{Primjeri u raznim programskim jezicima}

	OK, uspio sam vas obeshrabriti\dots Molim? Nisam? Hhhmmm\dots Ajmo onda probati
	ovako: slijedi nekoliko programa u razli\v{c}itim programskim jezicima koji rade
	jednu te istu stvar:

	\source{Pascal:}{
		program trlababalan;\\
		integer x;\\
		begin\\
		\hspace*{10mm}x := 5;\\
		\hspace*{10mm}if x $=$ 10 then\\
		\hspace*{20mm}writeln( "x je jednako 10" );\\
		\hspace*{10mm}else\\
		\hspace*{20mm}writeln( "x nije jednako 10" );\\
		end.
	}

	Ili, recimo ovako:

	\source{Perl:}{
		x = 5;\\
		if( x == 10 ) \{\\
		\hspace*{10mm}print "x je jednako 10$\setminus$n";\\
		\}\\
		else \{\\
		\hspace*{10mm}print "x nije jednako 10$\setminus$n";\\
		\}
	}

	\v{C}esto je u istom programskom jeziku mogu\'{c}e isti problem rije\v{s}iti na
	razli\v{c}ite na\v{c}ine, a programski jezik Perl je posebno poznat po tome. 
	Ovaj gornji program\v{c}i\'{c} iskusan programer bi napisao ovako:

	\source{Perl:}{
		x = 5;\\
		print x $==$ 10 ? "x je manji od 10$\setminus$n" : "x je ve\'{c}i od 10$\setminus$n";
		}

	U Pythonu bi to izgledalo:

	\source{Python:}{
		x = 5;\\
		if x $<$ 10:\\
		\hspace*{10mm}print "x je manje od 10"\\
		else:\\
		\hspace*{10mm}print "x je ve\'{c}e od 10"
	}
	
	U programskom jeziku Java:

	\source{Java:}{
		public class trlababalan \{\\
		\hspace*{10mm}public static void main( String[] args ) \{\\
		\hspace*{20mm}int x = 5;\\
		\hspace*{20mm}if( x == 10 ) \{\\
		\hspace*{30mm}System.out.println( "x je jednako 10" );\\
		\hspace*{20mm}\}\\
		\hspace*{20mm}else \{\\
		\hspace*{30mm}System.out.println( "x nije jednako 10" );\\
		\hspace*{20mm}\}\\
		\hspace*{10mm}\}\\
		\}
	}
	
	U programskomjeziku HP48 kalkulatora:

	\source{hp48:}{
		$<<$ 5 'x' STO x 10 IF == THEN "x je jednako 10" ELSE "x nije jednako 10" END
		MSGBOX $>>$
	}
	

	Vjerujem da ste uo\v{c}ili neke sli\v{c}nosti u ovim kratkim program\v{c}i\'{c}ima.
	Nemojte dopustiti vas ono \v{s}to ne razumijete obeshrabri. Toga \'{c}e uvijek
	biti -- na veliku \v zalost onih koji brzo odustaju, a na zadovoljstvo onima koji
	je "nepoznato" samo jo\v s jedan izazov.

\section{O programskom jeziku \emph{Python}}

	Python je programski jezik kojeg je stvorio/kreirao/dizajnirao/smislio Guido Van
	Rossum. Za razliku od komercijalnih programskih jezika, Guido je odlu\v{c}io da
	njegov programski jezik mora biti svima dostupan. I, ne samo da \'{c}e biti svima
	dostupan nego nego programeri sami mogu mijenjati doti\v{c}ni jezik prema svojim
	potrebama. S vremenom se stvorila grupa ljudi koji su po\v{c}eli pisati programe u
	tom jeziku, a ako im se svi\dj{}ala neka sitnica iz nekog drugog jezika jednostavno
	bi i nju dodali Pythonu.

	Python je interpretirani (ali ne ba\v{s} u smislu "simultanog prevo\dj{}enja") i
	objektno orijentirani (ali bez problema mogu se pisati strukturalni programi)
	programski jezik. Zbog svoje vrlo \v{c}iste i stroge sintakse je vrlo pogodan da
	bude "prvi programski jezik" ne-programerima. \v{C}esto je ulogu programskog jezika
	za u\v{c}enje naj\v{c}e\v{s}\'{c}e imao BASIC, Pascal, Logo i razni drugi jezici. 
	Well\dots Ne\'{c}u u detalje, ali Python je bolji od njih :-)

	Python je i besplatan, a mo\v{z}e se pokrenuti na skoro svakom ra\v{c}unalu koje
	vam padne na pamet (osim onih \emph{stvarno} prastarih). Ukoliko \v{z}elite
	nastaviti \v{c}itanje ove knjige po\v{z}eljno bi bilo da s internet adrese
	http://www.python.org/ skinete Python interpreter i instalirate ga na svom
	ra\v{c}unalu.

\subsection{"Python kao Monty Python?"}

	\emph{"Python kao Monty Python?"} -- Da, \emph{Python} kao \emph{Monty Python}. Da
	citiram "Python tutorial" (za one koji razumiju engleski):

	\begin{quote}
	By the way, the language is named after the BBC show ``Monty Python's
	Flying Circus'' and has nothing to do with nasty reptiles.  Making
	references to Monty Python skits in documentation is not only allowed,
	it is encouraged!
	\end{quote}

\chapter{Uvod u programiranje}

\section{\v{S}to je danas lijep i sun\v{c}an dan}

Kad savladate nekoliko programskih jezika uo\v{c}iti \'{c}ete
nekoliko pravilnosti koje se pojavljuju kod svih njih. Svi oni
koriste varijable, potprograme, dobar dio njih koristi objekte. Ne
samo to, nego u svim priru\v{c}nicima za programiranje prvi program
ima naziv "Hello world"\footnote{eng.  Zdravo svijete}.

Budu\'{c}i da autor ovog spisa nije nimalo revolucionaran tip, i
mi \'{c}emo zapo\v{c}eti Zdravo-svijetom\footnote{\hspace*{2mm}"Hello
world"}.  Daklem. "Hello world" u Pythonu izgleda ovako:

	\sourcee{
		\wrd{print} "Hello world"
	}

\textbf{Komentar:} Druga linija je ono \v{s}to je bitno u ovom
programu. Ona ka\v{z}e prevoditelju-interpreteru da ispi\v{s}e na
monitor niz znakova "Hello world" (bez navodnika, navodnici samo
slu\v{z}e da se zna gdje po\v{c}inje i gdje zavr\v{s}ava ono \v{s}to
\v{z}elimo ispisati).

\subsection{Snimanje programa}

Jedan od gornjih "zdravo-svijete" programa napi\v{s}ite u nekom
tekst editoru (ako radie na Windowsima onda to mo\v{z}e biti NotePad,
ako ste na Linuxu vi, joe, emacs, ako na Macintoshu\dots)

\dots

\subsection{Pokretanje programa}

Na windowsima; nakon \v{s}to ste program snimili kao datoteku s ekstenzijom
.py, na\dj{}ite gdje je ikona s tom datotekom i dvaput kliknite na nju.

Na linuxu; u komandnoj liniji treba napisati: "python $<$ime\_datoteke$>$".
Naravno, $<$ime\_datoteke$>$ zamijenite s imenom kojeg ste upisali pri
snimanju.

\dots

\subsection{Python kao kalkulator}

Pokrenete li\footnote{Na windowsima tako da kliknete na ikonu, a
na linuxu tako da u prompt-u upi\v{s}ete "python"} program kojeg
dobijete kad instalirate python dobiti cete nesto kao:

\sourcee{
Python 2.2 (\#1, Jan  9 2002, 02:47:26) \\
$[$GCC 2.95.2 19991024 (release)$]$ on darwin \\
Type "help", "copyright", "credits" or "license" for more information.  \\
$>>>$ 
}

Ovdje sada  mo\v{z}ete upisivati naredbe programskog jezika python i
vidjeti rezultat:

\sourcee{
Python 2.2 (\#1, Jan  9 2002, 02:47:26) \\
$[$GCC 2.95.2 19991024 (release)$]$ on darwin \\
Type "help", "copyright", "credits" or "license" for more information.  \\
$>>>$ \var{print 2+3} \\
5 \\
$>>>$ \var{d=2} \\
$>>>$ \var{print d+4} \\
6 \\
$>>>$ 
}

Crvenom bojom su ozna\v{c}ene naredbe koje sam u gornjem slu\v{c}aju
pisao ja, a ostatak je generiralo ra\v{c}unalo (odnosno python
interpreter).

Ovaj na\v{c}in pokretanja python interpretera se zove interaktivni. U
principu sve programe u ovom poglavlju mo\v{z}ete pokretati u interaktivnom
na\v{c}inu jednostavno tako da upisujete liniju po liniju programa.

\section{Komentari u programu}

	Gornji program je sasvim mogao izgledati i ovako:

	\sourcee{
		\com{\# "Hello world" (c) 2001 by "Trlababalan" d.d.}\\
		\wrd{print} "Hello world"
	}

	\textbf{Komentar:}
	Prva linija, ona koja zapo\v{c}inje sa znakom \# je \emph{komentar}.
	Kad interpreter do\dj{}e do ovog znaka on jednostavno presko\v{c}i sve
	\v{s}to na\dj{}e nakon tog znaka pa sve do kraja reda.

	Nije lo\v{s}e stavljati komentare u va\v{s}e programe. Oni koji \'{c}e slijediti
	\'{c}e u sebi imati puno komentara. 

	Mogli se napisati i ovako:

	\sourcee{
		\com{\# Do\v{s}ao \v{s}krtac u restoran pojesti juhu, a u jednom trenutku je\\
		\# morao oti\'{c}i na WC. Naravno, uhvatila ga panika da mu netko\\
		\# ne bi pojeo juhu za to vrijeme, pa je ispod juhe stavio papiri\'{c}\\
		\# na kojeg je napisao "Pljunuo sam u juhu" -- nadaju\'{c}i se da \\
		\# tako nikome ne\'{c}e pasti na pamet niti primirisati njegovoj tanjuru\\
		\# \\
		\# Kad se vratio, otkrio je da je na dnu njegovog papiri\'{c}a netko\\
		\# nadopisao "\dots i ja"}\\
		\\
		\wrd{print} "Hello world"
	}

	Mo\v{z}e i ovo:

	\sourcee{
		\wrd{print} "Hello world" 
		\com{\# ovaj program ispisuje na monitor: "Hello world"}
	}

	U stvari gornja \v{c}etiri slu\v{c}aja predstavljaju jedan te isti program.

	Postoji jo\v{s} jedan na\v{c}in kako se mogu pisati komentari u programu:

	\sourcee{
		\wrd{print} "Hello world"\\
		\com{""" Ovo je samo \\
		komentar """}
	}

	\vspace{3mm}
	Komentar po\v{c}inje s nizom od tri navodnika i zavr\v{s}ava s tri navodnika, a
	mo\v{z}e se protezati preko vi\v{s}e linija.

\section{Varijable}	

	Promotrite sljede\'{c}i primjer:



\sourcee{
\var{x} = 5 \\
\textcolor{green}{\# od sada (do daljnjega) varijabla x ima vrijednost 5}\\
\textcolor{blue}{print} "x je", \var{x} \\
\textcolor{green}{\# ispisuje prvo poruku "x je", a nakon nje vrijednost varijable x}\\
\var{y} = 10\\
\textcolor{blue}{print} "...a y je", \var{y}
}

	Nakon \v{s}to ga snimite i pokrenete ispis \'{c}e biti:

\sourcee{
x je sada 5\\
y ima vrijednost 10
}

	\textbf{Obja\v{s}njenje:} u prevom redu programa, gdje pi\v{s}e \verb|x = 5| smo
	definirali varijablu \verb|x|. \verb|x| od sada ima vrijednost 5. To ne zna\v{c}i
	da se ta vrijednost ne\'{c}e promijeniti u budu�nosti. 
	Od sad, svaki put kad poku\v samo ispisati vrijednost x-a dobiti \' cemo broj 5.

	\textbf{Napomena:} znak "=" u prvom redu programa ne treba shvatiti kao
	matemati\v{c}ku tvrdnju da je \verb+x+ isto \v{s}to i 5, odnosno da \verb+x+ ima
	vrijednost 5. Redak "\verb+x = 5+" treba shvatiti kao "Od sada \'{c}e \verb+x+
	imati rijednost 5" ili "Neka \verb+x+ ima vrijednost 5". Dakle varijabli
	\verb+x+ \underline{pridru\v{z}ujemo vrijednost} 5.

	U trenutku kad smo u programu napisali \verb|x = 5| ka\v{z}emo da smo varijablu
	\emph{inicijalizirali}, dakle odredili smo joj neku po\v cetnu vrijednost. Ako
	poku\v samo ispisati neku neinicijaliziranu varijablu interpreter \' ce nam ispisati
	poruku o gre\v sci:

	\sourcee{
	\textcolor{blue}{print} \var{x} \textcolor{green}{\# ...ali varijabla x nije prethodno inicijalizirana!}
	}

	Rezultat je:

	\sourcee{
Traceback (most recent call last):
\\
File "primjer.py", line 1, in ?
\\
print x
\\
NameError: name 'x' is not defined
	}

	Obratite pa\v znju na to da vam se to\v cno ka\v ze gdje je gre\v ska u programu
	("line 1"). To nam sad i nije od neke koristi jer na\v s program ima samo jednu
	liniju, ali kad se stvari zakompliciraju onda \' cemo takve informacije jako puno
	koristiti!
	
	Pogledajmo sljede\'{c}i primjer:

	\sourcee{
	\var{x} = 5 \textcolor{green}{\# x ima sada vrijednost 5}\\
	\textcolor{blue}{print} "x je", \var{x} \textcolor{green}{\# ispisujemo vrijednost od x}\\
	\var{y} = \var{x} \textcolor{green}{\# sada y poprima vrijednost varijable x}\\
	\textcolor{blue}{print} "y je", \var{y} \textcolor{green}{\# ispisujemo vrijednost od y}\\
	\var{x} = 10 \textcolor{green}{\# x je sad 10}\\
	\textcolor{blue}{print} "x je sad ", \var{x} \textcolor{green}{\# ispisujemo x}\\
	\textcolor{blue}{print} "y je sad", \var{y} \textcolor{green}{\# ispisujemo y}
	}

	Kad pokrenemo program ispis je:

	\sourcee{
	x je 5\\
	y je 5\\
	x je sad 10\\
	y je sad 5
	}

	Prvo je \verb|x| bio 5, tada smo varijabli \verb|y| dodijelili vrijednost varijable
	\verb|x|, dakle 5. Tada smo \verb|x| promijenili da sadr\v zi 10, ali \verb|y| je
	ostao 5 otprije.

\section{Naredbe}

	Program se sastoji od niza naredbi. Svaka naredba ra\v{c}unalu govori \v{s}ta
	treba \v{c}initi. Razmotriti \'{c}emo jo\v{s} jednom jedan primjer programa
	pisanog u Pythonu:
	
	\sourcee{
		\var{x} = 5;\\
		\wrd{if} \var{x} $<$ 10:\\
		\hspace*{10mm}\textcolor{blue}{print} "x je manje od 10"\\
		\wrd{else}:\\
		\hspace*{10mm}\textcolor{blue}{print} "x je ve\'{c}e od 10"
	}

	Python je programski jezik koji zahtijeva da se svaka naredba pi\v{s}e u odvojenom
	retku. Tako, svaki red predstavlja jednu naredbu, u prvom redu naredba "=" ka\v{z}e
	ra\v{c}unalu da varijabli \verb|x| pridru\v{z}i vrijednost 5. Nakon toga jedan
	\verb+if+--\emph{uvijet}\footnote{Vi\v{s}e rije\v{c}i o uvjetima kasnije\dots} 
	(naredba \verb+if+)
	provjerava je
	li x manji od 10, ako jest onda ispisuje s \verb+print+ poruku "x je manji od 10", a ako nije onda
	poruku "x je ve\'{c}e od 10". Program \'{c}e, naravno, ispisati prvu poruku

	Naredbu \verb+print+ smo upoznali kod pisanja "Hello world" programa -- ona ispisuje na monitor ono \v{s}to se nalazi iza nje.
	O toj naredbi \'{c}emo jo\v{s} puno pri\v{c}ati kasnije.

	U nekim programskim jezicima mogu se pisati
	programi koji u jednoj liniji imaju vi\v{s}e naredbi. Pogledajte na primjer kra\'{c}i
	od dva programa pisana u programskom jeziku Perl u odjeljku 1.3.1. -- ono za
	\v{s}to nam u Pythonu treba 5 redova mo\v{z}emo u Perlu napisati dva reda (mogo bi
	se i u jednom).
	Mo\v{z}da vam se to mo\v{z}e \v{c}initi prakti\v{c}nim, ali takvi programski jezici
	rezultiraju s programima koji su vrlo te\v{s}ko shvatljivi. U takvom programu je
	vrlo te\v{s}ko na\'{c}i gre\v{s}ku.

	Pogledajmo jo\v{s} jednom taj program u programskom jeziku Perl:

	\source{Perl:}{
		x = 5;\\
		print x $==$ 10 ? "x je manji od 10$\setminus$n" : "x je ve\'{c}i od 10$\setminus$n";
		}

	Ako znate imalo engleskog onda bi vam prvi primjer pisan u Pythonu trebao biti
	dosta jasan, \v{c}ak i ako ne znate programirati. Pogledajte sad ovaj drugi
	primjer, on je toliko zgusnut i kripti\v{c}an da je programeru-po\v{c}etniku vrlo
	te\v{s}ko shvatiti \v{s}ta je ovdje \v{s}ta. Zamislite da sad imate program koji
	ima 1000 ovakvih linija, a znate da se negdje me\dj{}u njima nalazi jedna
	gre\v{s}ka?!

	\textbf{Zapamtite:} Uvijek se potrudite pisati pregledne programe. Svaku komandu
	pi\v{s}ite u novu liniju. Svaki slo\v{z}eniji dio programa popratite s komentarima!
	Desiti \'{c}e vam se da morate prepraviti program koji ste nekad davno pisali. To
	\'{c}e biti tim lak\v{s}e \v{s}to ste tada taj program preglednije pisali!
	
\section{Imena}

	U zadnja dva primjera u na\v{s}im programima koristili smo se varijablama. Te
	varijable smo nazvali \verb+x+ i \verb+y+. Naravno, mogli smo ih nazvati i
	druk\v{c}ije. Probati \'{c}emo ne\v{s}to nau\v{c}iti iz sljede\'{c}eg primjera:

	\sourcee{
\var{pi} = 3.1415926 \\
\var{v} = 5 \\
\var{V} = 10 \\
\var{nova}\_varijabla = 12345 \\
\var{var1} = 12 \\
\var{var2} = 13 \\
\textcolor{blue}{print} "pi je",\var{pi} \\
\textcolor{blue}{print} "varijabla v ima vrijesnost",\var{v} \\
\textcolor{blue}{print} "varijabla V (veliko slovo!) ima vrijesnost",\var{V} \\
\textcolor{blue}{print} "jos jedna varijabla:", \var{nova\_varijabla} \\
\textcolor{blue}{print} "var1 je", \var{var1}, ", a var2 je", \var{var2} 
	}

	Prvih 6 linija programa varijablama pridru\v{z}uje neke vrijednosti. Imena tih
	varijabli su 
	\verb+pi+,
	\verb+v+,
	\verb+V+,
	\verb+nova_varijabla+,
	\verb+var1+ i 
	\verb+var2+.

	\textbf{Va\v{z}no:} Uo\v{c}ite da Python razlikuje velika i mala
	slova: varijabli \verb+v+ smo
	pridru\v{z}ili vrijednost 5, a varijabli \verb+V+ (veliko slovo V) vrijednost 10.
	Kasnije u programu se njih tretira kao dvije razli\v{c}ite varijable!

	Kad daj program upi\v{s}ete i pokrenete dobiti \'{c}ete:

	\sourcee{
	pi je 3.1415926\\
	varijabla v ima vrijesnost 5\\
	varijabla V (veliko slovo!) ima vrijesnost 5\\
	jos jedna varijabla: 12345\\
	var1 je 12 , a var2 je 13
	}

	\textbf{Zapamtite:} Imena varijabli se mogu sastojati kombinacije velikih i malih
	slova, decimalnih znamenaka
	i znaka "\_"\footnote{Engleski: underscore}. Decimalna znamenka se ne smije
	nalaziti na prvom mjestu imena varijable. Razlikuju se velika i mala slova; dakle
	\verb+kikiriki = 5+ i \verb+Kikiriki = 10+ su dvije razli\v{c}ite varijable s
	razli\v{c}itim vrijednostima! 

	Ima jo\v{s} jedno pravilo kojeg se morate dr\v{z}ati kod odre\dj{}ivanja imena
	varijable; ime varijable \emph{ne smije biti klju\v{c}na rije\v{c}!}

\section{Klju\v{c}ne rije\v{c}i}

	Klju\v{c}ne rije\v{c}i su "rije\v{c}i" ili nizovi znakova koje su rezervirane za
	kori\v{s}tenje u programskom jeziku i ne smiju se koristiti za imenovanje varijabli
	ili nekih drugih dijelova programa koje odre\dj{}uje programer. Ve\'{c} smo se
	sreli s  naredbom \verb+print+. \verb+print+ je klju\v{c}na rije\v{c} i ne smije
	biti kori\v{s}tena kao ime varijable. Mo\v{z}emo na\v{s}u varijablu imenovati npr
	\verb+var_print+ ili \verb+_print+ ili \verb+print1+, ali nikako \verb+print+!

	Svaki programski jezik ima odre\dj{}eni broj klju\v{c}nih rije\v{c}i. Klju\v{c}ne
	rije\v{c}i u programskom jeziku Python su:

\begin{center}
\begin{tabular}{llll}
and & del & for & is \\
raise & assert & elif & from\\
lambda & return & break & else\\
global & not & try & class\\
except & if & or & while\\
continue & exec & import & pass\\
def & finally & in & print
\end{tabular}
\end{center}

	Radi prakti\v{c}nosti, u programima koji se nalaze u ovoj knji\v{z}ici klju\v{c}ne
	rije\v{c}i su prikazane u plavoj boji.

\section{Komuniciranje s okolinom}

	Ra\v cunalo nema previ\v se smisla ako ne komunicira s okolinom. Pod
	\emph{komunikacija s okolinom} podradzumijeva se komunikacija s korisnikom ili
	komunikacija s nekim drugim ure\dj{}ajem vezanim uz ra\v cunalo. Ra\v cunalo
	komunicira s korisnikom na razli\v cite na\v cine; npr. klikanjem po ikonama,
	upisivanjem teksta u nekakav formular, diktiranjem u mikrofon spojen 
	na ra\v cunalo, i tako dalje i tako bli\v ze.

	Nau\v citi \' cemo sada jedan jednostavan na\v cin kako mo\v zemo na\v s program u
	pythonu natjerati da i komunicira s nama.

	\sourcee{
		\var{name} = \textcolor{blue}{raw\_input}( "Upi\v site svoje ime:" )\\
		\textcolor{blue}{print} "Dobar dan", \var{name}
	}

	Kad pokrenete program ra\v cunalo \' ce ispisati poruku "Upi\v site svoje ime:" i
	tra\v ziti od vas da upisujete ime. Rezultat mo\v ze biti:

	\sourcee{
		Upi\v site svoje ime:\textcolor{blue}{Aleksandar Makedonski}\\
		Dobar dan  Aleksandar Makedonski
	}

	Plavom bojom je ozna\v cen tekst kojeg upisuje sam korisnik.

	Pomo\' cu naredbe \verb+raw_input+ sad mo\v zemo bilo kojoj varijabli 
	pridru\v ziti vrijednost broja kojeg \'{c}e korisnik utipkati tek u trenutku kad 
	se program pokrene. \verb"raw_input" je sli\v{c}an, ali s njime mo\v{z}emo toj
	varijabli pridru\v{z}iti i vrijednost stringa (a ne isklju\v{c}ivo brojevnu
	vrijednost).

	Porobati \' cemo sada napisati program koji tra\v zi od korisnika da upi\v se 
	tri broja, a nakon toga ispisuje njihovu aritmeti\v cku sredinu (prosijek).

	\sourcee{
		\var{n1} = \textcolor{blue}{input}( "Upi\v si prvi broj:" )\\
		\var{n2} = \textcolor{blue}{input}( "Upi\v si drugi broj:" )\\
		\var{n3} = \textcolor{blue}{input}( "Upi\v si tre\' ci broj:" )\\
		\var{avg} = ( \var{n1} + \var{n2} + \var{n3} ) / 3\\
		\textcolor{blue}{print} "Aritmeti\v cka sredina je ", \var{avg}
	}

	\textbf{Komentar:} Program prvo u varijable \verb+n1+, \verb+n2+ i \verb+n2+
	smje\v sta ono \v sto \' ce korisnik sam upisati kad ga se upita da upi\v se broj.
	Varijabla \verb"avg" (eng. "average" = "prosjek") zatim prime vrijednost
	aritmeti\v cke sredine brojeva 
	\verb"n1",
	\verb"n2" i 
	\verb"n3".
	Na kraju se samo ispisuje vrijednost od \verb"avg".

	Evo jo\v s dvije varijante istog programa:

	\sourcee{
		\var{n1} = \textcolor{blue}{input}( "Upi\v si prvi broj:" )\\
		\var{n2} = \textcolor{blue}{input}( "Upi\v si drugi broj:" )\\
		\var{n3} = \textcolor{blue}{input}( "Upi\v si tre\' ci broj:" )\\
		\var{avg} = ( float( \var{n1} ) + float( \var{n2} ) + float( \var{n3} ) ) / 3\\
		\textcolor{blue}{print} "Aritmeti\v cka sredina je ", \var{avg}
	}

	Ili jo\v s kra\' ce:

	\sourcee{
		\var{n1} = float( \textcolor{blue}{input}( "Upi\v si prvi broj:" ) )\\
		\var{n2} = float( \textcolor{blue}{input}( "Upi\v si drugi broj:" ) )\\
		\var{n3} = float( \textcolor{blue}{input}( "Upi\v si tre\' ci broj:" ) )\\
		\textcolor{blue}{print} "Aritmeti\v cka sredina je ", ( \var{n1} + \var{n2} + \var{n3} ) / 3
	}

	\textbf{Napomena:} \verb+float()+ nije jedini na\v cin kako se string mo\v ze
	pretvoriti u broj. U sljede\' cem poglavlju \' ce biti obja\v snjeno za\v sto se
	ovdje koristi ba\v s taj.

\chapter{Izrazi}

\section{Aritmeti\v{c}ki izrazi}

Aritmeti\v{c}ki izrazi su matemati\v{c}ki izrazi s kakvima se ra\v{c}una u
osnovnoj \v{s}koli. Naj\v{c}e\v{s}\'{c}e se sastoje od brojeva ili varijabli
koje imaju broj\v{c}anu vrijednost i matemati\v{c}kih operacija. Primjer
aritmeti\v{c}kog izraza mo\v{z}e biti: 
$2+4-2$, 
$\displaystyle \frac{3+6\cdot 5}{7-4}$,
ili 
$\displaystyle \frac{s_2-s_1}{t_2-t_1}$. 

Kad nam negdje u programu zatreba
aritmeti\v{c}ki izraz zapisujemo ga na sli\v{c}an na\v{c}in kako bismo ga napisali u
bilje\v{z}nici s nekoliko sitnih razlika;

\begin{itemize}
	\item Mno\v{z}enje zapisujemo pomo\'{c}u znaka *, a ne $\cdot$.
	\item Dijeljenje zapisujemo pomo\'{c}u znaka / umjesto :
	\item Razlomke zapisujemo pomo\'{c}u operacije dijeljenja.
	\item Potencije zapisujemo pomo\'{c}u "**". Dakle $3^2$ bi zapisalo kao
		3**2
	\item Ostatak\footnote{"Modulo"} pri dijeljenju dobijemo pomo\'{c}u
		operacije \%.
\end{itemize}

Izraz 
$\displaystyle \frac{3+6\cdot 5}{7-4}$
bi napisali: \verb+(3+6*5)/(7-4)+. U ovom
slu\v{c}aju brojnik i nazivnik treba staviti unutar zagrada jer bi u
slu\v{c}aju da je izraz \verb+3+6*5/7-4+ kompjuter poku\v{s}ao prvo
izra\v{c}unati \verb+6*5/7+, python naime izraze ra\v{c}una paze\'{c}i na
prednost ra\v{c}unskih operacija (npr. mno\v{z}enje i dijeljenje imaju
prednost pred zbrajanjem i oduzimanjem).

\textbf{Va\v{z}no:}
Ima jo\v{s} jedna stvar na koju treba pripaziti pri pisanju algebarskih
izraza; ukoliko su brojevi s kojima ra\v{c}unamo cjelobrojni onda \'{c}e (u
programskom jeziku python) i rezultat biti cjelobrojan. Dakle, ako probate
izra\v{c}unati \verb+13/4+ dobiti \'{c}ete \verb+3+, a ne \verb+3.25+! To se
mo\v{z}e rije\v{s}iti tako da barem jedan od brojeva definiramo kao realan, a za
to je dovoljno dodati mu decimalnu to\v{c}ku na kraju. Da bi dobili to\v{c}an
rezultat dijeljenje 13 podijeljeno s 4 trebali bi dakle napisati
\verb+13/4+.

Zadaci:

\section{Logi\v{c}ki izrazi}

Sli\v{c}no kao aritmeti\v{c}ki izrazi logi\v{c}ki izrazi se sastoje od
operacija i \v{c}lanova izraza nad kojima ze izvr\v{s}avaju te operacije.
Kod aritmeti\v{c}kih izraza \v{c}lanovi su brojevi ili varijable s
brojevnom vrijedno\v{s}\'{c}u, a \v{c}lanovi logi\v{c}kih izraza mogu biti
\emph{sudovi} ili \v{c}ak drugi aritmeti\v{c}ki izrazi.

\emph{Sud} je tvrdnja koja mo\v{z}e biti istinita ili la\v{z}na. Primjer
suda je \emph{"Zemlja kru\v{z}i oko Mjeseca"} ili \emph{"Postoji
beskona\v{c}no mnogo prirodnih brojeva"}.  Svaki sud mora imati svoju
istinosnu vrijednost koja mo\v{z}e biti \emph{"istina"} ili \emph{"la\v{z}"}.
Ukoliko za neku tvrdnju ne mo\v{z}emo sa sigurno\v{s}\'{c}u kazati je li
istinita ili la\v{z}na tada to nije sud. Na primjer \emph{"Zemlja
kru\v{z}i oko Mjeseca"} jest sud zato \v{s}to ima istinosnu vrijednost
\emph{"la\v{z}"}, kao i \emph{"Postoji beskona\v{c}no mnogo prirodnih
brojeva"} \v{c}ija je istinita vrijednost \emph{"istina"}.  Tvrdnja
\emph{"Frank Sinatra pjeva bolje od Tine Turner"} nije sud jer je
nemogu\'{c}e odrediti istinitost te tvrdnje budu\'{c}i da je ona \v{c}isto
subjektivne prirode (nekome se vi\v{s}e svidja Sinatra, a nekome Tina
Turner). Isto tako nije sud "U\v{c}ini to!" ili "Mo\v{z}da \'{c}u jednog
dana nau\v{c}iti programirati".

Umjesto "istina" ili "la\v{z}" se \v{c}esto koriste velika slova "T" (od
engleskog "true" = "istina") odnosno "F" (eng. "false" = "la\v{z}").

Sud mo\v{z}emo zapisati i matemati\v{c}kim simbolima: $1<2$ je sud u
kojemu se tvrti da je jedan manji od 2, a istinosna vrijednost tog
suda je T (= istina). Matemati\v{c}ki sudovi kojima se koristimo u
programiranju naj\v{c}e\v{s}\'{c}e se koriste za opisivanje odnosa izmedju
brojeva. Pri tome se koristimo sljede\'{c}im simbolima iz sljede\'{c}e tablice;
u prvom stupcu se nalazi simbol kako bismo za zapisali u bilje\v{z}nicu
ili na plo\v{c}u, u drugom stupcu zapisa tog istog simbola u programu,
a u tre\'{c}em kako \v{c}itamo taj simbol:

\begin{tabular}{lll}
	$=$ & $==$ & je jednako \\
	$\neq$ & $!=$ & nije jednako, je razli\v{c}ito \\
	$<$ & $<$ & je manje od \\
	$\leq$ & $<=$ & je manje ili jednako \\
	$>$ & $>$ & je ve\'{c}e od \\
	$\geq$ & $>=$ & je ve\'{c}e ili jednako
\end{tabular}

Probajmo sada utvrditi istinosnu vrijednost nekih matemati\v{c}kih izraza:

\begin{tabular}{lll}
	$12<12.01$ & \verb+12<12.01+ & T\\
	$1+2\leq 5$ & \verb"1+2<=5" & T\\
	$10-3\geq 6+1$ & \verb"10-3>=6+1" & T \\
	$10-2\geq 6+1$ & \verb"10-2>=6+1" & F \\
	$5\neq 5$ & \verb"5!=5" & F
\end{tabular}

Gornji primjeri su primjeri \emph{jednostavnih sudova}. Slo\v{z}eni
sudovi su sudovi koji se dobijaju od jednostavnih sudova i logi\v{c}kih
operacija \emph{and}, \emph{or} i \emph{not}.

\subsection{Logi\v{c}ka operacija \emph{and}}

Promotrimo re\v{c}enicu \emph{"Ako je lijepo vrijeme
idemo na izlet."}. O \v{c}emu ovisi o\'{c}ete li
oti\'{c}i na izlet? Ovisi o tome je li lijepo vrijeme, dakle ovisi o
istinosnoj vrijednosti suda \emph{"Lijepo je vrijeme."}. Ako je
taj sud istinit (T) oti\'{c}i \'{c}ete na izlet, a ako nije (F) -- ni\v{s}ta
od izleta.

Malo \'{c}emo stvar zakomplicirati ako osim lijepog vremena va\v{s} izleta
ovisi o jo\v{s} ne\v{c}emu; \emph{"Ako je lijepo vrijeme i nemam drugih obaveza
i\'{c}i \'{c}u na izlet"}.
Sad va\v{s} izlet ovisi o istinitosti suda \emph{"Lijepo je vrijeme i nemam obaveza"}, a on je istinit
kad su istovremeno istinita sljede\'{c}a dva suda:

\begin{itemize}
	\item[\emph{(a)}] \emph{"Lijepo je vrijeme."}
	\item[\emph{(b)}] \emph{"Nemam obaveza."}
\end{itemize}

Dakle, treba vrijediti da je istinito i \emph{(a)} i \emph{(b)},
jer ako je bilo koje od ta dva la\v{z} onda je i tvrdnja \emph{"Lijepo
je vrijeme i nemam obaveza"} la\v{z}na.

Ukoliko imamo dva suda koja \'{c}emo ovdje ozna\v{c}iti s $A$, odnosno
$B$ onda \'{c}emo takvu kombinaciju zapisivati s $A and B$. Suda $A
and B$ je \emph{slo\v{z}eni sud} koji se sastoji od jednostavnijih
sudova $A$ i $B$. Istinitost suda $A and B$ ovisi o istinitosti
sudova $A$ i $B$; tek ako su oba istinita onda je i $A and B$
istinit. To se mo\v{z}e prikazati pomo\'{c}u sljede\'{c}e tablice:

\begin{tabular}{ll|l}
	$A$ & $B$ & $A and B$ \\
	\hline
	istina & istina & istina \\
	istina & la\v{z} & la\v{z} \\
	la\v{z} & istina & la\v{z} \\
	la\v{z} & la\v{z} & la\v{z} 
\end{tabular}

Garnju tablicu zovemo \emph{tablica istinitosti} logi\v{c}ke operacije $and$.

\subsection{Logi\v{c}ka operacija \emph{or}}

Pretpostavimo da \'{c}ete oti\'{c}i na izlet ako vrijedi \emph{"Lijepo
je vrijeme ili imam dobro dru\v{s}tvo"}. Dakle, ako je lijepo vrijeme
idete na izlet, ako nije lijepo vrijeme i imate dobro dru\v{s}tvo ipak
idete na izlet, ako nemate dobro dru\v{s}tvo i lijepo je vrijeme opet
idete na izlet, a jedini slu\v{c}aj kad ne idete na izlet je kad, niti
je vrijeme lijepo niti imate dobro dru\v{s}tvo.

Tablica istinitosti logi\v{c}ke operacije $or$ izgleda ovako:

\begin{tabular}{ll|l}
	$A$ & $B$ & $A or B$ \\
	\hline
	istina & istina & istina \\
	istina & la\v{z} & istina \\
	la\v{z} & istina & istina \\
	la\v{z} & la\v{z} & la\v{z} 
\end{tabular}

\subsection{Logi\v{c}ka operacija \emph{not}}

U zadnjem slu\v{c}aju oti\'{c}i \'{c}ete na izlet tek ako vrijedi \emph{"Nemam
drugih obaveza"}. Dakle, tek ako \emph{nije} istit sud \emph{"Imam
drugih obaveza"}. Logi\v{c}ka operacija $not$ nekom sudu pridodaje
suprotanu istinosnu vrijednost.

\begin{tabular}{l|l}
	$A$ & $not A$ \\
	\hline
	istina & la\v{z}\\
	la\v{z} & istina
\end{tabular}

Zadaci: Probajte odrediti istinitost sljede\'{c}ih sudova:
	- 1<2 and 13!=5
	- 2>5 or 1=2
	- 1<2 and 5!=5
	- ( 1<2 or 9=5 ) and 3==3
	- ( 2==2 and 3!=5 ) or ( 3==4 )

\subsection{Nekoliko dodatnih pravila}

U programskom jeziku python (sli\v{c}no kao u mnogim drugim) postoji
jo\v{s} nekoliko dodadnih pravila kod utvrdjivanjaistinitosti sudova:
- svaki broj/izraz je po definiciji sud:
	- ako njegova vrijednost 0 onda je njegova istinosna vrijednost "la\v{z}" (F)
	- ako je razli\v{c}it od 0 onda je njegova istinosna vrijednost "istina" (T)
- svaki string je po definiciji sud:
	- ako je string prazan onda je njegova istinosna vrijednost "la\v{z}" (F)
	- ako string nije prazan onda je njegova istinosna vrijednost "istina" (T)

dakl ima smisla sud \verb"2<3 or 0". Budu\'{c}i da je $2<3$ ondaje prvi dio suda istinit, a (broj) 0 je
po gornjim pravilima la\v{z}an, dakle imamo slu\v{c}aj "istina" $or$ "la\v{z}", dakle rezultat je istina.

Zadaci: Probajte odrediti istinitost sljede\'{c}ih sudova:
\begin{itemize}
\item \verb"1 or 3>3"
\item \verb"2!=3 and 1"
\item \verb+"" or "jkljkl"+
\item \verb+( 0 or "jkljkl" ) and 2<3+
\item \verb+( not "jkljkl" ) and ( not 12 )+
\end{itemize}
\chapter{Kontrola toka programa}

\section{if \dots then \dots else \dots}

Pretpostavimo da imamo program koji od korisnika tra\v{z}i da upi\v{s}e
koliko je bodova zaradio na odre\dj{}enom testu i zatim ovisno o broju
bodova ispisuje koju je ocjenu dobio. Ako je dobio manje ili jednako
39 bodova ocjena je 1, ako ima 40-54 ocjena je 2, za 55-69 ocjena
je 3, za 70-85 ocjena je 4, a za vi\v{s}e od 85 ocjenjen je s 5.

Probati \'{c}emo za po\v{c}etak napisati program koji samo ispisuje je
li ocjena 1 ili ve\'{c}a. Za to program treba nekako imati na\v{c}in kako
\'{c}e provjeriti je li broj bodova ve\'{c}i, manji ili jednak 40.

Program izgleda ovako:

\sourcee{
\var{bodovi} = \wrd{input}( "Upi\v{s}i broj bodova:" )
\\
\wrd{print} "Imate", bodovi, "bodova..."
\\
\wrd{if} \var{bodovi} $<$ 40:
\\
\hspace*{10mm}\wrd{print} "Na\v{z}alost dobili ste negativnu ocjenu :("
\\
\wrd{print} "Kraj programa"
}

Kori\v{s}tena je naredba if, ona se koristi na sljede\'{c}i na\v{c}in:

\sourcee{
\wrd{if} \emph{logi\v{c}ki izraz}:
\\
\hspace*{10mm}\emph{komande programa u slu\v{c}aju da je logi\v{c}ki izraz istinit}
}

Kao prvo uo\v{c}ite da je dio programa koji seizvr\v{s}ava u slu\v{c}aju da
je logi\v{c}ki izraz istinit \emph{uvu\v{c}en} u odnosu na ostatak
programa. To uvla\v{c}enje je to\v{c}no definirano i mora biti jedan
$<$tab$>$ (ili 8 razmaknica) ra\v{c}unaju\'{c}i od lijevog ruba polja u kojem se
editira program!

U po\v{c}etnom programu se komanda nakon \verb"if" naredbe izvr\v{s}ava
jedino u slu\v{c}aju ako je broj bodova manji od 40, ako je broj bodova
ve\'{c}i ili jednak 40 program tu liniju jednostavno preska\v{c}e. Mo\v{z}emo
srediti i da program preska\v{c}e ve\'{c}i broj linija:

\sourcee{
\var{bodovi} = \wrd{input}( "Upi\v{s}i broj bodova:" )
\\
\wrd{print} "Imate", bodovi, "bodova..."
\\
\wrd{if} \var{bodovi} $<$ 40:
\\
\hspace*{10mm}
	\wrd{print} "Na\v{z}alost dobili ste negativnu ocjenu :("
\\
\hspace*{10mm}
	\wrd{print} "Molimo vas lijepo da za sljede\'{c}i put malo bolje nau\v{c}ite"
\\
\wrd{print} "Kraj programa"
}

Poruka koja se sad ispisuje ukoliko nemate dovoljno bodova je 

\sourcee{
Na\v{z}alost dobili ste negativnu ocjenu :(
\\
Molimo vas lijepo da za sljede\'{c}i put malo bolje nau\v{c}ite
}

Opet, \verb"Kraj programa" se ispisuje bez obzira na broj bodova.

\v{Z}elimo li da na\v{s} program ispisuje i poruku ukoliko imate vi\v{s}e ili jednako od 40 bodova to se mo\v{z}e
tako da koristite dva puta naredbu \verb"if":

\sourcee{
\var{bodovi} = \wrd{input}( "Upi\v{s}i broj bodova:" )
\\
\wrd{print} "Imate", bodovi, "bodova..."
\\
\wrd{if} \var{bodovi} $<$ 40:
\\
\hspace*{10mm}
	\wrd{print} "Na\v{z}alost dobili ste negativnu ocjenu :("
\\
\wrd{if} \var{bodovi} $>$= 40:
\\
\hspace*{10mm}
	\wrd{print} "Imate vi\v{s}e od 40 bodova"
\\
\wrd{print} "Kraj programa"
}

Ili pomo\'{c}u jednog dodatka naredbi \verb"if":

\sourcee{
\var{bodovi} = \wrd{input}( "Upi\v{s}i broj bodova:" )
\\
\wrd{print} "Imate", bodovi, "bodova..."
\\
\wrd{if} \var{bodovi} $<$ 40:
\\
\hspace*{10mm}
	\wrd{print} "Na\v{z}alost dobili ste negativnu ocjenu :("
\\
\wrd{else}:
\\
\hspace*{10mm}
	\wrd{print} "Imate vi\v{s}e od 40 bodova"
\\
\wrd{print} "Kraj programa"
}

Program govori ra\v cunalu:
\emph{Ako je broj bodova manji od 40 tada ste dobili negativnu
ocjenu, ina\v{c}e imate vi\v{s}e od 40 bodova}

Sve ono \v{s}to se nalazi nakon \verb"else", naravno uvu\v{c}eno za jedan
$<$tab$>$ \'{c}e biti ispisano u slu\v{c}aju da uvjet \verb"bodovi$<$40" nije
istinit!

\v{S}ta ako ne \v{z}elimo samo podatak o tome jesmo li dobili vi\v{s}e ili
manje od 40 bodova nego i koju smo ocjenu dobili:

Prvi na\v{c}in koriste\'{c}i niz \verb"if"-ova:

\sourcee{
\var{bodovi} = \wrd{input}( "Upi\v{s}i broj bodova:" )
\\
\wrd{print} "Imate", bodovi, "bodova..."
\\
\wrd{if} \var{bodovi} $<$ 40:
\\
\hspace*{10mm}
	\wrd{print} "Na\v{z}alost dobili ste negativnu ocjenu :("
\\
\wrd{if} 40$<$=\var{bodovi} and \var{bodovi}$<$=54:
\\
\hspace*{10mm}
	\wrd{print} "Dovoljan (2)"
\\
\wrd{if} 55$<$=\var{bodovi} and \var{bodovi}$<$=69:
\\
\hspace*{10mm}
	\wrd{print} "Dobar (3)"
\\
\wrd{if} 70$<$=\var{bodovi} and \var{bodovi}$<$=84:
\\
\hspace*{10mm}
	\wrd{print} "Vrlo dobar (4)"
\\
\wrd{if} 85$<$=\var{bodovi}:
\\
\hspace*{10mm}
	\wrd{print} "Odli\v{c}an (5)"
}

Drugi na\v{c}in koriste\'{c}i \verb"elif" (od eng. "else if"):

\sourcee{
\var{bodovi} = \var{input}( "Upi\v{s}i broj bodova:" )
\\
\wrd{print} "Imate", \var{bodovi}, "bodova..."
\\
\wrd{if} \var{bodovi} $<$ 40:
\\
\hspace*{10mm}
	\wrd{print} "Na\v{z}alost dobili ste negativnu ocjenu :("
\\
\wrd{elif} 40$<$=\var{bodovi} and \var{bodovi}$<$=54:
\\
\hspace*{10mm}
	\wrd{print} "Dovoljan (2)"
\\
\wrd{elif} 55$<$=\var{bodovi} and \var{bodovi}$<$=69:
\\
\hspace*{10mm}
	\wrd{print} "Dobar (3)"
\\
\wrd{elif} 70$<$=\var{bodovi} and \var{bodovi}$<$=84:
\\
\hspace*{10mm}
	\wrd{print} "Vrlo dobar (4)"
\\
\wrd{elif} 85$<$=\var{bodovi}:
\\
\hspace*{10mm}
	\wrd{print} "Odli\v{c}an (5)"
}

Na hrvatskom bi ovo napisali: \emph{Ako imate manje od 40 bodova dobili ste 2, ina\v{c}e ukoliko imate
izme\dj{}u 40 i 54 dobili ste 2, ina\v{c}e ukoliko imate izme\dj{}u 55 i 59 dobili ste 3, ina\v{c}e ukoliko imate
izme\dj{}u 70 i 84 dobili ste 4, ina\v{c}e ukoliko imate vi\v{s}e od 84 dobili ste 5}.

Mali problem mo\v{z}e nastati u tome \v{s}to mo\v{z}ete napisati i 120 ili -340 za broj bodova. Probajmo
napisati program koji ispravlja tu gre\v{s}ku:

\sourcee{
\var{bodovi} = \wrd{input}( "Upi\v{s}i broj bodova:" ) \\
\wrd{print} "Imate", \var{bodovi}, "bodova..." \\
\wrd{if} 0$<$= \var{bodovi} and \var{bodovi} $<$ 40: \\
\hspace*{10mm} \wrd{print} "Na\v{z}alost dobili ste negativnu ocjenu :(" \\
\wrd{elif} 40$<$=\var{bodovi} and \var{bodovi}$<$=54: \\
\hspace*{10mm} \wrd{print} "Dovoljan (2)" \\
\wrd{elif} 55$<$=\var{bodovi} and \var{bodovi}$<$=69: \\
\hspace*{10mm} \wrd{print} "Dobar (3)" \\
\wrd{elif} 70$<$=\var{bodovi} and \var{bodovi}$<$=84: \\
\hspace*{10mm} \wrd{print} "Vrlo dobar (4)" \\
\wrd{elif} 85$<$=\var{bodovi} and \var{bodovi} $<$=100: \\
\hspace*{10mm} \wrd{print} "Odli\v{c}an (5)" \\
\wrd{else}: \\
\hspace*{10mm} \wrd{print} "Niste upisali broj izme\dj{}u 0 i 100 za broj bodova"
}

Sad bi "prijevod" ovog programa na hrvatski glasio: \emph{Ako imate
izme\dj{}u 0 i 40 bodova dobili ste 2, ina\v{c}e ukoliko imate izme\dj{}u
40 i 54 dobili ste 2, ina\v{c}e ukoliko imate izme\dj{}u 55 i 59 dobili
ste 3, ina\v{c}e ukoliko imate izme\dj{}u 70 i 84 dobili ste 4, ina\v{c}e
ukoliko imate izme\dj{}u 84 i 100 dobili ste 5, a ako nije niti jedan
od ovih slu\v{c}ajeva onda ste pogre\v{s}no upsali broj bodova}.

Zadaci: \textbf{Za napraviti}

\section{for \dots in range( \dots )}

Problem je sljede\'{c}i; treba ispisati tablicu kvadrata brojeva od do 10.
Tablica kvadrata je tablica koja ima dva stupca, u prvom se nalazeprirodni
brojevi, a u drugom njihovi kvadrati (da podsjetimo, kvadrat prirodnog
broja dobijemo tako da taj broj pomno\v{z}imo sa samim sobom).

To se mo\v{z}e posti\'{c}i na sljede\'{c}i na\v{c}in:

\sourcee{
\wrd{print} "n -$>$ n*n" \\
\wrd{print} 1, " -$>$ ", 1**2 \\
\wrd{print} 2, " -$>$ ", 2**2 \\
\wrd{print} 3, " -$>$ ", 3**2 \\
\wrd{print} 4, " -$>$ ", 4**2 \\
\wrd{print} 5, " -$>$ ", 5**2 \\
\wrd{print} 6, " -$>$ ", 6**2 \\
\wrd{print} 7, " -$>$ ", 7**2 \\
\wrd{print} 8, " -$>$ ", 8**2 \\
\wrd{print} 9, " -$>$ ", 9**2 \\
\wrd{print} 10, " -$>$ ", 10**2
}

\verb"5**2" zna\v{c}i $5^2$. U svakom retku se naredbom print ispisuje broj i
njegov kvadrat. 

Me\dj{}utim, ovo je jedan prili\v{c}no \emph{neelegantan} na\v{c}in rje\v{s}avanja
problema, a to zato \v{s}to je ovakav program te\v{s}ko generalizirati,
odnosno te\v{s}ko ga je upotrijebiti u slu\v{c}aju da njime moramo
rije\v{s}iti neki analogni ili opcehnitiji problem. 
\footnote{
	Ako je na\v{s} trenutni
	problem \emph{"Napi\v{s}i program koji ispisuje tablicku kvadrata
	prvih 10 brojeva"} analogni (ali slo\v{z}eniji) problem mo\v{z}e glasiti
	\emph{"Napi\v{s}i program koji ispisuje tablicu kvadrata za brojeve
	od 0 do 100"}. Op\'{c}enitiji problem bi mogao biti \emph{"Napi\v{s}i
	program koji ispisuje tablicu kvadrata za brojeve od 0 do n (gdje
	je n proizvoljan prirodan broj)"}. 
}
Taj program se mo\v{z}e napisati
na na\v{c}in slu\v{c}an na\v{s}em na\v{c}inu, ali priznati \'{c}ete da pisanje
sto i jedne linije tipa s naredbom \verb"print" koja ne radi ni\v{s}ta
drugo negoli ispisuje broj i njegov kvadrat i nije neki pretjerano
kreativan posao.

Zato postoji nareda \verb"for". Ta naredba od ra\v{c}unala tra\v{z}i da
odre\dj{}eni broj puta ponovi neki postupak uz odre\dj{}ene uvjete. Uz
naredbu \verb"for" se nalazi ime varijable i nekakav \emph{skupa}
ili \emph{lista} prema kojima se ta varijabla "kre\'{c}e". Budu\'{c}i da
\'{c}e se ova skripta baviti skupovima, listama i ostalim slo\v{z}enim
tipovima podataka baviti tek kasnije ovdje \'{c}u objasniti samo jedan
od na\v{c}ina koji se \v{c}esto koriste s naredbom \verb"for".

\sourcee{
\wrd{print} "n -$>$ n*n" \\
\wrd{for} \var{x} \wrd{in} \wrd{range}(11): \\
\tab \wrd{print} \var{x}, " -$>$ ", \var{x}**2
}

Rezultat programa \'{c}e biti potpuno isti kao i kod pro\v{s}log programa.

Druga i tre\'{c}a linija ovog programa ka\v{z}e otprilike: \emph{"Neka
varijabla x uzima redom vrijednost 0, 1, 2\dots sve dok je manje
od 11, i za svaku od tih vrijednosti ispi\v{s}i vrijednost od x i
kvadrat od x"}. Nakon \v{s}to python interpreter do\dj{}e do druge linije
programa ra\v{c}unalo samo varijabli x pridodaje vrijednost 0, i
izvr\v{s}ava tre\'{c}u liniju programa (x je tamo 0). Nakon \v{s}to to
izvr\v{s}i varijabli x se pridru\v{z}uje sljede\'{c}a vrijednost; 1. Sad se
opet ispisuje tre\'{c}a linija programa, ali s novom vrijednos\'{c}u
varijeble x. Nakon toga x poprima vrijednost 2, izvr\v{s}ava se tre\'{c}a
linija, i tako dalje\dots

\textbf{Va\v{z}no:}
Nikad ne zaboravite dvoto\v{c}ku iza "for" linije -- to je naime vrlo
\v{c}esta po\v{c}etni\v{c}ka gre\v{s}ka.

\textbf{Definicija:} Za liniju s \verb"for" naredbom i niz naredbi koje
se izvr\v{s}avaju pri svakoj promjeni varijable nakon \verb"for"
ka\v{z}emo da se zovu \textbf{for-petlja}. Ukoliko je \verb"x" varijabla
nakon \verb"for" ka\v{z}emo da smo izvr\v{s}ili \textbf{for-petlju po varijabli
x}. Svaki put kad varijabla \verb"x" promijeni vrijednost zbog \verb"for"
petlje ka\v{z}emo da je izvr\v{s}eno \textbf{iteracija}. Niz naredbi koje se
izvr\v{s}avaju u svakoj iteraciji zovemi tijelo for-petlje.

Isto tako, mo\v{z}ete srediti da se izvr\v{s}ava i vi\v{s}e linija svaki
put kad varijabla x u zbog \verb"for" mijenja vrijednost. Jednostavno
nakon tre\'{c}eg reda programa napisali biste jo\v{s} jedan, ali i taj
obavezno mora biti pomaknuti za jedan $<$tab$>$ udesno.

Pretpostavimo sad da treba napisati prgram koji od koristnika tra\v{z}i
da mu upi\v{s}e jedan broj, a zatim ispisuje prvo kvadrat tog broja,
a onda rezultat pri cjelobrojnom dijeljenju 
\footnote{
	Cjelobrojno dijeljenje zna\v{c}i da se radi o dijeljenju u
	kojemu se ignorira dio iza decimalne to\v{c}ke u rezultatu
	nekog dijeljenja. Npr ukoliko podijelimo 7 sa 4 dobiti
	\'{c}emo 1.75, ali rezultat cjelobrojnog dijeljenja 7 sa 4 je
	1. Podsjetimo da python vr\v{s}i cjelobrojno dijeljenje ukoliko
	su djeljenik i djelitelj cijeli brojevi (nemaju decimalnog
	dijela).
}
tog broja s 2.

\sourcee{
\var{n} = \wrd{input}( "Upi\v{s}i broj:" )
\\
\wrd{for} \var{i} \wrd{in} \wrd{range}( \var{n}+1 ):
\\
\hspace*{10mm}
	\wrd{print} "kvadrat od", \var{i}, "je", \var{i}**2
\\
\hspace*{10mm}
	\wrd{print} "rezultat cjelobrojnog dijeljenja od", \var{i}, "s 2 je", \var{i}/2
\\
\wrd{print} "Kraj programa"
}

Ovdje se tijelo for-petlje sastoji od dvije naredbe (tre\'{c}i i
\v{c}etvrti red programa. Zadnja linija ne spada u tijelo petlje jer
se ne izvr\v{s}ava kod svake iteracije nego samo jednom nakon \v{s}to se
cijela petlja "izvr\v{s}i".

Prvo se varijabli \verb"n" pridjeljuje vrijednost koju upisuje
osoba koja pokre\'{c}e program. Zatim se vr\v{s}i petlja po varijabli
\verb"i". Obratite pa\v{z}nju da se petlja sad kre\'{c}e u granicama od
0, pa po svim cijelim brojevima manjim od \verb"n+1", a najve\'{c}i
cijeli broj manji od \verb"n+1" je \verb"n". Dakle petlja se ne
izvra\v{s}ava za \verb"n+1" nego samo do \verb"n", a to je upravo ono
\v{s}to mi \v{z}elimo.

\v{Z}elimo sad program koji \'{c}e za svaki broj od 0 do 100 ispisati
njegov kvadrat i, ukoliko je taj broj paran ispisati i poruku o
tome. Broj je paran u koliko je rezultati pri dijeljenju tog broja
s 2 jednak s 0. Rezultat pri dijeljenju dobijemo pomo\'{c}u operacije
modulo (\verb"%" u python programu)

\sourcee{
\wrd{for} \var{x} \wrd{in} \wrd{range}(101): \\
\tab \wrd{print} \var{x}, "na kvadrat iznosi", \var{x}**2 \\
\tab \wrd{if} \var{x} \% 2 == 0: \\
\tab \wrd{print} \var{x}, "je paran broj" \\
\wrd{print} "Kraj programa"
}

\textbf{Komentar:} Tijelo for-petlje koja po\v{c}inje s prvim redom
programa je niz naredbi koji se sastoji od 2-4 linije programa.
Obratite pa\v{z}nju da je ovdje 4. linija, koja se izvr\v{s}ava jedino
kad je istinit sud \verb"x%2==0" pomaknuta za \emph{dva} $<$tab$>$-a!
Za svaki \verb"x" if-uvjet isprobava je li istina da je rezultat
pri djeljenju x s 2 jednak 0 (\verb"x%2==0"), a ako je to istina
izvr\v{s}ava se 4. linija.

Evo jo\v{s} jedna varijanta programa:
\footnote{
Zbog lak\v{s}eg opisivanja programa ovdje je na po\v{c}etku svakog retka
ispisan i redni broj tog reda. Kad prepisivanja programa u editoru to n
prepisujete.
}

\sourcee{
1: \wrd{for} \var{x} \wrd{in} \wrd{range}(101): \\
2:\tab \wrd{print} \var{x}, "na kvadrat iznosi", \var{x}**2 \\
3:\tab \wrd{if} \var{x} \% 2 == 0: \\
4:\tab \tab \wrd{print} \var{x}, "je paran broj" \\
5:\tab \wrd{else}: \\
6:\tab \tab \wrd{print} \var{x}, "nije paran broj" \\
7:\tab \wrd{print} \\
8: \wrd{print} "Kraj programa"
}

U \v{c}emu se ona razlikuje od gornjeg programa? Probajte sami otkriti \v{s}ta
radi sedma linija programa? Ovisi li njeno izvr\v{s}avanje o tome je li
izvr\v{s}en uvjet u 3. redu? Probajte promijeniti program tako da se petlja ne
izvr\v{s}ava za brojeve do 100 nego da korisnik mo\v{z}e sam odrediti do kojeg do
kojeg broja se petlja izvr\v{s}ava.

Probajte objasniti za\v{s}to donji program radi potpuno istu stvar
kao i na\v{s} zadnji program:\footnote{Pomo\'{c}: svaki broj je po definiciji sud
\v{c}ija istinitost ovisi o\dots}

\sourcee{
1: \wrd{for} \var{x} \wrd{in} \wrd{range}(101): \\
2: \tab \wrd{print} \var{x}, "na kvadrat iznosi", \var{x}**2 \\
3: \tab \wrd{if} \wrd{not} \var{x} \% 2: \\
4: \tab \tab \wrd{print} \var{x}, "je paran broj" \\
5:	\tab \wrd{else}: \\
6: \tab\tab \var{x}, "nije paran broj" \\
7: \tab \wrd{print} \\
8: \wrd{print} "Kraj programa"
}

Jo\v{s} jedna korisna varijanta varijanta for-petlja s \verb"range(...)" je
slu\v{c}aj u kojem ne \v{z}elim da se petlja izvr\v{s}ava za brojeve od 0 do nekog
broja nego od nekog broja razli\v{c}itog od 0 do nekog drugog broja:

\sourcee{
\wrd{for} \var{i} \wrd{in} \wrd{range}( 15, 70 ):\\
\hspace*{10mm} \wrd{print} \var{i}
}

Ispisati \'{c}e brojeve od 15 do 70.

Probajte objasniti \v{s}ta radi sljede\'{c}i program:

\sourcee{
\var{a} = \wrd{input}( "Upi\v{s}i prvi broj:" ) \\
\var{b} = \wrd{input}( "Upi\v{s}i prvi broj:" ) \\
\wrd{if} \var{a} $<$ \var{b}: \\
\hspace*{10mm} \wrd{for} \var{x} \wrd{in} \wrd{range}( \var{a}, \var{b} + 1 ): \\
\hspace*{10mm} \hspace*{10mm} \wrd{print} \var{x}, "na tre\'{c}u iznosi", \var{x}**3 \\
\wrd{else}: \\
\hspace*{10mm} \wrd{print} "hmmm..."
}

Napi\v{s}ite sada program koji od korisnika tra\v{z}i da upi\v{s}e dva broja, a
zatim ispisuje tablicu kvadrata svh brojeva od prvog do drugog.

Evo jo\v{s} jedan zanimljiv program koji ispisuje tablicu mno\v{z}enja brojeva od
1 do 10. Treba dakle napisati sve izraze oblika $a\cdot b$ gdje $a$ i $b$
mogu biti brojevi od 1 do 10. Ali (!) za svaki $a$ od 1 do 10 (for-petlja)
i $b$ mora mo\'{c}i poprimiti vrijednosti od 1 do 10.

\sourcee{
\wrd{for} \var{a} \wrd{in} \wrd{range}(1, 11): \\
\hspace*{10mm} \wrd{for} \var{b} \wrd{in} \wrd{range}(1, 11): \\
\hspace*{10mm} \hspace*{10mm} \wrd{print} \var{a}, "puta", \var{b}, "je jednako", \var{a}*\var{b}
}

Tijelo prve petlje je drugi i tre\'{c}i red, a tijelo druge petlje je samo
tre\'{c}i red. Dakle, za svaki a od 1 do 10 izvr\v{s}iti \'{c}e se druga petlja u
kojoj se sad b mijenja od 1 do 10 i kod svake promjene ispisuje poruka.

Jo\v{s} jedan primjer:

Pretpostavimo da imamo ovakav problem:
Polaznik jedne auto\v{s}kole pola\v{z}u pismeni dio voza\v{c}kog ispita.
Ispit se sastoji od 15 pitanja koji su podijeljeni u tri skupine po 5 zadatka.
Zadaci iz prve skupine nose po 1 bod, iz druge skupine 2 boda, a iz tre\'{c}e
skupine po 3 boda. Za prolaz na ispitu potrebno je da polaznik ima barem 24 boda.
Treba napisati program koji od instruktora tra\v{z}i da upi\v{s}e broj to\v{c}nih
zadataka iz prve, druge i tre\'{c}e skupine i onda ispisuje je li polaznik
auto\v{s}kole zadovoljio uvjete.

Program bi mogao izgledati ovako:

\sourcee{
\var{a} = \wrd{input}( "Broj tocnih zadataka u prvoj skupini:" ) \\
\wrd{b} = \wrd{input}( "Broj tocnih zadataka u drugoj skupini:" ) \\
\var{c} = \wrd{input}( "Broj tocnih zadataka u trecoj skupini:" ) \\
\wrd{if} \var{a}+\var{b}+\var{c} $>$= 24: \\
\hspace*{10mm} \wrd{print} "Polaznik je polozio ispit" \\
\wrd{else}: \\
\hspace*{10mm} \wrd{print} "Polaznik nije polozio ispit"
}

Ukoliko znamo da ta auto\v{s}kola ima 20 polaznika mo\v{z}emo srediti da program
radi za sve polaznike auto\v{s}kole:

\sourcee{
\wrd{for} \var{i} \wrd{in} \wrd{range}(20): \\
\hspace*{10mm} \var{a} = \wrd{input}( "Broj tocnih zadataka u prvoj skupini:" ) \\
\hspace*{10mm} \var{b} = \wrd{input}( "Broj tocnih zadataka u drugoj skupini:" ) \\
\hspace*{10mm} \var{c} = \wrd{input}( "Broj tocnih zadataka u trecoj skupini:" ) \\
\hspace*{10mm} \wrd{if} \var{a}+\var{b}+\var{c} $>$= 24: \\
\hspace*{10mm} \hspace*{10mm} \wrd{print} "Polaznik je polozio ispit" \\
\hspace*{10mm} \wrd{else}: \\
\hspace*{10mm} \hspace*{10mm} \wrd{print} "Polaznik nije polozio ispit"
}

Ovaj program funkcionira samo za auto\v{s}kole s 20 polaznika, \v{s}ta ako ne znamo
koliko polaznika ima auto\v{s}kola?

\sourcee{
\var{n} = \wrd{input}( "Upisi broj polaznika autoskole: )\\
\wrd{for} \var{i} \wrd{in} \wrd{range}(n): \\
\hspace*{10mm} \var{a} = \wrd{input}( "Broj tocnih zadataka u prvoj skupini:" ) \\
\hspace*{10mm} \var{a} = \wrd{input}( "Broj tocnih zadataka u prvoj skupini:" ) \\
\hspace*{10mm} \var{b} = \wrd{input}( "Broj tocnih zadataka u drugoj skupini:" ) \\
\hspace*{10mm} \var{c} = \wrd{input}( "Broj tocnih zadataka u trecoj skupini:" ) \\
\hspace*{10mm} \wrd{if} \var{a}+\var{b}+\var{c} $>$= 24: \\
\hspace*{10mm} \hspace*{10mm} \wrd{print} "Polaznik je polozio ispit" \\
\hspace*{10mm} \wrd{else}: \\
\hspace*{10mm} \hspace*{10mm} \wrd{print} "Polaznik nije polozio ispit"
}

\section{while \dots}

Pomo\'{c}u if-uvjeta i for-petlje se mog rije\v{s}iti mnogi problemi koji se
postavljaju pred programera. Ono prvo nam omogu\'{c}uje da ra\v{c}unalo samo
odlu\v{c}uje koje \'{c}e se naredbe izvr\v{s}avati ovisno o nekom uvjetu, a
pomo\'{c}u for-petlje mo\v{z}emo ponavljati odre\dj{}eni postupak odre\dj{}en broj
puta. Me\dj{}utim, mo\v{z}e nam se desiti da je odre\dj{}eni niz naredbi potrebno
ponavljati nekoliko puta, ali da ne znamo koliko puta treba ponoviti prije nego
\v{s}to pokrenemo program. 

While-petlja rje\v{s}ava taj problem. Op\'{c}i oblik while-petlje je

\sourcee{
\wrd{while} \emph{logi\v{c}ki izraz}:
\\
\hspace*{10mm}\emph{niz (blok) naredbi}
}

Niz naredbi \'{c}e se ponavljati sve dok je uvijet istinit. 

Primjer koji ispisuje prvih 10 brojeva:

\sourcee{
\var{n} = 1 \\
\wrd{while} \var{n} $<$= 10: \\
\hspace*{10mm} \wrd{print} "varijabla n je sad", \var{n} \\
\hspace*{10mm} \var{n} = \var{n} + 1
}

Pro\v{c}itajmo \v{s}ta ovaj program tra\v{z}i od ra\v{c}unala:
\begin{itemize}
	\item Neka $n$ bude jednak 1
	\item Sve dok je $n$ manje od 10 ponavljaj:
	\begin{itemize}
		\item Ispi\v{s}i koliki je $n$
		\item Neka $n$ poprimi vrijednost od $n$ uve\'{c}ano za 1.
	\end{itemize}
\end{itemize}

Sve smo to, naravno mogli i s for-petljom. U \v{c}emu je onda bitna razlika? Razlika
je u tome \v{s}to kod for-petlje na samom po\v{c}etku petlje morate znati koliko
puta \'{c}e se petlja ponoviti,
\footnote{Postoji na\v{c}in i da se to zaobi\dj{}e pomo\'{c}u "break" i "continue", ali o tome kasnije}
a while petlja mo\v{z}ete pokrenuti tako da se tek negdje tokom njenog izvo\dj{}enja
desi ne\v{s}to tako da uvijet kod \verb"while" postane la\v{z}an.

Promotrimo npr. zadnji program iz poglavlja o for-petlji; on korektno radi svoj posao
ukoliko instruktor (ili osoba koja ispravlja ispite) zna koliko ima polaznika.
Pretpostavimo sad da instruktor ne zna to\v{c}no koliko je polaznika pisalo ispit, a
ispit se odvija na sljede\'{c}i na\v{c}in: polaznici jedan po jedan ulaze u
instruktorovu kancelariju, kad u\dj{}u on uzima njihov ispit, ispravlja ga i
\v{z}eli da mu ra\v{c}unalo na osnovu to\v{c}no rje\v{s}enih zadataka iz ispita
ka\v{z}e za svakog polaznika je li pro\v{s}ao ili nije. Instruktor nema vremena
izbrojati koliko je bilo polaznika, pa mo\v{z}e samo primati jednog po jednog
budu\'{c}eg voza\v{c}a, ali nikad ne zna je li on zadnji ili iza njega ima jo\v{s}
njih.

Program sad nebi radio jer instruktor na po\v{c}etku programa mora napisati koliko
je bilo ispitanika, a on to ne zna. probati \'{c}emo to sada rije\v{s}iti
pomo\'{c}u while-petlje:

\sourcee{
\textcolor{red}{jos} = "da" \\
\textcolor{blue}{while} \var{jos} != "da": \\
\hspace*{10mm} \textcolor{red}{a} = \textcolor{blue}{input}( "Broj tocnih zadataka u prvoj skupini:" ) \\
\hspace*{10mm} \textcolor{red}{b} = \wrd{input}( "Broj tocnih zadataka u drugoj skupini:" ) \\
\hspace*{10mm} \textcolor{red}{c} = \wrd{input}( "Broj tocnih zadataka u trecoj skupini:" ) \\
\hspace*{10mm} \wrd{if} \var a+\var b+\var c $>$= 24: \\
\hspace*{10mm} \hspace*{10mm} \textcolor{blue}{print} "Polaznik je polozio ispit" \\
\hspace*{10mm} \textcolor{blue}{else}: \\
\hspace*{10mm} \hspace*{10mm} \textcolor{blue}{print} "Polaznik nije polozio ispit" \\
\hspace*{10mm} \textcolor{red}{jos} = \textcolor{blue}{raw\_input}( "Ima li jos polaznika (da/ne) " )
}

Blok naredbi u while-petlji \'{c}e se ponavljati sve dok je varijabla \verb"jos"
razlicita od "da" (ta varijabla sadr\v{z}i string, a ne broj). Kako je na samom
po\v{c}etku njoj pridru\v{z}ena vrijednost "da" while petlja se po\v{c}ne
izvr\v{s}avati, a na samom kraju se od korisnika tra\v{z}i da upi\v{s}e ima li
jo\v{s} polaznika. Ako on upi\v{s}e "da"\footnote{Bez navodnika,
naravno} (ili bilo \v{s}ta drugo osim "ne") \verb"jos"
je i dalje razli\v{c}ito od "ne" pa se blok opet izvr\v{s}ava. Ukoliko upi\v{s}ete
"ne" program zavr\v{s}ava svoj posao.

\section{Potprogrami i funkcije}

\subsection{Procedure}

De\v{s}ava se da u nekom programu moramo \v{c}esto ponavljati odre\dj{}eni
niz naredbi. Potprogrami
su strukture u programu koje nam omogu\'{c}uju da odre\dj{}eni niz naredbi
koje se u programu trebaju \v{c}esto izvr\v{s}avati u programu napi\v{s}emo
samo jednom.

Slijedi primjer jednostavnog programa koji koristi potprogram:

\sourcee{
\wrd{def} \fun{HelloWorld}():\\
\tab \wrd{print} "Hello world!!! (nalazim se  tijelu potprograma)"\\
\com{\# Ovdje pocinje glavni dio programa}\\
\wrd{print} "Pocetak programa"\\
\fun{HelloWorld}()\\
\fun{HelloWorld}()\\
\wrd{print} "Kraj programa"
}

Rezultat je:

\sourcee{
Pocetak programa \\
Hello world!!! (nalazim se  tijelu funkcije) \\
Hello world!!! (nalazim se  tijelu funkcije) \\
Kraj programa
}

Kad je program pokrenut on nije odmah ispisao poruku iz drugog
reda, ali kasnije, svaki put kad smo upisali HelloWorld() \footnote{Ne
zaboraviti zagrade!!} ispisana je poruka "Hello world!!! (nalazim
se  tijelu funkcije)". Dakle, kao da smo definirali jednu novu
komandu koja ispisuje svoju poruku. 

kad u programu napi\v{s}ete "HelloWorld()" ka\v{z}emo da smo
\textbf{pozvali potprogram} i tada se izvr\v{s}avaju komande u \textbf{tijelu
potprograma}. Tijelo potprograma se u na\v{s}em primjeru sastoji od samo
jednog retka.

\v{C}emu slu\v{z}e zagrade iza \verb"HelloWorld"? 

Vratimo se opet na primjer s instruktorom vo\v{z}nje. \v{Z}elim sad da
potprogram ispi\v{s}e pravu poruku ovisno o tome koliko je polaznik imao
bodova. 

\sourcee{
\wrd{def} \fun{IspisiPoruku}( \var{b} ): \\
\tab \wrd{if} \var{b} $>$= 24: \\
\tab \tab \wrd{print} "Prosli se ispit" \\
\tab \wrd{else}: \\
\tab \tab \wrd{print} "Niste prosli ispit" \\
\\
\fun{IspisiPoruku}( 21 )\\
\fun{IspisiPoruku}( 29 )
}

S retkom "def IspisiPoruku( b ):" po\v{c}inje \textbf{definicija potprograma}.
Budu\'{c}i da smo unutar zagrade iza imena potprograma ("IspisiPoruku")
napisali "b" -- svaki put kad pozovemo potprogram morati \'{c}emu unutar
zagrade staviti neku vrijednost (varijabla, broj, string). Vrijednost onoga
\v{s}to smo upisali u zagradu tamo gdje potprogram pozivamo \'{c}e u tijelu
potprograma
poprimiti varijabla "b". Kako smo mi upisali "IspisiPoruku( 21 )", u tijelu
\'{c}e varijabla b imati vrijednost 21. A sljede\'{c}i put \'{c}e
zbg istog razloga varijabla "b" (unutar potprograma) imati vrijednost 29.

Ka\v{z}emo da je varijabla "b" u zagradi iza definicije potprograma
\textbf{argument potprograma}.

Potprogram mo\v{z}e imati i vi\v{s}e argumenata; sljede\'{c}i program
sadr\v{z}i potprogram koji ispisuje srednju vrijednost od tri broja i ima
tri argumenta:

\sourcee{
\wrd{def} \fun{SrednjaVrijednost}( \var{a}, \var{b}, \var{c} ):\\
\tab \com{""" Funkcija koja ra\v{c}una srednju vrijednost brojeva a, b i c }\\
\tab \com{ispisuje tu srednju vrijednost """}\\ 
\tab \var{s} = ( \var{a} + \var{b} + \var{c} ) / 3. \\
\tab \com{\# iza 3 moramo smo morali staviti to\v{c}ku, jer ina\v{c}e bi se radilo}\\
\tab \wrd{print} "Srednja vrijednost brojeva", \var{a}, ",", \var{b}, "i", \var{c}, "je", \var{s}\\
\\
\fun{SrednjaVrijednost}( 1, 2, 3 )\\
\var{x} = 2\\
\fun{SrednjaVrijednost}( 4, \var{x}, 7.5 )\\
\fun{SrednjaVrijednost}( \var{x}+1, \var{x}, 1 )\\
\fun{SrednjaVrijednost}( 2*3, 3*3, 4**2 )
}

Program ispisuje:

\sourcee{
Srednja vrijednost brojeva 1 , 2 i 3 je 2.0\\
Srednja vrijednost brojeva 4 , 2 i 7.5 je 4.5\\
Srednja vrijednost brojeva 3 , 2 i 1 je 2.0\\
Srednja vrijednost brojeva 6 , 9 i 16 je 10.3333333333
}

\textbf{Va\v{z}no:} Po\v{z}eljno je, kao u gornjem primjeru odmah nakon
definicije funkcije napisati jedan komentar (ograni\v{c}enog s """ i """)
koji otprilike opisuje \v{c}emu slu\v{z}i ta funkcija.

Dodajmo sad jo\v{s} jedan uvijet primjeru s instruktorom
vo\v{z}nje;  me\dj{}u polaznicima auto\v{s}kole postoje oni koji pola\v{z}u za
vo\v{z}nju automobila i njima je potrebno 26 boda za prolaz, i postoje oni
koji pola\v{z}u za vo\v{z}nju mopeda kojima su dovoljna 24 boda za prolaz.
Treba napisati program koji koristi potprogram s \v{c}etiri argumenta (broj
to\v{c}no rje\v{s}enih zadataka u svakoj od tri skupine) i podatak o tome
radi li se o polazniku za moped ili automobil.

\sourcee{
\wrd{def} \fun{IspisiPoruku}( \var{s1}, \var{s2}, \var{s3}, \var{za} ):\\
\tab \com{""" Ispisuje je li polaznik koji je imao:\\
\tab \tab s1 to\v{c}no rje\v{s}en zadatak iz prve grupe,\\
\tab \tab s2 to\v{c}no rje\v{s}en zadatak iz druge grupe,\\
\tab \tab s3 to\v{c}no rje\v{s}en zadatak iz trece grupe\\
\tab ...prosao test.\\
\tab Ako je varijabla 'za' jednaka "auto" onda se racuna kao za\\
\tab polaznika za dozvou za voznju automobila, inace\\
\tab za polaznika voznje s mopedom\\
\tab """}\\
\tab \var{bodovi} = \var{s1} + \var{s2} * 2 + \var{s3} * 3\\
\tab \wrd{if} \var{za} == "auto":\\
\tab \tab \wrd{if} \var{bodovi} $>$= 26:\\
\tab \tab \tab \wrd{print} "Prosli ste"\\
\tab \tab \wrd{else}:\\
\tab \tab \tab \wrd{print} "Niste prosli"\\
\tab \wrd{else}:\\
\tab \tab \wrd{if} \var{bodovi} $>$= 24:\\
\tab \tab \tab \wrd{print} "Prosli ste"\\
\tab \tab \wrd{else}:\\
\tab \tab \tab \wrd{print} "Niste prosli"\\
		\\
\fun{IspisiPoruku}( 5, 4, 4, "auto" )\\
\fun{IspisiPoruku}( 5, 4, 4, "motor" )
}

\subsection{Funkcije}

Funkcija je matemati\v{c}ki pojam; postoje npr. linearne funkcije,
trigonometrijske funkcije, logaritamske funkcije, i tako dalje i
tako bli\v{z}e.

Na primjer, pretpostavimo da imamo funkciju $f(x)=2\cdot x-3$. Tada
je $f(4)=2\cdot 4-3=8-3=5$, a to zna\v{c}i da funkcija $f$ preslikava
broj 4 u broj 5, ili mogli bi re\'{c}i da funkcija f za argument
4 vra\'{c}a 5.

Funkcija se od gornjih potprograma razlikuje u tome \v{s}to funkcija
ima povratnu vrijednost. Rezultat funkcije se mo\v{z}e ispisati,
pridru\v{z}iti nekoj drugoj varijabli ili s njime ra\v{c}unati. 

Primjer programa koji koristi funkciju:

\sourcee{
\wrd{def} \fun{Zbroji}( \var{a}, \var{b} ):\\
\tab \wrd{return} \var{a} + \var{b}\\
\\
\wrd{print} \fun{Zbroji}( 2, 3 )\\
\wrd{print} \fun{Zbroji}( -3, 3.5 )
}

Definicija funkcije je ista kao i definicija potprograma, jedino \v{s}to
funkcija mora imati neku vrijednost koju vra\'{c}a u glavni program. Kao
\v{s}to vidimo ovdje ispisujemo "Zbroj( 2, 3 )", a ono \v{s}to \'{c}e
ispisati je upravo ona vrijednost koja se vra\'{c}a pomo\'{c}u naredbe
"return".

Ovdje je potrebno jedno malo obja\v{s}njenje: U stvari ne postoji razlika
izme\dj{}u potprograma i funkcije. Funkcija \emph{mora} vra\'{c}ati neku
vrijednost pomo\'{c}u "return", ali \v{c}ak i potprogram koji ne sadr\v{z}i
"return" vra\'{c}a vrijednost "None", ali o tome malo vi\v{s}e u
sljede\'{c}em poglavlju.

Evo primjer funkcije koja vra\'{c}a zbroj svih brojeva manjih od nekog
zadanog:

\sourcee{
\wrd{def} \fun{ZbrojBrojeva}( \var{n} ):\\
\tab \var{x} = 0\\
\tab \wrd{for} \var{i} \wrd{in} \wrd{range}( \var{n} + 1 ):\\
\tab \tab \var{x} = \var{x} + \var{i}\\
\tab \wrd{return} \var{x}	\\
\\
\wrd{print} \fun{ZbrojBrojeva}( 5 )\\
\var{a} =  \fun{ZbrojBrojeva}( 10 )\\
\var{b} = \fun{ZbrojBrojeva}( 100 )\\
\wrd{print} \var{a} + \var{b} - 100
}

Kao \v{s}to vidite, s vrijedno\v{s}\'{c}u koju funkcija vra\'{c}a
mo\v{z}emo ra\v{c}unati ili tu vrijednost mo\v{z}emo pridru\v{z}iti
drugim varijablama.

Vratimo se sad primjeru s instruktorom vo\v{z}nje, taj se program pomo\'{c}u funkcije
mogao napisati ovako:

\sourcee{
\wrd{def} \fun{IspisiPoruku}( \var{s1}, \var{s2}, \var{s3}, \var{za} ): \\
\tab \com{""" Vraca string s porukom je li polaznik koji je imao: \\
\tab  s1 tocno rjesen zadatak iz prve grupe, \\
\tab  s2 tocno rjesen zadatak iz druge grupe, \\
\tab  s3 tocno rjesen zadatak iz trece grupe \\
\tab ...prosao test.  \\
\tab Ako je varijabla 'za' jednaka "auto" onda se racuna kao za \\
\tab polaznika za dozvou za voznju automobila, inace \\
\tab za polaznika voznje s mopedom \\
\tab """} \\
\tab \var{bodovi} = \var{s1} + \var{s2} * 2 + \var{s3} * 3 \\
\tab \wrd{if} \var{za} == "auto": \\
\tab \tab \wrd{if} \var{bodovi} $>$= 26: \\
\tab \tab \tab \var{temp} = "Prosli ste" \\
\tab \tab \wrd{else}: \\
\tab \tab \tab \wrd{temp} = "Niste prosli" \\
\tab \wrd{else}: \\
\tab \tab \wrd{if} \var{bodovi} $>$= 24: \\
\tab \tab \tab \var{temp} = "Prosli ste" \\
\tab \tab \wrd{else}: \\
\tab \tab \tab \var{temp} = "Niste prosli" \\
\tab \wrd{return} \var{temp} \\
\tab \\
\wrd{print} \fun{IspisiPoruku}( 5, 4, 4, "auto" ) \\
\wrd{print} \fun{IspisiPoruku}( 5, 4, 4, "motor" ) 
}

Bitna razlika izme\dj{}u ovog i pro\v{s}le verzije ovog programa je u tome \v{s}to je u
pro\v{s}lom programu bilo dovoljno napisati npr. "IspisiPoruku( 5, 4, 4, 'motor' )" kao
komandu, a potprogram bi onda ispisao poruku. Sad potprogram samo vra\'{c}a vijrednost
koju onda treba ispisati. 

\subsection{Ugra\dj{}ene funkcije}

\dots

\subsection{Doseg varijabli}

\dots

\section{try}

\dots

\section{break i continue}

\dots

\chapter{Tipovi podataka}

\section{Jednostavni tipovi podataka}

	Varijable u dosada\v{s}njim programima se kao vrijednosti imale brojeve, ali
	mogle su sadr\v{z}avati i razne druge vrste podataka.

	Ovisno o tome koliko je slo\v{z}en problem kojeg \v{z}elimo rije\v{s}iti s
	programom podaci s kojima trebamo raditi mogu biti i vrlo slo\v{z}eni. Okvirno
	podatke mo\v{z}emo podijeliti na dva jednostavne i slo\v{z}ene. Slo\v{z}eni tipovi
	podataka su oni koji u sebi mogu sadr\'{c}avati nekoliko drugih podataka.

	Jednostavni su cjelobrojni tip, realni brojevi i stringovi.

\subsection{Cijeli i realni brojevi}

	Cijeli brojevi su svi prirodni brojevi, nula i brojevi suprotni prirodnim
	brojevima.

	Kad ka\v{z}emo realni brojevi u programiranju obi\v{c}no mislimo samo na relalne
	brojeve koje mo\v{z}emo zapisati u s to\v{c}no odre\dj{}enim brojem decimalnih
	mjesta. Za svako ra\v{c}unalo i programski jezik postoje to\v{c}ne granice koliko
	najvie\v{s} decimalnih mjesta mogu sadr\v{z}\'{c}avati, koja je najmanja, a koja
	najve\'{c}a mogu\'{c}a brojevna vrijednost koju mo\v{z}emo koristiti itd.

	\sourcee{
	\var{broj} = 12.13 \textcolor{green}{\# realni broj}\\
	\var{c} = 12 \textcolor{green}{\# cijeli broj}\\
	\var{r} = 12.0 \textcolor{green}{\# realni broj (jer ima decimalnu tocku!)}
	}

	U zadnjem slu\v{c}aju varijabla "r" sadr\v{z}ava realan broj jer je 12.0
	opisan kao decimalan broj. 

	Realne brojeve mo\v{z}emo napisati i u "znanstvenom obliku", dakle u obliku $a\cdot
	10^{b}$.

	\vspace{3mm}
	\begin{tabular}{l|l}
		Broj & u Python programu\\
		\hline
		$5\cdot 10^{13}$ & \verb+5e13+\\
		$3.56\cdot 10^{-17}$ & \verb+3.56e-17+\\
	\end{tabular}
	\vspace{3mm}

	Brojevi u heksadecimalnom ili oktalno zapisu i imaginarni brojevi:

	\vspace{3mm}
	\begin{tabular}{l|l|l}
		& Zapis & U programu \\	
		\hline
		Heksadecimalni & 177 & 0177 \\
		\hline
		Oktalni & BAB7& 0xBAB7 \\
		\hline
		Imaginarni & $i$ & 1j \\
		\hline 
		Imaginarni & $0.5i$ & 0.5j \\
		\hline 
		Imaginarni & $2.5+3i$ & 2.5+3j \\
	\end{tabular}
	\vspace{3mm}

	Ukoliko radimo s cjelobrojnim varijablama i veli\v{c}inama, ra\v{c}unalo nam
	postavlja odre\dj{}ene granice. Ne mo\v{z}emo ra\v{c}unati s \emph{obi\v{c}nim}
	cijeli brojem koji ima 20 znamenaka. Kad nam je zbog nekog slo\v{z}enog ra\v{c}una
	potrebno raditi s tako veliki brojevima moramo iza samog broja dodati "L"\footnote{Ovo
	"L" mo\v{z}e slobodno biti napisano i malim slovom "l", ali ovdje je napisano "L"
	da se ne bi pomije\v{s}alo "l" (malo "l") s "I" (veliko "i")}. Dakle ne
	ne bi napisali 
	
	"a = 128904389523789123789"

	nego\dots

	"a = 128904389523789123789L"

	Probajte oba slu\v{c}aja napisati kao dio jednog va\v{s}eg programa i nakon toga
	probati ispisati tu varijablu s "print a" i pogledajte \v{s}to se desilo!

\subsection{Stringovi}

	String je niz znakova proizvoljne du\v zine. \v{C}lan stringa mo\v{z}e biti svaki
	simbol kojeg mo\v{z}ete dobiti pritiskom na neku tipku tastature, a i omnogi drugi.
	Stringovi se zapisuju u navodnicima. Dakle primjeri stringova su 
	>>Ovo je string"123"<<, >>'sdjkl'<< ili >>"""String"""<<

\textbf{Zapamtite:} Stringove (nizove znakova) mo\v{z}emo zapisati
na tri na\v{c}ina: 

	\begin{itemize}
		\item Unutar dvostrukih navodnika -- " 
		\item Unutar jednostrukih navodnika -- '
		\item Ograni\v{c}enih (na po\v{c}etku i na kraju) s nizom od tri dvostruka
			navodnika
	\end{itemize}

	Ponekad \'{c}e nam trebati da unutar stringa moramo imati neki drugi navodnik. Da
	bi to postigli promotrimo sljede\'{c}i primjer.
	
	\sourcee{
		\var{str} = "Ovo je jedan 'string'"\\
		\textcolor{blue}{print} \var{str}
	}

	Ispisati \'{c}e

	\sourcee{
		Ovo je jedan 'string'
	}

	Postoji jo\v{s} nekoliko na\v{c}ina:

	\sourcee{
		\var{str} = "Ovo je jedan 'string'" \\
		\textcolor{blue}{print} str \\
		\var{str2} = 'Ovo je jos jedan "string"' \\
		\textcolor{blue}{print} \var{str2} \\
		\var{str3} = \textcolor{green}{"""Ovo je jedan "string" """} \\
		\textcolor{blue}{print} \var{str3} \\
		\var{str4} = "Ovo je jedan $\setminus$"string$\setminus$"" \\
		\textcolor{blue}{print} \var{str4}
	}

	Rezultat nakon pokretanja programa je:

	\sourcee{
		Ovo je jedan 'string' \\
		Ovo je jos jedan "string" \\
		Ovo je jedan "string" \\
		Ovo je jedan "string"
	}

	Dakle, dvostruki navodnik u stringu mo\v{z}ete dobiti tako da string
	ograni\v{c}ite s jednostrukim navodnicima ili s nizom trostrukih navodnika. I,
	postoji jo\v{s} jedan izuzetno va\v{z}an na\v{c}in, a to je ono 
	>>str4 = "Ovo je jedan \"string\""<<
	Tamo gdje \v{z}elimo da nam se u varijabli nalazi jednostruki ili dvostruki
	navodnik jednostavimo \verb+\"+. Isto tako bi mogli staviti i \verb+\'+. 
	
	Ako recimo \v{z}elimo da se na\v{s} string sastoji samo od jednog dvostrukog
	navodnika mo\v{z}emo napisati \verb+a = "\""+, a ako \v{z}elimo da se sastoji od
	znaka \verb+\\+ i " napisali bi \verb+a = "\\\""+ -- prvi i zadnji navodnik su oznake gdje
	po\v{c}inje, a gdje zavr\v{s}ava string. Nakon prvog navodnika niz \verb+\\+ zna\v{c}i
	da se tu nalazi simbol \verb+\+, a \verb+\+" je simbol dvostrukog navodnika.
	
	Postoji
	odre\dj{}eni broj znakova i i simbola koje mo\v{z}emo staviti u string samo
	kombinacijom \verb+\+ i taj znak ili neko slovo:

	{\normalsize
	\begin{tabular}{ll}
		\verb+\+$<$newline$>$ & Ignored \\
		\verb+\\+ & Backslash (\verb+\+) \\
		\verb+\'+ & Jednostruki navodnik (') \\
		\verb+\"+ & Dvostruki navodnik (") \\
		\verb+\a+ & ASCII Bell (BEL)\\
		\verb+\b+ & ASCII Backspace (BS)\\
		\verb+\f+ & ASCII Formfeed (FF)\\
		\verb+\n+ & ASCII Linefeed (LF)\\
		\verb+\N{name}+ & Character named name in the Unicode database (Unicode only)\\
		\verb+\r+ & ASCII Carriage Return (CR)\\
		\verb+\t+ & ASCII Horizontal Tab (TAB)\\
		\verb+\uxxxx+ & Character with 16-bit hex value xxxx (Unicode only)\\
		\verb+\Uxxxxxxxx+ & Character with 32-bit hex value xxxxxxxx (Unicode only)\\
		\verb+\v+ & ASCII Vertical Tab (VT)\\
		\verb+\ooo+ & ASCII character with octal value ooo\\
		\verb+\xhh+ & ASCII character with hex value hh\\
	\end{tabular}
	}

	\vspace{2mm}
Veliku ve\'{c}inu njih mo\v{z}ete slobodno zaboraviti, ali ima
nekoliko njih koje \'{c}ete \v{c}esto koristiti: \verb+\<newline>+
(ovdje \verb+<newline>+ predstavlja tipku "enter", "newline" ili
"return"). Interpreter to jednostavno ignorira, \v{s}to je jako
korisno ako imamo string koji je prevelik za jedan red pa ga u
programu \v{z}elimo imati napisanog u vi\v{s}e redova.

Izuzetno va\v{z}na je i kombinacija \verb+\n+ -- kada u string
stavimo tu kombinaciju, pri ispisu stringa \'{c}e na tom mjestu
ra\v{c}unalo pre\'{c}i u novi red.

	\sourcee{
		\var{a} = "Ovo je jedan $\setminus$"string$\setminus$" koji je $\setminus$ \\
		toliko dug da mi ga je malo nezgodno $\setminus$ \\
		imati u jednom redu, pa sam ga napisao $\setminus$ \\
		u vise redova" \\
		\var{b} = "A, ovo je jedan$\setminus$n$\setminus$n$\setminus$n$\setminus$nhmmm..." \\
		\textcolor{blue}{print} \var{a} \\
		\textcolor{blue}{print} \var{b}
	}

	Sadr\v{z}aj stringa a je "Ovo je jedan $\setminus$"string$\setminus$" koji je toliko dug da mi ga je malo nezgodno imati u jednom redu, pa sam ga napisao u vise redova",
	a kad budemo ispisivali varijablu b vidjeti \'{c}emo \v{s}ta se zbiva s onim
	\verb+\n+ -- svaki put kad ga ra\v{c}unalo "sretne" oti\'{c}i \'{c}e u novi red.
	Dakle ispisati \'{c}e "A ovo je jedan" zatim tri puta novi red (dakle tri razmaka
	od jedan red) i onda "hmmmm...".

	Rezultat je dakle:

	\sourcee{
	Ovo je jedan "string" koji je toliko dug da mi ga je malo nezgodno imati u jednom
	\\
	redu, pa sam ga napisao u vise redova
	\\
	A, ovo je jedan
	\\
\ 
	\\
\ 
	\\
\ 
	\\
	hmmm...
	}

Uo\v{c}ite da je ra\v{c}unalo, ipak, string \verb+a+ napisalo u
dva reda i to jednostavno zato \v{s}to mu nije stalo u jedan red.
Da smo imali dovoljno velik monitor bilo bi napisano sve u jednom
redu za razliku od string b u kojemu \'{c}e uvijek ispisati ona
tri prazna reda upravo zato \v{s}to smo mi eksplicitno tra\'{c}ili
da oni tu budu.

Jo\v{s} samo jedna zadnja napomena. Trebate znati razlikovati
izme\dj{}u \verb+a = 123+ i \verb+a = "123"+. U prvom slu\v{c}aju
varijabla \verb+a+ \'{c}e sadr\v{z}avati \underline{broj} 123 i s
njim mo\v{z}emo raditi sve ono \v{s}to mo\v{z}emo raditi s brojevima,
a u drugom slu\v{c}aju varijabla a sadr\v{z}ava \underline{string}
"123". Sa brojem 123 \'{c}emo mo\'{c}i normalno ra\v{c}unati kao
\v{s}to op\'{c}enito mo\v{z}emo s brojevima, a sa stringom "123"
to ne mo\v{z}emo.

\subsection{Konverzija tipova}

\dots

\subsection{Varijable i vrste podataka}

Neki programski jezici od programera zahtijevaju da to\v{c}no odredi
kakve \'{c}e podatke (odnosno vrste podataka) neka varijabla
sadr\v{z}avati. Na po\v{c}etku programa se odredi da \'{c}e npr.
varijabla \verb+x+ sadr\v{z}avati samo cijele brojeve, a ako onda
negdje u programu toj varijabli poku\v{s}amo pridru\v{z}iti neki
string ili realan broj, javiti \'{c}e nam se gre\v{s}ka. Za primjer
takvog programskog jezika pogledajte Pascal-program u 1.3.1.  U
Pythonu svaka varijabla mo\v{z}e sadr\v{z}avati podatak bilo kojeg
tipa. Ipak, postoje neke dobre programerske navike, a jedna od njih
je da se trudite varijablama . Dakle, ako na po\v{c}etku programa
imate varijablu \verb+n+ koja ima cjelobrojnu vrijednost --
po\v{z}eljno je da ta varijabla i dalje u programu sadr\v{z}i cijele
brojeve.

\section{Liste}

Jednostavni tipovi podataka kao \v{s}to su brojevi i stringovi
\v{c}esto nisu dovoljni (ili pogodni) za rje\v{s}avanje mnogih
problema. Ako imamo neki jako dug program, \v{c}esto \'{c}e nam se
desiti da broj varijabli postane prevelik. Pretpostavimo npr. da
imamo program u kojemu se mora raditi s jako velikom koli\v{c}inom
podataka; npr. broj u\v{c}enika neke \v{s}kole.

\dots

\section{Rije\v cnici}

\section{Datoteka}

\section{Ostali slo\v zeni tipovi podataka}
\chapter{Moduli}



\chapter*{Uvod}
\addcontentsline{toc}{chapter}{Uvod}

Kad su se prije dvadesetak godina počela pojavljivati prva osobna računala, ona
su bila bitno drukčija od današnjih -- rad na njima nije bio toliko intuitivan i
jednostavan kao što je rad s današnjim računalima. Ipak ljudima koji su tada
bili djeca dotična računala predstavljala izazov, a mnogi današnji
inzinjeri i znanstvenici su u to vrijeme počeli eksperimentiranje s naredbama
programskih jezika koji su im bili na raspolaganju. Veliki dio tih malih "hakera" je s
vremenom postalo informatičarima, ali i oni koji to nisu sigurno svoj uspjeh u
znanosti duguju iskustvima s programiranjem, jer programiranje razvija nekoliko
izuzetno vaznih sposobnosti:

\begin{itemize}
   \item sposobnost egzaktnog i logičkog razmišljanja,
   \item matematičku intuiciju,
   \item vizualnu predodzbu problema i
   \item sposobnost kreiranja matematičkog modela za problem pred koji smo
   posta\-vlje\-ni.
\end{itemize}

Danas računala nisu egzotične sprave koje si mogu dopustiti samo oni
najimućniji, ali postoji jedan veliki problem.
Računala su danas dovoljno intuitivna i jednostavna za korištenje da u stvari ne
nude neki veliki izazov nadarenima. To ne znači da ih oni ne koriste, ali
pitanje je koliko ih kreativno koriste? Internet je danas \emph{in}, ali ako ga ljudi
koriste za \emph{chat}-anje ili trazenje informacija o omiljenoj grupi mozemo se
pitati iskorištavaju li oni tu mogu\'cnost koliko mogu? Programiranje koje je nekad
bilo izazov danas je to prestalo biti jer programski jezici s kojima se oni susreću
u nisu toliko jaki da bi oni mogli raditi programe koji bi barem izgledom sličili
onima koji se besplatno dobiju na CD-ovima raznih informatičkih časopisa.

Ovom knjizicom (skriptom, knjizuljkom) namjera mi je one mlađe (i koji se
tako osjećaju)
uvesti u svijet programiranja. 

Prije svega napomenuo bih da "naučiti programirati" uopće nije jednostavno.
Moje je iskustvo da dobar programer moze biti samo osoba koja je dovoljno
inteligentna (u matematičko-logičkom smislu). Proces učenja programiranja
je dug i naporan. Biti ćete bombardirani hrpom činjenica od kojih neće sve
biti toliko vazne da ih morate pamtiti napamet (ali moguće je da će vam
prije ili kasnije trebati). Jedan od tezih zadataka u cijelom tom procesu je i
sposobnost da znate prepoznate ono što je bitno od onog što nije.
Kad i ako jednog dana uspijete savladati neki programski jezik to će vam
sigurno predstavljati veliko zadovoljstvo.

\section*{Kako se uči programirati?}
\addcontentsline{toc}{section}{Kako se uči programirati?}

	"Naučiti programirati" je dug i naporan proces koji traje godinama i nikad ne
	prestaje. Nešto jednostavnije je "naučiti određeni programski jezik",
	ali nije dovoljno. Kad i ako naučite programirati uprogramskom jeziku Python,
	još niste niti blizu tome da se mozete smatrati dobrim programerom.

	Da bi postali dobar programer morate:
	\begin{itemize}
		\item puno vjezbati; rješavati razne programerske probleme, 
			pokušavati naći uvijek bolja rješenja,
		\item naučiti nekoliko programskih jezika (što više to bolje). 
			Kad savladate 3-4 programska jezika otkriti ćete da novi programski
			jezici nisu niti otprilike onoliko teški koliko je to bilo na
			početku.
		\item čitati knjige i proučavati programe iskusnih programera
		\item stvoriti način matematičko-logički način razmišljanja
			razmišljanja za rješavanje raznih problema
		\item \dots i za kraj; morate imati \emph{strast} za programiranjem. Ako vam se
			ikad desi da pišete neki program i naiđete naproblem kojeg ne znate
			riješiti. Promatrajte sebe; Kako reagirate u tom trenutku? Ako odmah
			odustajete, onda ovo nije za vas. Ako ste uporni i provodite sate, dane 
			pokušavajući naći izlaz ili grešku u nekom vašem programu
			onda je ovo pravi izazov za vas!
	\end{itemize}

\section*{Zašto Python?}
\addcontentsline{toc}{section}{Zašto Python}

	Ovih nekoliko redova je namijenjeno ljudima koji smatraju da je najbolji programski
	jezik za početi učiti programiranje Pascal, BASIC, Logo ili C.

	Dakle, Python zato jer\dots

	\begin{itemize}
		\item Python ima čistu sintaksu koja programera prisiljava da piše
			pregledne programe (za razliku od kriptične sintakse programskog jezika
			C)
		\item je Python zivi programski jezik kojeg programeri koriste i koji će
			se sve više koristiti (za razliku od BASIC-a i Logo-a)
		\item Pythonova stroga sintaktička pravila programera \emph{prisiljavaju} na
			neke vrlo dobre programerske navike (za razliku od BASIC-a)
		\item je Python objektno objektno orijentiran, pa se za učenje OO programiranja
			moze koristiti isti jezik, a ne učiti jedan za strukturalno, a jedan
			za OO programiranje (za razliku od BASIC-a i C-a)
		\item se u Pythonu, iako je zamišljen kao OO jezik bez problema mogu pisati 
			programi koji nemaju niti traga objektnom programiranju (za razliku od Jave)
		\item je Python besplatan i dostupan na skoro svim platformama koje vam padaju
			na pamet
	\end{itemize}
\chapter{O programiranju, programskim jezicima i programima}

\section{O programiranju}

	Program je skup naredbi pomoću kojih računalu pokušavamo objasniti kako
	da riješava neki problem.

	Dakle, program pišemo kad treba riješiti neki
	problem. Moze se raditi o nekom
	svakodnevnom problemu (napisati program koji će nam pamtiti brojeve telefona),
	nekom matematičkom ili logičkom zadatku (napisati program
	koji zbraja kvadrate brojeva od 1 do 100000)\dots Računalo samo, ne zna
	rješavati probleme, netko treba prije svega naš problem prevesti na 
	računalu razumljiv jezik.

	Program počinjemo pisati kad \emph{znamo} kako ćemo neki problem
	riješiti, ali nemam dovoljno vremena da to riješavamo na neki klasičan
	način. Na primjer, zamislite da vam u jednom trenutku treba podatak je li
	71283789238881999511\footnote{Broj je prost ako je djeljiv samo s 1 i sa samim sobom} prost broj ili nije. Prije nego li se uhvatite za glavu moramo
	se sloziti da postoji nekoliko vrsta problema:

	\begin{itemize}
		\item[\emph{(a)}] Problem kod kojeg znamo postupak kako doći do rješenja i
			mozemo jednostavno doći do tog riješenja.
		\item[\emph{(b)}] Problemi kod kojih znamo postupak kako doći do 
			riješenja, ali zbog
			nekog razloga nije tako jednostavno doći do riješenja.
		\item[\emph{(c)}] Problemi kod kojih (još) ne znamo postupak do riješenja.
	\end{itemize}

	Tako bi recimo sljedeći zadatak "Nađi peto slovo abecede" sigurno spadao u 
	probleme pod \emph{(a)}, i bilo bi besmisleno pisati program i tjerati računalo
	da rješava jedan tako jednostavan problem. Pitanje prostosti broja 
	71283789238881999511 spada u \emph{(b)}, jer \emph{znamo kako} ispitati je li neki
	broj prost ili nije. Prvo treba provjeriti djeljivost s $2$, pa s $3$, pa s
	$5$\dots Ako ispadne da naš broj nije djeljiv s ničim drugim osim s $1$ i
	sa samim sobom onda jest prost. 

	Postupak je smiješno jednostavan,\footnote{\dots u stvari i nije. Ima tu puno sitnica kako se taj postupak moze
	još više ubrzati.}
	ali pitanje je koliko dugo vremena nam treba
	da završimo. 

	Evo i primjer problema koji moze spadati u skupinu \emph{(c)}: "Pomoću
	brojeva 3, 3, 7, 7 i koristeći elementarne računske operacije trebate
	dobiti broj 24". Sad bi trebalo napisati program s kojim će računalo samo
	riješiti taj problem.

\section{O programskim jezicima}

	Znati postupak za riješavanje nekog problema nije garancija da ćemo taj
	problem i riješiti. 
	Čak i ako budem ikada imao dovoljno vremena da ispitam je li broj
	71283789238881999511 
	prost nitko mi ne moze garantirati da neću negdje
	pogriješiti u računu ili, ako radim s kalkulatorom -- da neću negdje
	pogrešno utipkati broj.
	Htio bih, zato, da taj posao računalo obavi umjesto mene.
	
	Problem je u tome što računalo ne govori moj jezik. 
	Mogao bih se do besvjesti truditi mom PC-ju objašnjavati kako sam ja 
	zamišljao da se riješi neki problem. 

	Računala razumiju samo jedan jezik -- \emph{mašinski jezik}. Naredbe tog
	jezika se sastoje od nula i jedinica (poznati binarni brojevni sustav), a pisanje
	programa u mašinskom jeziku je sve samo ne jednostavno.

	Zato koristimo neke druge programske jezike koji se onda prevode u 
	računalu razumljiv mašinski jezik. 
	Postoje stotine programskih jezika. Nabrojati
	ću samo neke od popularnijih: Python, C, C++, Java, Pascal, Lisp, Fortran, 
	Perl, Forth, PHP, JavaScript, BASIC, Smalltalk, Ada\dots Neki od tih jezika su
	\emph{interpreteri}, neki \emph{kompajleri}. Neki su \emph{objektno orijentirani},
	neki \emph{strukturalni}. Neki su \emph{komercijalni}, neki su \emph{open source},
	ali svi oni imaju neke slične osobine: svi oni koriste \emph{varijable},
	\emph{potprograme}, \emph{naredbe grananje}, razne \emph{kontrolne strukture} i
	tako dalje i tako blize\dots Kada dobro savladate jedan programski
	jezik neće vam predstavljati velik problem savladati mnoge druge.

	O svim ovim terminima će se naći ponešto u ovoj knjizici, ali dva
	su izuzetno bitna pa ću ih objasniti odmah:

\subsection{Interpreteri i kompajleri}

	Već spomenuh da računalo razumije samo mašinski jezik. Kako onda
	računalo moze shvatiti naredbe bilo kojeg od gore nabrojanih jezika?
	Odgovor je jednostavan, niz naredbi u nekom programskom jeziku se \emph{prevode} na
	mašinski jezik. 

	Zamislite da vam netko tko govori \emph{Swahili}\footnote{\emph{Swahili}
	je jedan od jezika koji se govore u Zanzibaru. Pretpostavka je da prosječni
	čitatelj ove knjizice ne razumije Swahili} Zeli reći sljedeće:
	\emph{"Otvori knjigu X na stranici 137. pročitaj prvu riječ u petom retku.
	Ako ta riječ počinje suglasnikom odi u kuhinju i skuhaj ručak, a ako
	počinje samoglasnikom -- idi se prošeći i pusti me na miru"}.
	Govornik Swahilija se nalazi u situaciji u kojoj se nalazi programer, a vi glumite
	računalo koje pokušava shvatiti šta vam ovaj ima za kazati. 
	
	Postoji, naravno i prevoditelj. Prevoditelj moze prevoditi simultano ili
	prevesti cijelu poruku odjednom, a vi ćete onda to pročitati i slijediti
	upute. 

	Ako se prevodi simultano cijela stvar izgleda otprilike ovako ("Sw" je govornik
	Swahilija, "V" ste vi, "V" je prevoditelj):

	\begin{itemize}
		\item[\textbf{Sw:}] ".... .... .... ..... ..... ... .... ..... .... .... ....."\footnote{Zamislite da je ovdje neka rečenica na jeziku Swahili}
		\item[\textbf{P:}] "Sw je rekao da trebaš otvoriti knjigu X"
		\item[\textbf{V:}] (otvarate knjigu X)
		\item[\textbf{Sw:}] "..... .... .... .... 137. ....."
		\item[\textbf{P:}] "Sw kaze da otvoriš stranicu 137."
		\item[\textbf{V:}] (otvarate knjigu 137.)
		\item[\textbf{Sw:}] "..... . .... ....  ....."
		\item[\textbf{P:}] "Sw kaze da nađeš prvu riječ u petom retku."
		\item[\textbf{V:}] (trazite prvu riječ u petom retku)
		\item[\textbf{Sw:}] "... ....  ....."
		\item[\textbf{P:}] "Sw kaze da pogledaš prvo slovo te riječi"
		\item[\textbf{V:}] (OK, prvo slovo te riječi je "r")
		\item[\textbf{Sw:}] "...  ... ... ....  ....."
		\item[\textbf{P:}] "Sw kaze; odi skuhati ručak, ako je to slovo samoglasnik"
		\item[\textbf{V:}] (nije samoglasnik)
		\item[\textbf{Sw:}] "...  ... ... ......  .. ...!"
		\item[\textbf{P:}] "Sw kaze da je najbolje da odeš u šetnju"
		\item[\textbf{V:}] (odlazite u šetnju\dots Bilo je i vrijeme da vas puste na miru!)
	\end{itemize}

	Ovako funkcioniraju programski jezici \emph{interpreteri} -- čitaju naredbu
	programa kojeg ste vi napisali, prevedu ju na mašinski jezik, a računalo to
	onda izvršava, nakon toga pročitaju sljedeću naredbu, prevedu,
	računalo izvršava, pročitaju, prevedu, računalo izvršava\dots

	Drugi način je da prevodioc cijelu poruku posluša do kraja, prevede ju
	na vama razumljiv jezik, a vi to onda idete izvršavati. Programski jezici
	kompajleri rade upravo to -- pročitaju cijeli vaš program, prevedu na
	mašinski jezik, a računalo onda izvrši program kojeg sada
	ima u obliku kojeg razumije.

	Da bi izvršili program u programskom jeziku koji se interpretira morate uvijek
	imati interpreter (prevodilac) već instaliran na svom računalu. Kod
	izvršavanja kompajliranog programa dovoljno je da program jednom kompajlirate,
	odmah dobijete program u obliku kojeg računalo razumije, a moći ćete ga
	pokrenuti na nekom drugom računalu (na kojem taj programski jezik "nije
	instaliran"). 

	Ako na računalu imate program koji je prethodno kompajliran nećete moći
	vidjeti kako taj program izgleda, jer on je u memoriji sačuvan u mašinskom
	jeziku (a kojeg vjerojatno ne razumijete, kao Swahili uostalom, jelte 
	:-)\footnote{Ako lutate bespućima interneta onda znate šta ova
	dvotočka-crtica-zagrada znače, ako vam je surfanje nepoznanica, onda
	pogledajte taj niz znakova tako da ukosite glavu ulijevo. Uz malo truda trebali
	biste vidjeti nasmijano lice}. 
	S druge
	strane, ako u memoriji imate program u nekom od interpretiranih programskih jezika,
	moći ćete slobodno pogledati kako taj program izgleda (editirati kao tekst
	datoteku). Moći ćete ga pokrenuti ako i samo ako imate instaliran i
	interpreter-prevodioc za upravo taj programski jezik.

	I, još samo ovo. Razlika između programskih jezika koji se interpretiraju
	i onih koji se kompajliraju nije točno određena. Postoje, naime programski
	jezici koji se mogu pokretati i interpreterom, a mogu se i kompajlirati. Još
	jedna komplikacija su programski jezici koji se kompajliraju, ali ne u mašinski
	jezik nego u nešto što razumije samo poseban program koji to onda
	interpretira\dots Zaboravite, prezivjeti ćete čak i ako ne shvatite sve
	ove tehnikalije.

\section{O programima}
	
	Program je niz naredbi koji opisuje kako se moze riješiti neki problem. Taj
	niz naredbi mora \emph{točno} i \underline{vrlo detaljno} opisati taj
	postupak. Zamislite da nekom stroju morate opisati kako se jede za ručkom.
	Izgleda vrlo jednostavno, ipak -- razmislite malo koliko tu ima detalja. Kao prvo
	morate sjesti za stol, ali i to ne mozete bez da prethodno ne odmaknete stolicu
	od stola. Kad sjednete, stolicu treba opet pribliziti stolu. OK, idemo na juhu:
	imate zlicu, grabite u tanjur i stavljate u usta. Opet! Jeste li sigurno da je
	juha uopće na tanjuru, ako nije trebate opisati kako staviti juhu u tanjur.
	Zlicom, jel? Hm, jeste li sigurni da ste točno opisali kako se zlicom
	stavlja juha, jer ako zlicu ne drzite pod pravim kutom juha će
	se proliti iz zlice. A još nismo niti počeli s jelom. Zamislite
	koliko komplicirano moze biti ako morate objašnjavati neki specijalitet od
	ribe kao drugo jelo.
	Tko bi imao volje stroju objašnjavati kako se
	čiste riblje kosti?

	\subsection{Primjeri u raznim programskim jezicima}

	OK, uspio sam vas obeshrabriti\dots Molim? Nisam? Hhhmmm\dots Ajmo onda probati
	ovako: slijedi nekoliko programa u različitim programskim jezicima koji rade
	jednu te istu stvar:

	\source{Pascal:}{
		program trlababalan;\\
		integer x;\\
		begin\\
		\hspace*{10mm}x := 5;\\
		\hspace*{10mm}if x $=$ 10 then\\
		\hspace*{20mm}writeln( "x je jednako 10" );\\
		\hspace*{10mm}else\\
		\hspace*{20mm}writeln( "x nije jednako 10" );\\
		end.
	}

	Ili, recimo ovako:

	\source{Perl:}{
		x = 5;\\
		if( x == 10 ) \{\\
		\hspace*{10mm}print "x je jednako 10$\setminus$n";\\
		\}\\
		else \{\\
		\hspace*{10mm}print "x nije jednako 10$\setminus$n";\\
		\}
	}

	Često je u istom programskom jeziku moguće isti problem riješiti na
	različite načine, a programski jezik Perl je posebno poznat po tome. 
	Ovaj gornji programčić iskusan programer bi napisao ovako:

	\source{Perl:}{
		x = 5;\\
		print x $==$ 10 ? "x je manji od 10$\setminus$n" : "x je veći od 10$\setminus$n";
		}

	U Pythonu bi to izgledalo:

	\source{Python:}{
		x = 5;\\
		if x $<$ 10:\\
		\hspace*{10mm}print "x je manje od 10"\\
		else:\\
		\hspace*{10mm}print "x je veće od 10"
	}
	
	U programskom jeziku Java:

	\source{Java:}{
		public class trlababalan \{\\
		\hspace*{10mm}public static void main( String[] args ) \{\\
		\hspace*{20mm}int x = 5;\\
		\hspace*{20mm}if( x == 10 ) \{\\
		\hspace*{30mm}System.out.println( "x je jednako 10" );\\
		\hspace*{20mm}\}\\
		\hspace*{20mm}else \{\\
		\hspace*{30mm}System.out.println( "x nije jednako 10" );\\
		\hspace*{20mm}\}\\
		\hspace*{10mm}\}\\
		\}
	}
	
	U programskomjeziku HP48 kalkulatora:

	\source{hp48:}{
		$<<$ 5 'x' STO x 10 IF == THEN "x je jednako 10" ELSE "x nije jednako 10" END
		MSGBOX $>>$
	}
	

	Vjerujem da ste uočili neke sličnosti u ovim kratkim programčićima.
	Nemojte dopustiti vas ono što ne razumijete obeshrabri. Toga će uvijek
	biti -- na veliku zalost onih koji brzo odustaju, a na zadovoljstvo onima koji
	je "nepoznato" samo još jedan izazov.

\section{O programskom jeziku \emph{Python}}

	Python je programski jezik kojeg je stvorio/kreirao/dizajnirao/smislio Guido Van
	Rossum. Za razliku od komercijalnih programskih jezika, Guido je odlučio da
	njegov programski jezik mora biti svima dostupan. I, ne samo da će biti svima
	dostupan nego nego programeri sami mogu mijenjati dotični jezik prema svojim
	potrebama. S vremenom se stvorila grupa ljudi koji su počeli pisati programe u
	tom jeziku, a ako im se sviđala neka sitnica iz nekog drugog jezika jednostavno
	bi i nju dodali Pythonu.

	Python je interpretirani (ali ne baš u smislu "simultanog prevođenja") i
	objektno orijentirani (ali bez problema mogu se pisati strukturalni programi)
	programski jezik. Zbog svoje vrlo čiste i stroge sintakse je vrlo pogodan da
	bude "prvi programski jezik" ne-programerima. Često je ulogu programskog jezika
	za učenje najčešće imao BASIC, Pascal, Logo i razni drugi jezici. 
	Well\dots Neću u detalje, ali Python je bolji od njih :-)

	Python je i besplatan, a moze se pokrenuti na skoro svakom računalu koje
	vam padne na pamet (osim onih \emph{stvarno} prastarih). Ukoliko zelite
	nastaviti čitanje ove knjige pozeljno bi bilo da s internet adrese
	http://www.python.org/ skinete Python interpreter i instalirate ga na svom
	računalu.

\subsection{"Python kao Monty Python?"}

	\emph{"Python kao Monty Python?"} -- Da, \emph{Python} kao \emph{Monty Python}. Da
	citiram "Python tutorial" (za one koji razumiju engleski):

	\begin{quote}
	By the way, the language is named after the BBC show ``Monty Python's
	Flying Circus'' and has nothing to do with nasty reptiles.  Making
	references to Monty Python skits in documentation is not only allowed,
	it is encouraged!
	\end{quote}

\chapter{Uvod u programiranje}

\section{Sto je danas lijep i sunčan dan}

Kad savladate nekoliko programskih jezika uočiti ćete
nekoliko pravilnosti koje se pojavljuju kod svih njih. Svi oni
koriste varijable, potprograme, dobar dio njih koristi objekte. Ne
samo to, nego u svim priručnicima za programiranje prvi program
ima naziv "Hello world"\footnote{eng.  Zdravo svijete}.

Budući da autor ovog spisa nije nimalo revolucionaran tip, i
mi ćemo započeti Zdravo-svijetom\footnote{\hspace*{2mm}"Hello
world"}.  Daklem. "Hello world" u Pythonu izgleda ovako:

	\sourcee{
		\wrd{print} "Hello world"
	}

\textbf{Komentar:} Druga linija je ono što je bitno u ovom
programu. Ona kaze prevoditelju-interpreteru da ispiše na
monitor niz znakova "Hello world" (bez navodnika, navodnici samo
sluze da se zna gdje počinje i gdje završava ono što
zelimo ispisati).

\subsection{Snimanje programa}

Jedan od gornjih "zdravo-svijete" programa napišite u nekom
tekst editoru (ako radie na Windowsima onda to moze biti NotePad,
ako ste na Linuxu vi, joe, emacs, ako na Macintoshu\dots)

\dots

\subsection{Pokretanje programa}

Na windowsima; nakon što ste program snimili kao datoteku s ekstenzijom
.py, nađite gdje je ikona s tom datotekom i dvaput kliknite na nju.

Na linuxu; u komandnoj liniji treba napisati: "python $<$ime\_datoteke$>$".
Naravno, $<$ime\_datoteke$>$ zamijenite s imenom kojeg ste upisali pri
snimanju.

\dots

\subsection{Python kao kalkulator}

Pokrenete li\footnote{Na windowsima tako da kliknete na ikonu, a
na linuxu tako da u prompt-u upišete "python"} program kojeg
dobijete kad instalirate python dobiti cete nesto kao:

\sourcee{
Python 2.2 (\#1, Jan  9 2002, 02:47:26) \\
$[$GCC 2.95.2 19991024 (release)$]$ on darwin \\
Type "help", "copyright", "credits" or "license" for more information.  \\
$>>>$ 
}

Ovdje sada  mozete upisivati naredbe programskog jezika python i
vidjeti rezultat:

\sourcee{
Python 2.2 (\#1, Jan  9 2002, 02:47:26) \\
$[$GCC 2.95.2 19991024 (release)$]$ on darwin \\
Type "help", "copyright", "credits" or "license" for more information.  \\
$>>>$ \var{print 2+3} \\
5 \\
$>>>$ \var{d=2} \\
$>>>$ \var{print d+4} \\
6 \\
$>>>$ 
}

Crvenom bojom su označene naredbe koje sam u gornjem slučaju
pisao ja, a ostatak je generiralo računalo (odnosno python
interpreter).

Ovaj način pokretanja python interpretera se zove interaktivni. U
principu sve programe u ovom poglavlju mozete pokretati u interaktivnom
načinu jednostavno tako da upisujete liniju po liniju programa.

\section{Komentari u programu}

	Gornji program je sasvim mogao izgledati i ovako:

	\sourcee{
		\com{\# "Hello world" (c) 2001 by "Trlababalan" d.d.}\\
		\wrd{print} "Hello world"
	}

	\textbf{Komentar:}
	Prva linija, ona koja započinje sa znakom \# je \emph{komentar}.
	Kad interpreter dođe do ovog znaka on jednostavno preskoči sve
	što nađe nakon tog znaka pa sve do kraja reda.

	Nije loše stavljati komentare u vaše programe. Oni koji će slijediti
	će u sebi imati puno komentara. 

	Mogli se napisati i ovako:

	\sourcee{
		\com{\# Došao škrtac u restoran pojesti juhu, a u jednom trenutku je\\
		\# morao otići na WC. Naravno, uhvatila ga panika da mu netko\\
		\# ne bi pojeo juhu za to vrijeme, pa je ispod juhe stavio papirić\\
		\# na kojeg je napisao "Pljunuo sam u juhu" -- nadajući se da \\
		\# tako nikome neće pasti na pamet niti primirisati njegovoj tanjuru\\
		\# \\
		\# Kad se vratio, otkrio je da je na dnu njegovog papirića netko\\
		\# nadopisao "\dots i ja"}\\
		\\
		\wrd{print} "Hello world"
	}

	Moze i ovo:

	\sourcee{
		\wrd{print} "Hello world" 
		\com{\# ovaj program ispisuje na monitor: "Hello world"}
	}

	U stvari gornja četiri slučaja predstavljaju jedan te isti program.

	Postoji još jedan način kako se mogu pisati komentari u programu:

	\sourcee{
		\wrd{print} "Hello world"\\
		\com{""" Ovo je samo \\
		komentar """}
	}

	\vspace{3mm}
	Komentar počinje s nizom od tri navodnika i završava s tri navodnika, a
	moze se protezati preko više linija.

\section{Varijable}	

	Promotrite sljedeći primjer:



\sourcee{
\var{x} = 5 \\
\textcolor{green}{\# od sada (do daljnjega) varijabla x ima vrijednost 5}\\
\textcolor{blue}{print} "x je", \var{x} \\
\textcolor{green}{\# ispisuje prvo poruku "x je", a nakon nje vrijednost varijable x}\\
\var{y} = 10\\
\textcolor{blue}{print} "...a y je", \var{y}
}

	Nakon što ga snimite i pokrenete ispis će biti:

\sourcee{
x je sada 5\\
y ima vrijednost 10
}

	\textbf{Objašnjenje:} u prevom redu programa, gdje piše \verb|x = 5| smo
	definirali varijablu \verb|x|. \verb|x| od sada ima vrijednost 5. To ne znači
	da se ta vrijednost neće promijeniti u budu�nosti. 
	Od sad, svaki put kad pokušamo ispisati vrijednost x-a dobiti ćemo broj 5.

	\textbf{Napomena:} znak "=" u prvom redu programa ne treba shvatiti kao
	matematičku tvrdnju da je \verb+x+ isto što i 5, odnosno da \verb+x+ ima
	vrijednost 5. Redak "\verb+x = 5+" treba shvatiti kao "Od sada će \verb+x+
	imati rijednost 5" ili "Neka \verb+x+ ima vrijednost 5". Dakle varijabli
	\verb+x+ \underline{pridruzujemo vrijednost} 5.

	U trenutku kad smo u programu napisali \verb|x = 5| kazemo da smo varijablu
	\emph{inicijalizirali}, dakle odredili smo joj neku početnu vrijednost. Ako
	pokušamo ispisati neku neinicijaliziranu varijablu interpreter će nam ispisati
	poruku o grešci:

	\sourcee{
	\textcolor{blue}{print} \var{x} \textcolor{green}{\# ...ali varijabla x nije prethodno inicijalizirana!}
	}

	Rezultat je:

	\sourcee{
Traceback (most recent call last):
\\
File "primjer.py", line 1, in ?
\\
print x
\\
NameError: name 'x' is not defined
	}

	Obratite paznju na to da vam se točno kaze gdje je greška u programu
	("line 1"). To nam sad i nije od neke koristi jer naš program ima samo jednu
	liniju, ali kad se stvari zakompliciraju onda ćemo takve informacije jako puno
	koristiti!
	
	Pogledajmo sljedeći primjer:

	\sourcee{
	\var{x} = 5 \textcolor{green}{\# x ima sada vrijednost 5}\\
	\textcolor{blue}{print} "x je", \var{x} \textcolor{green}{\# ispisujemo vrijednost od x}\\
	\var{y} = \var{x} \textcolor{green}{\# sada y poprima vrijednost varijable x}\\
	\textcolor{blue}{print} "y je", \var{y} \textcolor{green}{\# ispisujemo vrijednost od y}\\
	\var{x} = 10 \textcolor{green}{\# x je sad 10}\\
	\textcolor{blue}{print} "x je sad ", \var{x} \textcolor{green}{\# ispisujemo x}\\
	\textcolor{blue}{print} "y je sad", \var{y} \textcolor{green}{\# ispisujemo y}
	}

	Kad pokrenemo program ispis je:

	\sourcee{
	x je 5\\
	y je 5\\
	x je sad 10\\
	y je sad 5
	}

	Prvo je \verb|x| bio 5, tada smo varijabli \verb|y| dodijelili vrijednost varijable
	\verb|x|, dakle 5. Tada smo \verb|x| promijenili da sadrzi 10, ali \verb|y| je
	ostao 5 otprije.

\section{Naredbe}

	Program se sastoji od niza naredbi. Svaka naredba računalu govori šta
	treba činiti. Razmotriti ćemo još jednom jedan primjer programa
	pisanog u Pythonu:
	
	\sourcee{
		\var{x} = 5;\\
		\wrd{if} \var{x} $<$ 10:\\
		\hspace*{10mm}\textcolor{blue}{print} "x je manje od 10"\\
		\wrd{else}:\\
		\hspace*{10mm}\textcolor{blue}{print} "x je veće od 10"
	}

	Python je programski jezik koji zahtijeva da se svaka naredba piše u odvojenom
	retku. Tako, svaki red predstavlja jednu naredbu, u prvom redu naredba "=" kaze
	računalu da varijabli \verb|x| pridruzi vrijednost 5. Nakon toga jedan
	\verb+if+--\emph{uvijet}\footnote{Više riječi o uvjetima kasnije\dots} 
	(naredba \verb+if+)
	provjerava je
	li x manji od 10, ako jest onda ispisuje s \verb+print+ poruku "x je manji od 10", a ako nije onda
	poruku "x je veće od 10". Program će, naravno, ispisati prvu poruku

	Naredbu \verb+print+ smo upoznali kod pisanja "Hello world" programa -- ona ispisuje na monitor ono što se nalazi iza nje.
	O toj naredbi ćemo još puno pričati kasnije.

	U nekim programskim jezicima mogu se pisati
	programi koji u jednoj liniji imaju više naredbi. Pogledajte na primjer kraći
	od dva programa pisana u programskom jeziku Perl u odjeljku 1.3.1. -- ono za
	što nam u Pythonu treba 5 redova mozemo u Perlu napisati dva reda (mogo bi
	se i u jednom).
	Mozda vam se to moze činiti praktičnim, ali takvi programski jezici
	rezultiraju s programima koji su vrlo teško shvatljivi. U takvom programu je
	vrlo teško naći grešku.

	Pogledajmo još jednom taj program u programskom jeziku Perl:

	\source{Perl:}{
		x = 5;\\
		print x $==$ 10 ? "x je manji od 10$\setminus$n" : "x je veći od 10$\setminus$n";
		}

	Ako znate imalo engleskog onda bi vam prvi primjer pisan u Pythonu trebao biti
	dosta jasan, čak i ako ne znate programirati. Pogledajte sad ovaj drugi
	primjer, on je toliko zgusnut i kriptičan da je programeru-početniku vrlo
	teško shvatiti šta je ovdje šta. Zamislite da sad imate program koji
	ima 1000 ovakvih linija, a znate da se negdje među njima nalazi jedna
	greška?!

	\textbf{Zapamtite:} Uvijek se potrudite pisati pregledne programe. Svaku komandu
	pišite u novu liniju. Svaki slozeniji dio programa popratite s komentarima!
	Desiti će vam se da morate prepraviti program koji ste nekad davno pisali. To
	će biti tim lakše što ste tada taj program preglednije pisali!
	
\section{Imena}

	U zadnja dva primjera u našim programima koristili smo se varijablama. Te
	varijable smo nazvali \verb+x+ i \verb+y+. Naravno, mogli smo ih nazvati i
	drukčije. Probati ćemo nešto naučiti iz sljedećeg primjera:

	\sourcee{
\var{pi} = 3.1415926 \\
\var{v} = 5 \\
\var{V} = 10 \\
\var{nova}\_varijabla = 12345 \\
\var{var1} = 12 \\
\var{var2} = 13 \\
\textcolor{blue}{print} "pi je",\var{pi} \\
\textcolor{blue}{print} "varijabla v ima vrijesnost",\var{v} \\
\textcolor{blue}{print} "varijabla V (veliko slovo!) ima vrijesnost",\var{V} \\
\textcolor{blue}{print} "jos jedna varijabla:", \var{nova\_varijabla} \\
\textcolor{blue}{print} "var1 je", \var{var1}, ", a var2 je", \var{var2} 
	}

	Prvih 6 linija programa varijablama pridruzuje neke vrijednosti. Imena tih
	varijabli su 
	\verb+pi+,
	\verb+v+,
	\verb+V+,
	\verb+nova_varijabla+,
	\verb+var1+ i 
	\verb+var2+.

	\textbf{Vazno:} Uočite da Python razlikuje velika i mala
	slova: varijabli \verb+v+ smo
	pridruzili vrijednost 5, a varijabli \verb+V+ (veliko slovo V) vrijednost 10.
	Kasnije u programu se njih tretira kao dvije različite varijable!

	Kad daj program upišete i pokrenete dobiti ćete:

	\sourcee{
	pi je 3.1415926\\
	varijabla v ima vrijesnost 5\\
	varijabla V (veliko slovo!) ima vrijesnost 5\\
	jos jedna varijabla: 12345\\
	var1 je 12 , a var2 je 13
	}

	\textbf{Zapamtite:} Imena varijabli se mogu sastojati kombinacije velikih i malih
	slova, decimalnih znamenaka
	i znaka "\_"\footnote{Engleski: underscore}. Decimalna znamenka se ne smije
	nalaziti na prvom mjestu imena varijable. Razlikuju se velika i mala slova; dakle
	\verb+kikiriki = 5+ i \verb+Kikiriki = 10+ su dvije različite varijable s
	različitim vrijednostima! 

	Ima još jedno pravilo kojeg se morate drzati kod određivanja imena
	varijable; ime varijable \emph{ne smije biti ključna riječ!}

\section{Ključne riječi}

	Ključne riječi su "riječi" ili nizovi znakova koje su rezervirane za
	korištenje u programskom jeziku i ne smiju se koristiti za imenovanje varijabli
	ili nekih drugih dijelova programa koje određuje programer. Već smo se
	sreli s  naredbom \verb+print+. \verb+print+ je ključna riječ i ne smije
	biti korištena kao ime varijable. Mozemo našu varijablu imenovati npr
	\verb+var_print+ ili \verb+_print+ ili \verb+print1+, ali nikako \verb+print+!

	Svaki programski jezik ima određeni broj ključnih riječi. Ključne
	riječi u programskom jeziku Python su:

\begin{center}
\begin{tabular}{llll}
and & del & for & is \\
raise & assert & elif & from\\
lambda & return & break & else\\
global & not & try & class\\
except & if & or & while\\
continue & exec & import & pass\\
def & finally & in & print
\end{tabular}
\end{center}

	Radi praktičnosti, u programima koji se nalaze u ovoj knjizici ključne
	riječi su prikazane u plavoj boji.

\section{Komuniciranje s okolinom}

	Računalo nema previše smisla ako ne komunicira s okolinom. Pod
	\emph{komunikacija s okolinom} podradzumijeva se komunikacija s korisnikom ili
	komunikacija s nekim drugim uređajem vezanim uz računalo. Računalo
	komunicira s korisnikom na različite načine; npr. klikanjem po ikonama,
	upisivanjem teksta u nekakav formular, diktiranjem u mikrofon spojen 
	na računalo, i tako dalje i tako blize.

	Naučiti ćemo sada jedan jednostavan način kako mozemo naš program u
	pythonu natjerati da i komunicira s nama.

	\sourcee{
		\var{name} = \textcolor{blue}{raw\_input}( "Upišite svoje ime:" )\\
		\textcolor{blue}{print} "Dobar dan", \var{name}
	}

	Kad pokrenete program računalo će ispisati poruku "Upišite svoje ime:" i
	traziti od vas da upisujete ime. Rezultat moze biti:

	\sourcee{
		Upišite svoje ime:\textcolor{blue}{Aleksandar Makedonski}\\
		Dobar dan  Aleksandar Makedonski
	}

	Plavom bojom je označen tekst kojeg upisuje sam korisnik.

	Pomoću naredbe \verb+raw_input+ sad mozemo bilo kojoj varijabli 
	pridruziti vrijednost broja kojeg će korisnik utipkati tek u trenutku kad 
	se program pokrene. \verb"raw_input" je sličan, ali s njime mozemo toj
	varijabli pridruziti i vrijednost stringa (a ne isključivo brojevnu
	vrijednost).

	Porobati ćemo sada napisati program koji trazi od korisnika da upiše 
	tri broja, a nakon toga ispisuje njihovu aritmetičku sredinu (prosijek).

	\sourcee{
		\var{n1} = \textcolor{blue}{input}( "Upiši prvi broj:" )\\
		\var{n2} = \textcolor{blue}{input}( "Upiši drugi broj:" )\\
		\var{n3} = \textcolor{blue}{input}( "Upiši treći broj:" )\\
		\var{avg} = ( \var{n1} + \var{n2} + \var{n3} ) / 3\\
		\textcolor{blue}{print} "Aritmetička sredina je ", \var{avg}
	}

	\textbf{Komentar:} Program prvo u varijable \verb+n1+, \verb+n2+ i \verb+n2+
	smješta ono što će korisnik sam upisati kad ga se upita da upiše broj.
	Varijabla \verb"avg" (eng. "average" = "prosjek") zatim prime vrijednost
	aritmetičke sredine brojeva 
	\verb"n1",
	\verb"n2" i 
	\verb"n3".
	Na kraju se samo ispisuje vrijednost od \verb"avg".

	Evo još dvije varijante istog programa:

	\sourcee{
		\var{n1} = \textcolor{blue}{input}( "Upiši prvi broj:" )\\
		\var{n2} = \textcolor{blue}{input}( "Upiši drugi broj:" )\\
		\var{n3} = \textcolor{blue}{input}( "Upiši treći broj:" )\\
		\var{avg} = ( float( \var{n1} ) + float( \var{n2} ) + float( \var{n3} ) ) / 3\\
		\textcolor{blue}{print} "Aritmetička sredina je ", \var{avg}
	}

	Ili još kraće:

	\sourcee{
		\var{n1} = float( \textcolor{blue}{input}( "Upiši prvi broj:" ) )\\
		\var{n2} = float( \textcolor{blue}{input}( "Upiši drugi broj:" ) )\\
		\var{n3} = float( \textcolor{blue}{input}( "Upiši treći broj:" ) )\\
		\textcolor{blue}{print} "Aritmetička sredina je ", ( \var{n1} + \var{n2} + \var{n3} ) / 3
	}

	\textbf{Napomena:} \verb+float()+ nije jedini način kako se string moze
	pretvoriti u broj. U sljedećem poglavlju će biti objašnjeno zašto se
	ovdje koristi baš taj.

\chapter{Izrazi}

\section{Aritmetički izrazi}

Aritmetički izrazi su matematički izrazi s kakvima se računa u
osnovnoj školi. Najčešće se sastoje od brojeva ili varijabli
koje imaju brojčanu vrijednost i matematičkih operacija. Primjer
aritmetičkog izraza moze biti: 
$2+4-2$, 
$\displaystyle \frac{3+6\cdot 5}{7-4}$,
ili 
$\displaystyle \frac{s_2-s_1}{t_2-t_1}$. 

Kad nam negdje u programu zatreba
aritmetički izraz zapisujemo ga na sličan način kako bismo ga napisali u
biljeznici s nekoliko sitnih razlika;

\begin{itemize}
	\item Mnozenje zapisujemo pomoću znaka *, a ne $\cdot$.
	\item Dijeljenje zapisujemo pomoću znaka / umjesto :
	\item Razlomke zapisujemo pomoću operacije dijeljenja.
	\item Potencije zapisujemo pomoću "**". Dakle $3^2$ bi zapisalo kao
		3**2
	\item Ostatak\footnote{"Modulo"} pri dijeljenju dobijemo pomoću
		operacije \%.
\end{itemize}

Izraz 
$\displaystyle \frac{3+6\cdot 5}{7-4}$
bi napisali: \verb+(3+6*5)/(7-4)+. U ovom
slučaju brojnik i nazivnik treba staviti unutar zagrada jer bi u
slučaju da je izraz \verb+3+6*5/7-4+ kompjuter pokušao prvo
izračunati \verb+6*5/7+, python naime izraze računa pazeći na
prednost računskih operacija (npr. mnozenje i dijeljenje imaju
prednost pred zbrajanjem i oduzimanjem).

\textbf{Vazno:}
Ima još jedna stvar na koju treba pripaziti pri pisanju algebarskih
izraza; ukoliko su brojevi s kojima računamo cjelobrojni onda će (u
programskom jeziku python) i rezultat biti cjelobrojan. Dakle, ako probate
izračunati \verb+13/4+ dobiti ćete \verb+3+, a ne \verb+3.25+! To se
moze riješiti tako da barem jedan od brojeva definiramo kao realan, a za
to je dovoljno dodati mu decimalnu točku na kraju. Da bi dobili točan
rezultat dijeljenje 13 podijeljeno s 4 trebali bi dakle napisati
\verb+13/4+.

Zadaci:

\section{Logički izrazi}

Slično kao aritmetički izrazi logički izrazi se sastoje od
operacija i članova izraza nad kojima ze izvršavaju te operacije.
Kod aritmetičkih izraza članovi su brojevi ili varijable s
brojevnom vrijednošću, a članovi logičkih izraza mogu biti
\emph{sudovi} ili čak drugi aritmetički izrazi.

\emph{Sud} je tvrdnja koja moze biti istinita ili lazna. Primjer
suda je \emph{"Zemlja kruzi oko Mjeseca"} ili \emph{"Postoji
beskonačno mnogo prirodnih brojeva"}.  Svaki sud mora imati svoju
istinosnu vrijednost koja moze biti \emph{"istina"} ili \emph{"laz"}.
Ukoliko za neku tvrdnju ne mozemo sa sigurnošću kazati je li
istinita ili lazna tada to nije sud. Na primjer \emph{"Zemlja
kruzi oko Mjeseca"} jest sud zato što ima istinosnu vrijednost
\emph{"laz"}, kao i \emph{"Postoji beskonačno mnogo prirodnih
brojeva"} čija je istinita vrijednost \emph{"istina"}.  Tvrdnja
\emph{"Frank Sinatra pjeva bolje od Tine Turner"} nije sud jer je
nemoguće odrediti istinitost te tvrdnje budući da je ona čisto
subjektivne prirode (nekome se više svidja Sinatra, a nekome Tina
Turner). Isto tako nije sud "Učini to!" ili "Mozda ću jednog
dana naučiti programirati".

Umjesto "istina" ili "laz" se često koriste velika slova "T" (od
engleskog "true" = "istina") odnosno "F" (eng. "false" = "laz").

Sud mozemo zapisati i matematičkim simbolima: $1<2$ je sud u
kojemu se tvrti da je jedan manji od 2, a istinosna vrijednost tog
suda je T (= istina). Matematički sudovi kojima se koristimo u
programiranju najčešće se koriste za opisivanje odnosa izmedju
brojeva. Pri tome se koristimo sljedećim simbolima iz sljedeće tablice;
u prvom stupcu se nalazi simbol kako bismo za zapisali u biljeznicu
ili na ploču, u drugom stupcu zapisa tog istog simbola u programu,
a u trećem kako čitamo taj simbol:

\begin{tabular}{lll}
	$=$ & $==$ & je jednako \\
	$\neq$ & $!=$ & nije jednako, je različito \\
	$<$ & $<$ & je manje od \\
	$\leq$ & $<=$ & je manje ili jednako \\
	$>$ & $>$ & je veće od \\
	$\geq$ & $>=$ & je veće ili jednako
\end{tabular}

Probajmo sada utvrditi istinosnu vrijednost nekih matematičkih izraza:

\begin{tabular}{lll}
	$12<12.01$ & \verb+12<12.01+ & T\\
	$1+2\leq 5$ & \verb"1+2<=5" & T\\
	$10-3\geq 6+1$ & \verb"10-3>=6+1" & T \\
	$10-2\geq 6+1$ & \verb"10-2>=6+1" & F \\
	$5\neq 5$ & \verb"5!=5" & F
\end{tabular}

Gornji primjeri su primjeri \emph{jednostavnih sudova}. Slozeni
sudovi su sudovi koji se dobijaju od jednostavnih sudova i logičkih
operacija \emph{and}, \emph{or} i \emph{not}.

\subsection{Logička operacija \emph{and}}

Promotrimo rečenicu \emph{"Ako je lijepo vrijeme
idemo na izlet."}. O čemu ovisi oćete li
otići na izlet? Ovisi o tome je li lijepo vrijeme, dakle ovisi o
istinosnoj vrijednosti suda \emph{"Lijepo je vrijeme."}. Ako je
taj sud istinit (T) otići ćete na izlet, a ako nije (F) -- ništa
od izleta.

Malo ćemo stvar zakomplicirati ako osim lijepog vremena vaš izleta
ovisi o još nečemu; \emph{"Ako je lijepo vrijeme i nemam drugih obaveza
ići ću na izlet"}.
Sad vaš izlet ovisi o istinitosti suda \emph{"Lijepo je vrijeme i nemam obaveza"}, a on je istinit
kad su istovremeno istinita sljedeća dva suda:

\begin{itemize}
	\item[\emph{(a)}] \emph{"Lijepo je vrijeme."}
	\item[\emph{(b)}] \emph{"Nemam obaveza."}
\end{itemize}

Dakle, treba vrijediti da je istinito i \emph{(a)} i \emph{(b)},
jer ako je bilo koje od ta dva laz onda je i tvrdnja \emph{"Lijepo
je vrijeme i nemam obaveza"} lazna.

Ukoliko imamo dva suda koja ćemo ovdje označiti s $A$, odnosno
$B$ onda ćemo takvu kombinaciju zapisivati s $A and B$. Suda $A
and B$ je \emph{slozeni sud} koji se sastoji od jednostavnijih
sudova $A$ i $B$. Istinitost suda $A and B$ ovisi o istinitosti
sudova $A$ i $B$; tek ako su oba istinita onda je i $A and B$
istinit. To se moze prikazati pomoću sljedeće tablice:

\begin{tabular}{ll|l}
	$A$ & $B$ & $A and B$ \\
	\hline
	istina & istina & istina \\
	istina & laz & laz \\
	laz & istina & laz \\
	laz & laz & laz 
\end{tabular}

Garnju tablicu zovemo \emph{tablica istinitosti} logičke operacije $and$.

\subsection{Logička operacija \emph{or}}

Pretpostavimo da ćete otići na izlet ako vrijedi \emph{"Lijepo
je vrijeme ili imam dobro društvo"}. Dakle, ako je lijepo vrijeme
idete na izlet, ako nije lijepo vrijeme i imate dobro društvo ipak
idete na izlet, ako nemate dobro društvo i lijepo je vrijeme opet
idete na izlet, a jedini slučaj kad ne idete na izlet je kad, niti
je vrijeme lijepo niti imate dobro društvo.

Tablica istinitosti logičke operacije $or$ izgleda ovako:

\begin{tabular}{ll|l}
	$A$ & $B$ & $A or B$ \\
	\hline
	istina & istina & istina \\
	istina & laz & istina \\
	laz & istina & istina \\
	laz & laz & laz 
\end{tabular}

\subsection{Logička operacija \emph{not}}

U zadnjem slučaju otići ćete na izlet tek ako vrijedi \emph{"Nemam
drugih obaveza"}. Dakle, tek ako \emph{nije} istit sud \emph{"Imam
drugih obaveza"}. Logička operacija $not$ nekom sudu pridodaje
suprotanu istinosnu vrijednost.

\begin{tabular}{l|l}
	$A$ & $not A$ \\
	\hline
	istina & laz\\
	laz & istina
\end{tabular}

Zadaci: Probajte odrediti istinitost sljedećih sudova:
	- 1<2 and 13!=5
	- 2>5 or 1=2
	- 1<2 and 5!=5
	- ( 1<2 or 9=5 ) and 3==3
	- ( 2==2 and 3!=5 ) or ( 3==4 )

\subsection{Nekoliko dodatnih pravila}

U programskom jeziku python (slično kao u mnogim drugim) postoji
još nekoliko dodadnih pravila kod utvrdjivanjaistinitosti sudova:
- svaki broj/izraz je po definiciji sud:
	- ako njegova vrijednost 0 onda je njegova istinosna vrijednost "laz" (F)
	- ako je različit od 0 onda je njegova istinosna vrijednost "istina" (T)
- svaki string je po definiciji sud:
	- ako je string prazan onda je njegova istinosna vrijednost "laz" (F)
	- ako string nije prazan onda je njegova istinosna vrijednost "istina" (T)

dakl ima smisla sud \verb"2<3 or 0". Budući da je $2<3$ ondaje prvi dio suda istinit, a (broj) 0 je
po gornjim pravilima lazan, dakle imamo slučaj "istina" $or$ "laz", dakle rezultat je istina.

Zadaci: Probajte odrediti istinitost sljedećih sudova:
\begin{itemize}
\item \verb"1 or 3>3"
\item \verb"2!=3 and 1"
\item \verb+"" or "jkljkl"+
\item \verb+( 0 or "jkljkl" ) and 2<3+
\item \verb+( not "jkljkl" ) and ( not 12 )+
\end{itemize}
\chapter{Kontrola toka programa}

\section{if \dots then \dots else \dots}

Pretpostavimo da imamo program koji od korisnika trazi da upiše
koliko je bodova zaradio na određenom testu i zatim ovisno o broju
bodova ispisuje koju je ocjenu dobio. Ako je dobio manje ili jednako
39 bodova ocjena je 1, ako ima 40-54 ocjena je 2, za 55-69 ocjena
je 3, za 70-85 ocjena je 4, a za više od 85 ocjenjen je s 5.

Probati ćemo za početak napisati program koji samo ispisuje je
li ocjena 1 ili veća. Za to program treba nekako imati način kako
će provjeriti je li broj bodova veći, manji ili jednak 40.

Program izgleda ovako:

\sourcee{
\var{bodovi} = \wrd{input}( "Upiši broj bodova:" )
\\
\wrd{print} "Imate", bodovi, "bodova..."
\\
\wrd{if} \var{bodovi} $<$ 40:
\\
\hspace*{10mm}\wrd{print} "Nazalost dobili ste negativnu ocjenu :("
\\
\wrd{print} "Kraj programa"
}

Korištena je naredba if, ona se koristi na sljedeći način:

\sourcee{
\wrd{if} \emph{logički izraz}:
\\
\hspace*{10mm}\emph{komande programa u slučaju da je logički izraz istinit}
}

Kao prvo uočite da je dio programa koji seizvršava u slučaju da
je logički izraz istinit \emph{uvučen} u odnosu na ostatak
programa. To uvlačenje je točno definirano i mora biti jedan
$<$tab$>$ (ili 8 razmaknica) računajući od lijevog ruba polja u kojem se
editira program!

U početnom programu se komanda nakon \verb"if" naredbe izvršava
jedino u slučaju ako je broj bodova manji od 40, ako je broj bodova
veći ili jednak 40 program tu liniju jednostavno preskače. Mozemo
srediti i da program preskače veći broj linija:

\sourcee{
\var{bodovi} = \wrd{input}( "Upiši broj bodova:" )
\\
\wrd{print} "Imate", bodovi, "bodova..."
\\
\wrd{if} \var{bodovi} $<$ 40:
\\
\hspace*{10mm}
	\wrd{print} "Nazalost dobili ste negativnu ocjenu :("
\\
\hspace*{10mm}
	\wrd{print} "Molimo vas lijepo da za sljedeći put malo bolje naučite"
\\
\wrd{print} "Kraj programa"
}

Poruka koja se sad ispisuje ukoliko nemate dovoljno bodova je 

\sourcee{
Nazalost dobili ste negativnu ocjenu :(
\\
Molimo vas lijepo da za sljedeći put malo bolje naučite
}

Opet, \verb"Kraj programa" se ispisuje bez obzira na broj bodova.

Zelimo li da naš program ispisuje i poruku ukoliko imate više ili jednako od 40 bodova to se moze
tako da koristite dva puta naredbu \verb"if":

\sourcee{
\var{bodovi} = \wrd{input}( "Upiši broj bodova:" )
\\
\wrd{print} "Imate", bodovi, "bodova..."
\\
\wrd{if} \var{bodovi} $<$ 40:
\\
\hspace*{10mm}
	\wrd{print} "Nazalost dobili ste negativnu ocjenu :("
\\
\wrd{if} \var{bodovi} $>$= 40:
\\
\hspace*{10mm}
	\wrd{print} "Imate više od 40 bodova"
\\
\wrd{print} "Kraj programa"
}

Ili pomoću jednog dodatka naredbi \verb"if":

\sourcee{
\var{bodovi} = \wrd{input}( "Upiši broj bodova:" )
\\
\wrd{print} "Imate", bodovi, "bodova..."
\\
\wrd{if} \var{bodovi} $<$ 40:
\\
\hspace*{10mm}
	\wrd{print} "Nazalost dobili ste negativnu ocjenu :("
\\
\wrd{else}:
\\
\hspace*{10mm}
	\wrd{print} "Imate više od 40 bodova"
\\
\wrd{print} "Kraj programa"
}

Program govori računalu:
\emph{Ako je broj bodova manji od 40 tada ste dobili negativnu
ocjenu, inače imate više od 40 bodova}

Sve ono što se nalazi nakon \verb"else", naravno uvučeno za jedan
$<$tab$>$ će biti ispisano u slučaju da uvjet \verb"bodovi$<$40" nije
istinit!

Sta ako ne zelimo samo podatak o tome jesmo li dobili više ili
manje od 40 bodova nego i koju smo ocjenu dobili:

Prvi način koristeći niz \verb"if"-ova:

\sourcee{
\var{bodovi} = \wrd{input}( "Upiši broj bodova:" )
\\
\wrd{print} "Imate", bodovi, "bodova..."
\\
\wrd{if} \var{bodovi} $<$ 40:
\\
\hspace*{10mm}
	\wrd{print} "Nazalost dobili ste negativnu ocjenu :("
\\
\wrd{if} 40$<$=\var{bodovi} and \var{bodovi}$<$=54:
\\
\hspace*{10mm}
	\wrd{print} "Dovoljan (2)"
\\
\wrd{if} 55$<$=\var{bodovi} and \var{bodovi}$<$=69:
\\
\hspace*{10mm}
	\wrd{print} "Dobar (3)"
\\
\wrd{if} 70$<$=\var{bodovi} and \var{bodovi}$<$=84:
\\
\hspace*{10mm}
	\wrd{print} "Vrlo dobar (4)"
\\
\wrd{if} 85$<$=\var{bodovi}:
\\
\hspace*{10mm}
	\wrd{print} "Odličan (5)"
}

Drugi način koristeći \verb"elif" (od eng. "else if"):

\sourcee{
\var{bodovi} = \var{input}( "Upiši broj bodova:" )
\\
\wrd{print} "Imate", \var{bodovi}, "bodova..."
\\
\wrd{if} \var{bodovi} $<$ 40:
\\
\hspace*{10mm}
	\wrd{print} "Nazalost dobili ste negativnu ocjenu :("
\\
\wrd{elif} 40$<$=\var{bodovi} and \var{bodovi}$<$=54:
\\
\hspace*{10mm}
	\wrd{print} "Dovoljan (2)"
\\
\wrd{elif} 55$<$=\var{bodovi} and \var{bodovi}$<$=69:
\\
\hspace*{10mm}
	\wrd{print} "Dobar (3)"
\\
\wrd{elif} 70$<$=\var{bodovi} and \var{bodovi}$<$=84:
\\
\hspace*{10mm}
	\wrd{print} "Vrlo dobar (4)"
\\
\wrd{elif} 85$<$=\var{bodovi}:
\\
\hspace*{10mm}
	\wrd{print} "Odličan (5)"
}

Na hrvatskom bi ovo napisali: \emph{Ako imate manje od 40 bodova dobili ste 2, inače ukoliko imate
između 40 i 54 dobili ste 2, inače ukoliko imate između 55 i 59 dobili ste 3, inače ukoliko imate
između 70 i 84 dobili ste 4, inače ukoliko imate više od 84 dobili ste 5}.

Mali problem moze nastati u tome što mozete napisati i 120 ili -340 za broj bodova. Probajmo
napisati program koji ispravlja tu grešku:

\sourcee{
\var{bodovi} = \wrd{input}( "Upiši broj bodova:" ) \\
\wrd{print} "Imate", \var{bodovi}, "bodova..." \\
\wrd{if} 0$<$= \var{bodovi} and \var{bodovi} $<$ 40: \\
\hspace*{10mm} \wrd{print} "Nazalost dobili ste negativnu ocjenu :(" \\
\wrd{elif} 40$<$=\var{bodovi} and \var{bodovi}$<$=54: \\
\hspace*{10mm} \wrd{print} "Dovoljan (2)" \\
\wrd{elif} 55$<$=\var{bodovi} and \var{bodovi}$<$=69: \\
\hspace*{10mm} \wrd{print} "Dobar (3)" \\
\wrd{elif} 70$<$=\var{bodovi} and \var{bodovi}$<$=84: \\
\hspace*{10mm} \wrd{print} "Vrlo dobar (4)" \\
\wrd{elif} 85$<$=\var{bodovi} and \var{bodovi} $<$=100: \\
\hspace*{10mm} \wrd{print} "Odličan (5)" \\
\wrd{else}: \\
\hspace*{10mm} \wrd{print} "Niste upisali broj između 0 i 100 za broj bodova"
}

Sad bi "prijevod" ovog programa na hrvatski glasio: \emph{Ako imate
između 0 i 40 bodova dobili ste 2, inače ukoliko imate između
40 i 54 dobili ste 2, inače ukoliko imate između 55 i 59 dobili
ste 3, inače ukoliko imate između 70 i 84 dobili ste 4, inače
ukoliko imate između 84 i 100 dobili ste 5, a ako nije niti jedan
od ovih slučajeva onda ste pogrešno upsali broj bodova}.

Zadaci: \textbf{Za napraviti}

\section{for \dots in range( \dots )}

Problem je sljedeći; treba ispisati tablicu kvadrata brojeva od do 10.
Tablica kvadrata je tablica koja ima dva stupca, u prvom se nalazeprirodni
brojevi, a u drugom njihovi kvadrati (da podsjetimo, kvadrat prirodnog
broja dobijemo tako da taj broj pomnozimo sa samim sobom).

To se moze postići na sljedeći način:

\sourcee{
\wrd{print} "n -$>$ n*n" \\
\wrd{print} 1, " -$>$ ", 1**2 \\
\wrd{print} 2, " -$>$ ", 2**2 \\
\wrd{print} 3, " -$>$ ", 3**2 \\
\wrd{print} 4, " -$>$ ", 4**2 \\
\wrd{print} 5, " -$>$ ", 5**2 \\
\wrd{print} 6, " -$>$ ", 6**2 \\
\wrd{print} 7, " -$>$ ", 7**2 \\
\wrd{print} 8, " -$>$ ", 8**2 \\
\wrd{print} 9, " -$>$ ", 9**2 \\
\wrd{print} 10, " -$>$ ", 10**2
}

\verb"5**2" znači $5^2$. U svakom retku se naredbom print ispisuje broj i
njegov kvadrat. 

Međutim, ovo je jedan prilično \emph{neelegantan} način rješavanja
problema, a to zato što je ovakav program teško generalizirati,
odnosno teško ga je upotrijebiti u slučaju da njime moramo
riješiti neki analogni ili opcehnitiji problem. 
\footnote{
	Ako je naš trenutni
	problem \emph{"Napiši program koji ispisuje tablicku kvadrata
	prvih 10 brojeva"} analogni (ali slozeniji) problem moze glasiti
	\emph{"Napiši program koji ispisuje tablicu kvadrata za brojeve
	od 0 do 100"}. Općenitiji problem bi mogao biti \emph{"Napiši
	program koji ispisuje tablicu kvadrata za brojeve od 0 do n (gdje
	je n proizvoljan prirodan broj)"}. 
}
Taj program se moze napisati
na način slučan našem načinu, ali priznati ćete da pisanje
sto i jedne linije tipa s naredbom \verb"print" koja ne radi ništa
drugo negoli ispisuje broj i njegov kvadrat i nije neki pretjerano
kreativan posao.

Zato postoji nareda \verb"for". Ta naredba od računala trazi da
određeni broj puta ponovi neki postupak uz određene uvjete. Uz
naredbu \verb"for" se nalazi ime varijable i nekakav \emph{skupa}
ili \emph{lista} prema kojima se ta varijabla "kreće". Budući da
će se ova skripta baviti skupovima, listama i ostalim slozenim
tipovima podataka baviti tek kasnije ovdje ću objasniti samo jedan
od načina koji se često koriste s naredbom \verb"for".

\sourcee{
\wrd{print} "n -$>$ n*n" \\
\wrd{for} \var{x} \wrd{in} \wrd{range}(11): \\
\tab \wrd{print} \var{x}, " -$>$ ", \var{x}**2
}

Rezultat programa će biti potpuno isti kao i kod prošlog programa.

Druga i treća linija ovog programa kaze otprilike: \emph{"Neka
varijabla x uzima redom vrijednost 0, 1, 2\dots sve dok je manje
od 11, i za svaku od tih vrijednosti ispiši vrijednost od x i
kvadrat od x"}. Nakon što python interpreter dođe do druge linije
programa računalo samo varijabli x pridodaje vrijednost 0, i
izvršava treću liniju programa (x je tamo 0). Nakon što to
izvrši varijabli x se pridruzuje sljedeća vrijednost; 1. Sad se
opet ispisuje treća linija programa, ali s novom vrijednosću
varijeble x. Nakon toga x poprima vrijednost 2, izvršava se treća
linija, i tako dalje\dots

\textbf{Vazno:}
Nikad ne zaboravite dvotočku iza "for" linije -- to je naime vrlo
česta početnička greška.

\textbf{Definicija:} Za liniju s \verb"for" naredbom i niz naredbi koje
se izvršavaju pri svakoj promjeni varijable nakon \verb"for"
kazemo da se zovu \textbf{for-petlja}. Ukoliko je \verb"x" varijabla
nakon \verb"for" kazemo da smo izvršili \textbf{for-petlju po varijabli
x}. Svaki put kad varijabla \verb"x" promijeni vrijednost zbog \verb"for"
petlje kazemo da je izvršeno \textbf{iteracija}. Niz naredbi koje se
izvršavaju u svakoj iteraciji zovemi tijelo for-petlje.

Isto tako, mozete srediti da se izvršava i više linija svaki
put kad varijabla x u zbog \verb"for" mijenja vrijednost. Jednostavno
nakon trećeg reda programa napisali biste još jedan, ali i taj
obavezno mora biti pomaknuti za jedan $<$tab$>$ udesno.

Pretpostavimo sad da treba napisati prgram koji od koristnika trazi
da mu upiše jedan broj, a zatim ispisuje prvo kvadrat tog broja,
a onda rezultat pri cjelobrojnom dijeljenju 
\footnote{
	Cjelobrojno dijeljenje znači da se radi o dijeljenju u
	kojemu se ignorira dio iza decimalne točke u rezultatu
	nekog dijeljenja. Npr ukoliko podijelimo 7 sa 4 dobiti
	ćemo 1.75, ali rezultat cjelobrojnog dijeljenja 7 sa 4 je
	1. Podsjetimo da python vrši cjelobrojno dijeljenje ukoliko
	su djeljenik i djelitelj cijeli brojevi (nemaju decimalnog
	dijela).
}
tog broja s 2.

\sourcee{
\var{n} = \wrd{input}( "Upiši broj:" )
\\
\wrd{for} \var{i} \wrd{in} \wrd{range}( \var{n}+1 ):
\\
\hspace*{10mm}
	\wrd{print} "kvadrat od", \var{i}, "je", \var{i}**2
\\
\hspace*{10mm}
	\wrd{print} "rezultat cjelobrojnog dijeljenja od", \var{i}, "s 2 je", \var{i}/2
\\
\wrd{print} "Kraj programa"
}

Ovdje se tijelo for-petlje sastoji od dvije naredbe (treći i
četvrti red programa. Zadnja linija ne spada u tijelo petlje jer
se ne izvršava kod svake iteracije nego samo jednom nakon što se
cijela petlja "izvrši".

Prvo se varijabli \verb"n" pridjeljuje vrijednost koju upisuje
osoba koja pokreće program. Zatim se vrši petlja po varijabli
\verb"i". Obratite paznju da se petlja sad kreće u granicama od
0, pa po svim cijelim brojevima manjim od \verb"n+1", a najveći
cijeli broj manji od \verb"n+1" je \verb"n". Dakle petlja se ne
izvrašava za \verb"n+1" nego samo do \verb"n", a to je upravo ono
što mi zelimo.

Zelimo sad program koji će za svaki broj od 0 do 100 ispisati
njegov kvadrat i, ukoliko je taj broj paran ispisati i poruku o
tome. Broj je paran u koliko je rezultati pri dijeljenju tog broja
s 2 jednak s 0. Rezultat pri dijeljenju dobijemo pomoću operacije
modulo (\verb"%" u python programu)

\sourcee{
\wrd{for} \var{x} \wrd{in} \wrd{range}(101): \\
\tab \wrd{print} \var{x}, "na kvadrat iznosi", \var{x}**2 \\
\tab \wrd{if} \var{x} \% 2 == 0: \\
\tab \wrd{print} \var{x}, "je paran broj" \\
\wrd{print} "Kraj programa"
}

\textbf{Komentar:} Tijelo for-petlje koja počinje s prvim redom
programa je niz naredbi koji se sastoji od 2-4 linije programa.
Obratite paznju da je ovdje 4. linija, koja se izvršava jedino
kad je istinit sud \verb"x%2==0" pomaknuta za \emph{dva} $<$tab$>$-a!
Za svaki \verb"x" if-uvjet isprobava je li istina da je rezultat
pri djeljenju x s 2 jednak 0 (\verb"x%2==0"), a ako je to istina
izvršava se 4. linija.

Evo još jedna varijanta programa:
\footnote{
Zbog lakšeg opisivanja programa ovdje je na početku svakog retka
ispisan i redni broj tog reda. Kad prepisivanja programa u editoru to n
prepisujete.
}

\sourcee{
1: \wrd{for} \var{x} \wrd{in} \wrd{range}(101): \\
2:\tab \wrd{print} \var{x}, "na kvadrat iznosi", \var{x}**2 \\
3:\tab \wrd{if} \var{x} \% 2 == 0: \\
4:\tab \tab \wrd{print} \var{x}, "je paran broj" \\
5:\tab \wrd{else}: \\
6:\tab \tab \wrd{print} \var{x}, "nije paran broj" \\
7:\tab \wrd{print} \\
8: \wrd{print} "Kraj programa"
}

U čemu se ona razlikuje od gornjeg programa? Probajte sami otkriti šta
radi sedma linija programa? Ovisi li njeno izvršavanje o tome je li
izvršen uvjet u 3. redu? Probajte promijeniti program tako da se petlja ne
izvršava za brojeve do 100 nego da korisnik moze sam odrediti do kojeg do
kojeg broja se petlja izvršava.

Probajte objasniti zašto donji program radi potpuno istu stvar
kao i naš zadnji program:\footnote{Pomoć: svaki broj je po definiciji sud
čija istinitost ovisi o\dots}

\sourcee{
1: \wrd{for} \var{x} \wrd{in} \wrd{range}(101): \\
2: \tab \wrd{print} \var{x}, "na kvadrat iznosi", \var{x}**2 \\
3: \tab \wrd{if} \wrd{not} \var{x} \% 2: \\
4: \tab \tab \wrd{print} \var{x}, "je paran broj" \\
5:	\tab \wrd{else}: \\
6: \tab\tab \var{x}, "nije paran broj" \\
7: \tab \wrd{print} \\
8: \wrd{print} "Kraj programa"
}

Još jedna korisna varijanta varijanta for-petlja s \verb"range(...)" je
slučaj u kojem ne zelim da se petlja izvršava za brojeve od 0 do nekog
broja nego od nekog broja različitog od 0 do nekog drugog broja:

\sourcee{
\wrd{for} \var{i} \wrd{in} \wrd{range}( 15, 70 ):\\
\hspace*{10mm} \wrd{print} \var{i}
}

Ispisati će brojeve od 15 do 70.

Probajte objasniti šta radi sljedeći program:

\sourcee{
\var{a} = \wrd{input}( "Upiši prvi broj:" ) \\
\var{b} = \wrd{input}( "Upiši prvi broj:" ) \\
\wrd{if} \var{a} $<$ \var{b}: \\
\hspace*{10mm} \wrd{for} \var{x} \wrd{in} \wrd{range}( \var{a}, \var{b} + 1 ): \\
\hspace*{10mm} \hspace*{10mm} \wrd{print} \var{x}, "na treću iznosi", \var{x}**3 \\
\wrd{else}: \\
\hspace*{10mm} \wrd{print} "hmmm..."
}

Napišite sada program koji od korisnika trazi da upiše dva broja, a
zatim ispisuje tablicu kvadrata svh brojeva od prvog do drugog.

Evo još jedan zanimljiv program koji ispisuje tablicu mnozenja brojeva od
1 do 10. Treba dakle napisati sve izraze oblika $a\cdot b$ gdje $a$ i $b$
mogu biti brojevi od 1 do 10. Ali (!) za svaki $a$ od 1 do 10 (for-petlja)
i $b$ mora moći poprimiti vrijednosti od 1 do 10.

\sourcee{
\wrd{for} \var{a} \wrd{in} \wrd{range}(1, 11): \\
\hspace*{10mm} \wrd{for} \var{b} \wrd{in} \wrd{range}(1, 11): \\
\hspace*{10mm} \hspace*{10mm} \wrd{print} \var{a}, "puta", \var{b}, "je jednako", \var{a}*\var{b}
}

Tijelo prve petlje je drugi i treći red, a tijelo druge petlje je samo
treći red. Dakle, za svaki a od 1 do 10 izvršiti će se druga petlja u
kojoj se sad b mijenja od 1 do 10 i kod svake promjene ispisuje poruka.

Još jedan primjer:

Pretpostavimo da imamo ovakav problem:
Polaznik jedne autoškole polazu pismeni dio vozačkog ispita.
Ispit se sastoji od 15 pitanja koji su podijeljeni u tri skupine po 5 zadatka.
Zadaci iz prve skupine nose po 1 bod, iz druge skupine 2 boda, a iz treće
skupine po 3 boda. Za prolaz na ispitu potrebno je da polaznik ima barem 24 boda.
Treba napisati program koji od instruktora trazi da upiše broj točnih
zadataka iz prve, druge i treće skupine i onda ispisuje je li polaznik
autoškole zadovoljio uvjete.

Program bi mogao izgledati ovako:

\sourcee{
\var{a} = \wrd{input}( "Broj tocnih zadataka u prvoj skupini:" ) \\
\wrd{b} = \wrd{input}( "Broj tocnih zadataka u drugoj skupini:" ) \\
\var{c} = \wrd{input}( "Broj tocnih zadataka u trecoj skupini:" ) \\
\wrd{if} \var{a}+\var{b}+\var{c} $>$= 24: \\
\hspace*{10mm} \wrd{print} "Polaznik je polozio ispit" \\
\wrd{else}: \\
\hspace*{10mm} \wrd{print} "Polaznik nije polozio ispit"
}

Ukoliko znamo da ta autoškola ima 20 polaznika mozemo srediti da program
radi za sve polaznike autoškole:

\sourcee{
\wrd{for} \var{i} \wrd{in} \wrd{range}(20): \\
\hspace*{10mm} \var{a} = \wrd{input}( "Broj tocnih zadataka u prvoj skupini:" ) \\
\hspace*{10mm} \var{b} = \wrd{input}( "Broj tocnih zadataka u drugoj skupini:" ) \\
\hspace*{10mm} \var{c} = \wrd{input}( "Broj tocnih zadataka u trecoj skupini:" ) \\
\hspace*{10mm} \wrd{if} \var{a}+\var{b}+\var{c} $>$= 24: \\
\hspace*{10mm} \hspace*{10mm} \wrd{print} "Polaznik je polozio ispit" \\
\hspace*{10mm} \wrd{else}: \\
\hspace*{10mm} \hspace*{10mm} \wrd{print} "Polaznik nije polozio ispit"
}

Ovaj program funkcionira samo za autoškole s 20 polaznika, šta ako ne znamo
koliko polaznika ima autoškola?

\sourcee{
\var{n} = \wrd{input}( "Upisi broj polaznika autoskole: )\\
\wrd{for} \var{i} \wrd{in} \wrd{range}(n): \\
\hspace*{10mm} \var{a} = \wrd{input}( "Broj tocnih zadataka u prvoj skupini:" ) \\
\hspace*{10mm} \var{a} = \wrd{input}( "Broj tocnih zadataka u prvoj skupini:" ) \\
\hspace*{10mm} \var{b} = \wrd{input}( "Broj tocnih zadataka u drugoj skupini:" ) \\
\hspace*{10mm} \var{c} = \wrd{input}( "Broj tocnih zadataka u trecoj skupini:" ) \\
\hspace*{10mm} \wrd{if} \var{a}+\var{b}+\var{c} $>$= 24: \\
\hspace*{10mm} \hspace*{10mm} \wrd{print} "Polaznik je polozio ispit" \\
\hspace*{10mm} \wrd{else}: \\
\hspace*{10mm} \hspace*{10mm} \wrd{print} "Polaznik nije polozio ispit"
}

\section{while \dots}

Pomoću if-uvjeta i for-petlje se mog riješiti mnogi problemi koji se
postavljaju pred programera. Ono prvo nam omogućuje da računalo samo
odlučuje koje će se naredbe izvršavati ovisno o nekom uvjetu, a
pomoću for-petlje mozemo ponavljati određeni postupak određen broj
puta. Međutim, moze nam se desiti da je određeni niz naredbi potrebno
ponavljati nekoliko puta, ali da ne znamo koliko puta treba ponoviti prije nego
što pokrenemo program. 

While-petlja rješava taj problem. Opći oblik while-petlje je

\sourcee{
\wrd{while} \emph{logički izraz}:
\\
\hspace*{10mm}\emph{niz (blok) naredbi}
}

Niz naredbi će se ponavljati sve dok je uvijet istinit. 

Primjer koji ispisuje prvih 10 brojeva:

\sourcee{
\var{n} = 1 \\
\wrd{while} \var{n} $<$= 10: \\
\hspace*{10mm} \wrd{print} "varijabla n je sad", \var{n} \\
\hspace*{10mm} \var{n} = \var{n} + 1
}

Pročitajmo šta ovaj program trazi od računala:
\begin{itemize}
	\item Neka $n$ bude jednak 1
	\item Sve dok je $n$ manje od 10 ponavljaj:
	\begin{itemize}
		\item Ispiši koliki je $n$
		\item Neka $n$ poprimi vrijednost od $n$ uvećano za 1.
	\end{itemize}
\end{itemize}

Sve smo to, naravno mogli i s for-petljom. U čemu je onda bitna razlika? Razlika
je u tome što kod for-petlje na samom početku petlje morate znati koliko
puta će se petlja ponoviti,
\footnote{Postoji način i da se to zaobiđe pomoću "break" i "continue", ali o tome kasnije}
a while petlja mozete pokrenuti tako da se tek negdje tokom njenog izvođenja
desi nešto tako da uvijet kod \verb"while" postane lazan.

Promotrimo npr. zadnji program iz poglavlja o for-petlji; on korektno radi svoj posao
ukoliko instruktor (ili osoba koja ispravlja ispite) zna koliko ima polaznika.
Pretpostavimo sad da instruktor ne zna točno koliko je polaznika pisalo ispit, a
ispit se odvija na sljedeći način: polaznici jedan po jedan ulaze u
instruktorovu kancelariju, kad uđu on uzima njihov ispit, ispravlja ga i
zeli da mu računalo na osnovu točno rješenih zadataka iz ispita
kaze za svakog polaznika je li prošao ili nije. Instruktor nema vremena
izbrojati koliko je bilo polaznika, pa moze samo primati jednog po jednog
budućeg vozača, ali nikad ne zna je li on zadnji ili iza njega ima još
njih.

Program sad nebi radio jer instruktor na početku programa mora napisati koliko
je bilo ispitanika, a on to ne zna. probati ćemo to sada riješiti
pomoću while-petlje:

\sourcee{
\textcolor{red}{jos} = "da" \\
\textcolor{blue}{while} \var{jos} != "da": \\
\hspace*{10mm} \textcolor{red}{a} = \textcolor{blue}{input}( "Broj tocnih zadataka u prvoj skupini:" ) \\
\hspace*{10mm} \textcolor{red}{b} = \wrd{input}( "Broj tocnih zadataka u drugoj skupini:" ) \\
\hspace*{10mm} \textcolor{red}{c} = \wrd{input}( "Broj tocnih zadataka u trecoj skupini:" ) \\
\hspace*{10mm} \wrd{if} \var a+\var b+\var c $>$= 24: \\
\hspace*{10mm} \hspace*{10mm} \textcolor{blue}{print} "Polaznik je polozio ispit" \\
\hspace*{10mm} \textcolor{blue}{else}: \\
\hspace*{10mm} \hspace*{10mm} \textcolor{blue}{print} "Polaznik nije polozio ispit" \\
\hspace*{10mm} \textcolor{red}{jos} = \textcolor{blue}{raw\_input}( "Ima li jos polaznika (da/ne) " )
}

Blok naredbi u while-petlji će se ponavljati sve dok je varijabla \verb"jos"
razlicita od "da" (ta varijabla sadrzi string, a ne broj). Kako je na samom
početku njoj pridruzena vrijednost "da" while petlja se počne
izvršavati, a na samom kraju se od korisnika trazi da upiše ima li
još polaznika. Ako on upiše "da"\footnote{Bez navodnika,
naravno} (ili bilo šta drugo osim "ne") \verb"jos"
je i dalje različito od "ne" pa se blok opet izvršava. Ukoliko upišete
"ne" program završava svoj posao.

\section{Potprogrami i funkcije}

\subsection{Procedure}

Dešava se da u nekom programu moramo često ponavljati određeni
niz naredbi. Potprogrami
su strukture u programu koje nam omogućuju da određeni niz naredbi
koje se u programu trebaju često izvršavati u programu napišemo
samo jednom.

Slijedi primjer jednostavnog programa koji koristi potprogram:

\sourcee{
\wrd{def} \fun{HelloWorld}():\\
\tab \wrd{print} "Hello world!!! (nalazim se  tijelu potprograma)"\\
\com{\# Ovdje pocinje glavni dio programa}\\
\wrd{print} "Pocetak programa"\\
\fun{HelloWorld}()\\
\fun{HelloWorld}()\\
\wrd{print} "Kraj programa"
}

Rezultat je:

\sourcee{
Pocetak programa \\
Hello world!!! (nalazim se  tijelu funkcije) \\
Hello world!!! (nalazim se  tijelu funkcije) \\
Kraj programa
}

Kad je program pokrenut on nije odmah ispisao poruku iz drugog
reda, ali kasnije, svaki put kad smo upisali HelloWorld() \footnote{Ne
zaboraviti zagrade!!} ispisana je poruka "Hello world!!! (nalazim
se  tijelu funkcije)". Dakle, kao da smo definirali jednu novu
komandu koja ispisuje svoju poruku. 

kad u programu napišete "HelloWorld()" kazemo da smo
\textbf{pozvali potprogram} i tada se izvršavaju komande u \textbf{tijelu
potprograma}. Tijelo potprograma se u našem primjeru sastoji od samo
jednog retka.

Čemu sluze zagrade iza \verb"HelloWorld"? 

Vratimo se opet na primjer s instruktorom voznje. Zelim sad da
potprogram ispiše pravu poruku ovisno o tome koliko je polaznik imao
bodova. 

\sourcee{
\wrd{def} \fun{IspisiPoruku}( \var{b} ): \\
\tab \wrd{if} \var{b} $>$= 24: \\
\tab \tab \wrd{print} "Prosli se ispit" \\
\tab \wrd{else}: \\
\tab \tab \wrd{print} "Niste prosli ispit" \\
\\
\fun{IspisiPoruku}( 21 )\\
\fun{IspisiPoruku}( 29 )
}

S retkom "def IspisiPoruku( b ):" počinje \textbf{definicija potprograma}.
Budući da smo unutar zagrade iza imena potprograma ("IspisiPoruku")
napisali "b" -- svaki put kad pozovemo potprogram morati ćemu unutar
zagrade staviti neku vrijednost (varijabla, broj, string). Vrijednost onoga
što smo upisali u zagradu tamo gdje potprogram pozivamo će u tijelu
potprograma
poprimiti varijabla "b". Kako smo mi upisali "IspisiPoruku( 21 )", u tijelu
će varijabla b imati vrijednost 21. A sljedeći put će
zbg istog razloga varijabla "b" (unutar potprograma) imati vrijednost 29.

Kazemo da je varijabla "b" u zagradi iza definicije potprograma
\textbf{argument potprograma}.

Potprogram moze imati i više argumenata; sljedeći program
sadrzi potprogram koji ispisuje srednju vrijednost od tri broja i ima
tri argumenta:

\sourcee{
\wrd{def} \fun{SrednjaVrijednost}( \var{a}, \var{b}, \var{c} ):\\
\tab \com{""" Funkcija koja računa srednju vrijednost brojeva a, b i c }\\
\tab \com{ispisuje tu srednju vrijednost """}\\ 
\tab \var{s} = ( \var{a} + \var{b} + \var{c} ) / 3. \\
\tab \com{\# iza 3 moramo smo morali staviti točku, jer inače bi se radilo}\\
\tab \wrd{print} "Srednja vrijednost brojeva", \var{a}, ",", \var{b}, "i", \var{c}, "je", \var{s}\\
\\
\fun{SrednjaVrijednost}( 1, 2, 3 )\\
\var{x} = 2\\
\fun{SrednjaVrijednost}( 4, \var{x}, 7.5 )\\
\fun{SrednjaVrijednost}( \var{x}+1, \var{x}, 1 )\\
\fun{SrednjaVrijednost}( 2*3, 3*3, 4**2 )
}

Program ispisuje:

\sourcee{
Srednja vrijednost brojeva 1 , 2 i 3 je 2.0\\
Srednja vrijednost brojeva 4 , 2 i 7.5 je 4.5\\
Srednja vrijednost brojeva 3 , 2 i 1 je 2.0\\
Srednja vrijednost brojeva 6 , 9 i 16 je 10.3333333333
}

\textbf{Vazno:} Pozeljno je, kao u gornjem primjeru odmah nakon
definicije funkcije napisati jedan komentar (ograničenog s """ i """)
koji otprilike opisuje čemu sluzi ta funkcija.

Dodajmo sad još jedan uvijet primjeru s instruktorom
voznje;  među polaznicima autoškole postoje oni koji polazu za
voznju automobila i njima je potrebno 26 boda za prolaz, i postoje oni
koji polazu za voznju mopeda kojima su dovoljna 24 boda za prolaz.
Treba napisati program koji koristi potprogram s četiri argumenta (broj
točno rješenih zadataka u svakoj od tri skupine) i podatak o tome
radi li se o polazniku za moped ili automobil.

\sourcee{
\wrd{def} \fun{IspisiPoruku}( \var{s1}, \var{s2}, \var{s3}, \var{za} ):\\
\tab \com{""" Ispisuje je li polaznik koji je imao:\\
\tab \tab s1 točno rješen zadatak iz prve grupe,\\
\tab \tab s2 točno rješen zadatak iz druge grupe,\\
\tab \tab s3 točno rješen zadatak iz trece grupe\\
\tab ...prosao test.\\
\tab Ako je varijabla 'za' jednaka "auto" onda se racuna kao za\\
\tab polaznika za dozvou za voznju automobila, inace\\
\tab za polaznika voznje s mopedom\\
\tab """}\\
\tab \var{bodovi} = \var{s1} + \var{s2} * 2 + \var{s3} * 3\\
\tab \wrd{if} \var{za} == "auto":\\
\tab \tab \wrd{if} \var{bodovi} $>$= 26:\\
\tab \tab \tab \wrd{print} "Prosli ste"\\
\tab \tab \wrd{else}:\\
\tab \tab \tab \wrd{print} "Niste prosli"\\
\tab \wrd{else}:\\
\tab \tab \wrd{if} \var{bodovi} $>$= 24:\\
\tab \tab \tab \wrd{print} "Prosli ste"\\
\tab \tab \wrd{else}:\\
\tab \tab \tab \wrd{print} "Niste prosli"\\
		\\
\fun{IspisiPoruku}( 5, 4, 4, "auto" )\\
\fun{IspisiPoruku}( 5, 4, 4, "motor" )
}

\subsection{Funkcije}

Funkcija je matematički pojam; postoje npr. linearne funkcije,
trigonometrijske funkcije, logaritamske funkcije, i tako dalje i
tako blize.

Na primjer, pretpostavimo da imamo funkciju $f(x)=2\cdot x-3$. Tada
je $f(4)=2\cdot 4-3=8-3=5$, a to znači da funkcija $f$ preslikava
broj 4 u broj 5, ili mogli bi reći da funkcija f za argument
4 vraća 5.

Funkcija se od gornjih potprograma razlikuje u tome što funkcija
ima povratnu vrijednost. Rezultat funkcije se moze ispisati,
pridruziti nekoj drugoj varijabli ili s njime računati. 

Primjer programa koji koristi funkciju:

\sourcee{
\wrd{def} \fun{Zbroji}( \var{a}, \var{b} ):\\
\tab \wrd{return} \var{a} + \var{b}\\
\\
\wrd{print} \fun{Zbroji}( 2, 3 )\\
\wrd{print} \fun{Zbroji}( -3, 3.5 )
}

Definicija funkcije je ista kao i definicija potprograma, jedino što
funkcija mora imati neku vrijednost koju vraća u glavni program. Kao
što vidimo ovdje ispisujemo "Zbroj( 2, 3 )", a ono što će
ispisati je upravo ona vrijednost koja se vraća pomoću naredbe
"return".

Ovdje je potrebno jedno malo objašnjenje: U stvari ne postoji razlika
između potprograma i funkcije. Funkcija \emph{mora} vraćati neku
vrijednost pomoću "return", ali čak i potprogram koji ne sadrzi
"return" vraća vrijednost "None", ali o tome malo više u
sljedećem poglavlju.

Evo primjer funkcije koja vraća zbroj svih brojeva manjih od nekog
zadanog:

\sourcee{
\wrd{def} \fun{ZbrojBrojeva}( \var{n} ):\\
\tab \var{x} = 0\\
\tab \wrd{for} \var{i} \wrd{in} \wrd{range}( \var{n} + 1 ):\\
\tab \tab \var{x} = \var{x} + \var{i}\\
\tab \wrd{return} \var{x}	\\
\\
\wrd{print} \fun{ZbrojBrojeva}( 5 )\\
\var{a} =  \fun{ZbrojBrojeva}( 10 )\\
\var{b} = \fun{ZbrojBrojeva}( 100 )\\
\wrd{print} \var{a} + \var{b} - 100
}

Kao što vidite, s vrijednošću koju funkcija vraća
mozemo računati ili tu vrijednost mozemo pridruziti
drugim varijablama.

Vratimo se sad primjeru s instruktorom voznje, taj se program pomoću funkcije
mogao napisati ovako:

\sourcee{
\wrd{def} \fun{IspisiPoruku}( \var{s1}, \var{s2}, \var{s3}, \var{za} ): \\
\tab \com{""" Vraca string s porukom je li polaznik koji je imao: \\
\tab  s1 tocno rjesen zadatak iz prve grupe, \\
\tab  s2 tocno rjesen zadatak iz druge grupe, \\
\tab  s3 tocno rjesen zadatak iz trece grupe \\
\tab ...prosao test.  \\
\tab Ako je varijabla 'za' jednaka "auto" onda se racuna kao za \\
\tab polaznika za dozvou za voznju automobila, inace \\
\tab za polaznika voznje s mopedom \\
\tab """} \\
\tab \var{bodovi} = \var{s1} + \var{s2} * 2 + \var{s3} * 3 \\
\tab \wrd{if} \var{za} == "auto": \\
\tab \tab \wrd{if} \var{bodovi} $>$= 26: \\
\tab \tab \tab \var{temp} = "Prosli ste" \\
\tab \tab \wrd{else}: \\
\tab \tab \tab \wrd{temp} = "Niste prosli" \\
\tab \wrd{else}: \\
\tab \tab \wrd{if} \var{bodovi} $>$= 24: \\
\tab \tab \tab \var{temp} = "Prosli ste" \\
\tab \tab \wrd{else}: \\
\tab \tab \tab \var{temp} = "Niste prosli" \\
\tab \wrd{return} \var{temp} \\
\tab \\
\wrd{print} \fun{IspisiPoruku}( 5, 4, 4, "auto" ) \\
\wrd{print} \fun{IspisiPoruku}( 5, 4, 4, "motor" ) 
}

Bitna razlika između ovog i prošle verzije ovog programa je u tome što je u
prošlom programu bilo dovoljno napisati npr. "IspisiPoruku( 5, 4, 4, 'motor' )" kao
komandu, a potprogram bi onda ispisao poruku. Sad potprogram samo vraća vijrednost
koju onda treba ispisati. 

\subsection{Ugrađene funkcije}

\dots

\subsection{Doseg varijabli}

\dots

\section{try}

\dots

\section{break i continue}

\dots

\chapter{Tipovi podataka}

\section{Jednostavni tipovi podataka}

	Varijable u dosadašnjim programima se kao vrijednosti imale brojeve, ali
	mogle su sadrzavati i razne druge vrste podataka.

	Ovisno o tome koliko je slozen problem kojeg zelimo riješiti s
	programom podaci s kojima trebamo raditi mogu biti i vrlo slozeni. Okvirno
	podatke mozemo podijeliti na dva jednostavne i slozene. Slozeni tipovi
	podataka su oni koji u sebi mogu sadrćavati nekoliko drugih podataka.

	Jednostavni su cjelobrojni tip, realni brojevi i stringovi.

\subsection{Cijeli i realni brojevi}

	Cijeli brojevi su svi prirodni brojevi, nula i brojevi suprotni prirodnim
	brojevima.

	Kad kazemo realni brojevi u programiranju obično mislimo samo na relalne
	brojeve koje mozemo zapisati u s točno određenim brojem decimalnih
	mjesta. Za svako računalo i programski jezik postoje točne granice koliko
	najvieš decimalnih mjesta mogu sadrzćavati, koja je najmanja, a koja
	najveća moguća brojevna vrijednost koju mozemo koristiti itd.

	\sourcee{
	\var{broj} = 12.13 \textcolor{green}{\# realni broj}\\
	\var{c} = 12 \textcolor{green}{\# cijeli broj}\\
	\var{r} = 12.0 \textcolor{green}{\# realni broj (jer ima decimalnu tocku!)}
	}

	U zadnjem slučaju varijabla "r" sadrzava realan broj jer je 12.0
	opisan kao decimalan broj. 

	Realne brojeve mozemo napisati i u "znanstvenom obliku", dakle u obliku $a\cdot
	10^{b}$.

	\vspace{3mm}
	\begin{tabular}{l|l}
		Broj & u Python programu\\
		\hline
		$5\cdot 10^{13}$ & \verb+5e13+\\
		$3.56\cdot 10^{-17}$ & \verb+3.56e-17+\\
	\end{tabular}
	\vspace{3mm}

	Brojevi u heksadecimalnom ili oktalno zapisu i imaginarni brojevi:

	\vspace{3mm}
	\begin{tabular}{l|l|l}
		& Zapis & U programu \\	
		\hline
		Heksadecimalni & 177 & 0177 \\
		\hline
		Oktalni & BAB7& 0xBAB7 \\
		\hline
		Imaginarni & $i$ & 1j \\
		\hline 
		Imaginarni & $0.5i$ & 0.5j \\
		\hline 
		Imaginarni & $2.5+3i$ & 2.5+3j \\
	\end{tabular}
	\vspace{3mm}

	Ukoliko radimo s cjelobrojnim varijablama i veličinama, računalo nam
	postavlja određene granice. Ne mozemo računati s \emph{običnim}
	cijeli brojem koji ima 20 znamenaka. Kad nam je zbog nekog slozenog računa
	potrebno raditi s tako veliki brojevima moramo iza samog broja dodati "L"\footnote{Ovo
	"L" moze slobodno biti napisano i malim slovom "l", ali ovdje je napisano "L"
	da se ne bi pomiješalo "l" (malo "l") s "I" (veliko "i")}. Dakle ne
	ne bi napisali 
	
	"a = 128904389523789123789"

	nego\dots

	"a = 128904389523789123789L"

	Probajte oba slučaja napisati kao dio jednog vašeg programa i nakon toga
	probati ispisati tu varijablu s "print a" i pogledajte što se desilo!

\subsection{Stringovi}

	String je niz znakova proizvoljne duzine. Član stringa moze biti svaki
	simbol kojeg mozete dobiti pritiskom na neku tipku tastature, a i omnogi drugi.
	Stringovi se zapisuju u navodnicima. Dakle primjeri stringova su 
	>>Ovo je string"123"<<, >>'sdjkl'<< ili >>"""String"""<<

\textbf{Zapamtite:} Stringove (nizove znakova) mozemo zapisati
na tri načina: 

	\begin{itemize}
		\item Unutar dvostrukih navodnika -- " 
		\item Unutar jednostrukih navodnika -- '
		\item Ograničenih (na početku i na kraju) s nizom od tri dvostruka
			navodnika
	\end{itemize}

	Ponekad će nam trebati da unutar stringa moramo imati neki drugi navodnik. Da
	bi to postigli promotrimo sljedeći primjer.
	
	\sourcee{
		\var{str} = "Ovo je jedan 'string'"\\
		\textcolor{blue}{print} \var{str}
	}

	Ispisati će

	\sourcee{
		Ovo je jedan 'string'
	}

	Postoji još nekoliko načina:

	\sourcee{
		\var{str} = "Ovo je jedan 'string'" \\
		\textcolor{blue}{print} str \\
		\var{str2} = 'Ovo je jos jedan "string"' \\
		\textcolor{blue}{print} \var{str2} \\
		\var{str3} = \textcolor{green}{"""Ovo je jedan "string" """} \\
		\textcolor{blue}{print} \var{str3} \\
		\var{str4} = "Ovo je jedan $\setminus$"string$\setminus$"" \\
		\textcolor{blue}{print} \var{str4}
	}

	Rezultat nakon pokretanja programa je:

	\sourcee{
		Ovo je jedan 'string' \\
		Ovo je jos jedan "string" \\
		Ovo je jedan "string" \\
		Ovo je jedan "string"
	}

	Dakle, dvostruki navodnik u stringu mozete dobiti tako da string
	ograničite s jednostrukim navodnicima ili s nizom trostrukih navodnika. I,
	postoji još jedan izuzetno vazan način, a to je ono 
	>>str4 = "Ovo je jedan \"string\""<<
	Tamo gdje zelimo da nam se u varijabli nalazi jednostruki ili dvostruki
	navodnik jednostavimo \verb+\"+. Isto tako bi mogli staviti i \verb+\'+. 
	
	Ako recimo zelimo da se naš string sastoji samo od jednog dvostrukog
	navodnika mozemo napisati \verb+a = "\""+, a ako zelimo da se sastoji od
	znaka \verb+\\+ i " napisali bi \verb+a = "\\\""+ -- prvi i zadnji navodnik su oznake gdje
	počinje, a gdje završava string. Nakon prvog navodnika niz \verb+\\+ znači
	da se tu nalazi simbol \verb+\+, a \verb+\+" je simbol dvostrukog navodnika.
	
	Postoji
	određeni broj znakova i i simbola koje mozemo staviti u string samo
	kombinacijom \verb+\+ i taj znak ili neko slovo:

	{\normalsize
	\begin{tabular}{ll}
		\verb+\+$<$newline$>$ & Ignored \\
		\verb+\\+ & Backslash (\verb+\+) \\
		\verb+\'+ & Jednostruki navodnik (') \\
		\verb+\"+ & Dvostruki navodnik (") \\
		\verb+\a+ & ASCII Bell (BEL)\\
		\verb+\b+ & ASCII Backspace (BS)\\
		\verb+\f+ & ASCII Formfeed (FF)\\
		\verb+\n+ & ASCII Linefeed (LF)\\
		\verb+\N{name}+ & Character named name in the Unicode database (Unicode only)\\
		\verb+\r+ & ASCII Carriage Return (CR)\\
		\verb+\t+ & ASCII Horizontal Tab (TAB)\\
		\verb+\uxxxx+ & Character with 16-bit hex value xxxx (Unicode only)\\
		\verb+\Uxxxxxxxx+ & Character with 32-bit hex value xxxxxxxx (Unicode only)\\
		\verb+\v+ & ASCII Vertical Tab (VT)\\
		\verb+\ooo+ & ASCII character with octal value ooo\\
		\verb+\xhh+ & ASCII character with hex value hh\\
	\end{tabular}
	}

	\vspace{2mm}
Veliku većinu njih mozete slobodno zaboraviti, ali ima
nekoliko njih koje ćete često koristiti: \verb+\<newline>+
(ovdje \verb+<newline>+ predstavlja tipku "enter", "newline" ili
"return"). Interpreter to jednostavno ignorira, što je jako
korisno ako imamo string koji je prevelik za jedan red pa ga u
programu zelimo imati napisanog u više redova.

Izuzetno vazna je i kombinacija \verb+\n+ -- kada u string
stavimo tu kombinaciju, pri ispisu stringa će na tom mjestu
računalo preći u novi red.

	\sourcee{
		\var{a} = "Ovo je jedan $\setminus$"string$\setminus$" koji je $\setminus$ \\
		toliko dug da mi ga je malo nezgodno $\setminus$ \\
		imati u jednom redu, pa sam ga napisao $\setminus$ \\
		u vise redova" \\
		\var{b} = "A, ovo je jedan$\setminus$n$\setminus$n$\setminus$n$\setminus$nhmmm..." \\
		\textcolor{blue}{print} \var{a} \\
		\textcolor{blue}{print} \var{b}
	}

	Sadrzaj stringa a je "Ovo je jedan $\setminus$"string$\setminus$" koji je toliko dug da mi ga je malo nezgodno imati u jednom redu, pa sam ga napisao u vise redova",
	a kad budemo ispisivali varijablu b vidjeti ćemo šta se zbiva s onim
	\verb+\n+ -- svaki put kad ga računalo "sretne" otići će u novi red.
	Dakle ispisati će "A ovo je jedan" zatim tri puta novi red (dakle tri razmaka
	od jedan red) i onda "hmmmm...".

	Rezultat je dakle:

	\sourcee{
	Ovo je jedan "string" koji je toliko dug da mi ga je malo nezgodno imati u jednom
	\\
	redu, pa sam ga napisao u vise redova
	\\
	A, ovo je jedan
	\\
\ 
	\\
\ 
	\\
\ 
	\\
	hmmm...
	}

Uočite da je računalo, ipak, string \verb+a+ napisalo u
dva reda i to jednostavno zato što mu nije stalo u jedan red.
Da smo imali dovoljno velik monitor bilo bi napisano sve u jednom
redu za razliku od string b u kojemu će uvijek ispisati ona
tri prazna reda upravo zato što smo mi eksplicitno traćili
da oni tu budu.

Još samo jedna zadnja napomena. Trebate znati razlikovati
između \verb+a = 123+ i \verb+a = "123"+. U prvom slučaju
varijabla \verb+a+ će sadrzavati \underline{broj} 123 i s
njim mozemo raditi sve ono što mozemo raditi s brojevima,
a u drugom slučaju varijabla a sadrzava \underline{string}
"123". Sa brojem 123 ćemo moći normalno računati kao
što općenito mozemo s brojevima, a sa stringom "123"
to ne mozemo.

\subsection{Konverzija tipova}

\dots

\subsection{Varijable i vrste podataka}

Neki programski jezici od programera zahtijevaju da točno odredi
kakve će podatke (odnosno vrste podataka) neka varijabla
sadrzavati. Na početku programa se odredi da će npr.
varijabla \verb+x+ sadrzavati samo cijele brojeve, a ako onda
negdje u programu toj varijabli pokušamo pridruziti neki
string ili realan broj, javiti će nam se greška. Za primjer
takvog programskog jezika pogledajte Pascal-program u 1.3.1.  U
Pythonu svaka varijabla moze sadrzavati podatak bilo kojeg
tipa. Ipak, postoje neke dobre programerske navike, a jedna od njih
je da se trudite varijablama . Dakle, ako na početku programa
imate varijablu \verb+n+ koja ima cjelobrojnu vrijednost --
pozeljno je da ta varijabla i dalje u programu sadrzi cijele
brojeve.

\section{Liste}

Jednostavni tipovi podataka kao što su brojevi i stringovi
često nisu dovoljni (ili pogodni) za rješavanje mnogih
problema. Ako imamo neki jako dug program, često će nam se
desiti da broj varijabli postane prevelik. Pretpostavimo npr. da
imamo program u kojemu se mora raditi s jako velikom količinom
podataka; npr. broj učenika neke škole.

\dots

\section{Riječnici}

\section{Datoteka}

\section{Ostali slozeni tipovi podataka}
\chapter{Moduli}



\chapter{Osnove objektno orijentiranog programiranja}

Pretpostavite da vam treba aplikacija koja izračunava statistike ocjena učenika u nekom 
razredu. Prvo i osnovno, ta aplikacija treba nekako čuvati sve podatke o svakom ueniku.
To bi, na primjer, mogli na sljedeći način:

\begin{lstlisting}
ana = { 'hrvatski': 4, 'matematika': 3, 'kemija': 2 }
\end{lstlisting}

I sad bi podaci učenicima nekog razreda mogli biti sačuvani u python programu ovako:

\begin{lstlisting}
ana = { 'hrvatski': 4, 'matematika': 3, 'kemija': 2 }
josip = { 'hrvatski': 5, 'matematika': 2, 'kemija': 3 }
ivana = { 'hrvatski': 5, 'matematika': 5, 'kemija': 5 }
ucenici_razreda = [ ana, josip, ivana ]
\end{lstlisting}

Dakle, u listi čuvam spisak svih učenika, a svaki učenik mi je \emph+dictionary+ \TODO
sa predmetima i njihovim ocjenama. 

Sad, kad mi treba prosjek ocjena za hrvatski jezik, izračunao bih to ovako:

\begin{lstlisting}
zbroj = 0.
for ucenik in ucenici_razreda:
	ocjena_iz_hrvatskog = ucenik[ 'hrvatski' ]
	zbroj = zbroj + ocjena_iz_hrvatskog
print zbroj / len( ucenici_razreda )
\end{lstlisting}

Međutim, što ako uz svakog učenika trebamo čuvati i grad i godinu rođenja, te
imena roditelja.

\begin{lstlisting}
ana = {
	'ocjene': { 'hrvatski': 4, 'matematika': 3, 'kemija': 2 },
	'grad_rodjenja': 'Split',
	'godina_rodjenja': 1999
}
\end{lstlisting}

Ili, na primjer, da dodamo još i slobne aktivnosti. 

\begin{lstlisting}
ana = {
	'ocjene': { 'hrvatski': 4, 'matematika': 3, 'kemija': 2 },
	'grad_rodjenja': 'Split',
	'godina_rodjenja': 1999,
	'slobodne_aktivnosti': ( 'Odbojka', 'Radio grupa' )
}
\end{lstlisting}

Prikaz te "strukture" se ovdje znatno zakomplicirao, no uspijeli smo u pythonu
prikazati jednu kompleksnu strukturu pomoću kombinacije listi, brojeva i stringova.
Međutim, ono čemuvi težimo kad radimo s takvim složenim strukturama je da nam
zapis bude što sliniji ljudskom jeziku. Pa tako, kao što mi neki podatak može
biti \emph+lista+, \emph+string+ ili \emph+broj+, zašto ne bi imali tip podatka
koji bi se zvao \emph+Ucenik+?

Ljudskim jezikom bi moj primjer izgledao ovako: \emph{Neka je Ana učenica. Ana
ima ocjene ih hrvatskog 4, iz matematike 3, iz kemije 2. Ana je rođena u Splitu.
Rođena je 1999 godine. Ona ide na sljedeće slobodne aktivnosti: odbojka i radio
grupa}.

\begin{lstlisting}
ana = Ucenik()
ana.grad_rodjenja = 'Split'
ana.slobodne_aktivnosti = ( 'Odbojka', 'Radio grupa' )
ana.ocjene = { 'hrvatski': 4, 'matematika': 3, 'kemija': 2 },
\end{lstlisting}

\section{Klase i objekti}

\section{Atributi i metode}

\section{Privatno i javno}

\section{Što znači "self" kod metoda}

\section{Razlika između objektnog i funkcionalnog pristupa}

